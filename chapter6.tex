\setcounter{chapter}{5}
\chapter{Compact Imbeddings of Sobolev Spaces}

\section{The Rellich-Kondrachov Theorem}

\begin{para}[Restricted Imbeddings]
  Let $\Omega$ be a domain in $\mathbb{R}^n$ and let $\Omega_0$ be a sundomain of $\Omega$.
  Let $X(\Omega)$ denote any of the possible target spaces for imbeddings of $W^{m,p}(\Omega)$, that is,
  $X(\Omega)$ is a space of the form $C_B^j(\Omega)$, $C^j(\overline{\Omega})$,
  $C^{j,\lambda}(\overline{\Omega})$, $L^q(\Omega_k)$, or $W^{j,q}(\Omega_k)$, where $\Omega_k$, $1\leq k\leq n$,
  is the intersection of $\Omega$ with a $k$-dimensional plane in $\mathbb{R}^n$. Since the linear restriction 
  operator $i_{\Omega_0}: u\to u|_{\Omega_0}$ is bounded from $X(\Omega)$ into $X(\Omega_0)$
  (in fact $\|i_{\Omega_0}u\|_{X(\Omega_0)} \leq \|u\|_{X(\Omega)}$) any imbeddings of the form
  \begin{equation}\label{eq:6.1}
    W^{m,p}(\Omega) \to X(\Omega)
  \end{equation}
  can be composed with this restriction to yield the imbedding
  \begin{equation}\label{eq:6.2}
    W^{m,p}(\Omega) \to X(\Omega_0)
  \end{equation}
  and \eqref{eq:6.2} has imbedding constant no larger than \eqref{eq:6.1}.
\end{para}


\begin{para}[Compact Imbeddings]
  Recall that a set $A$ in a normed space is precompact if every sequence of points in $A$ has a subsequence
  converging in norm to an element of the space. An operator between normed spaces is called compact if it
  maps bounded sets into precompact sets, and is called completely continuous if it is continuous and compact.
  (See Paragraph~1.24; for linear operators compactness and complete continuity are equivalent.)
  In this chapter we are concerned with the compactness of imbedding operators which are continuous whenever
  they exist, and so are completely continuous whenever they are compact.

  If $\Omega$ satisfies the hypothesis of the Sobolev imbedding Theorem~4.12 and if $\Omega_0$ is a bounded
  subset of $\Omega$, then, with the exception of certain extreme cases, all the restricted
  imbeddings \eqref{eq:6.1} corresponding to imbeddings asserted in Theorem~4.12 are compact.
  The most important of these compact imbedding results originated in a lemma of Rellich [Re] and was proved
  specifically for Sobolev spaces by Kondrachov [K]. Such compact imbeddings have many important applications
  in analysis, especially to showing that linear elliptic partial differential equations defined over
  bounded domains have discrete spectra. See, for example, [EE] and [ET] for such applications and further
  refinements.
\end{para}

We summarize the various compact imbeddings of $W^{m,p}(\Omega)$ in the following theorem

\begin{theorem}[The Rellich-Kondrachov Theorem]
  Let $\Omega$ be a domain in $\mathbb{R}^n$, let $\Omega_0$ be a bounded subdomain of $\Omega$, and let
  $\Omega_0^k$ be the intersection of $\Omega_0$ with a $k$-dimensional plane in $\mathbb{R}^n$.
  Let $j\geq 0$ and $m\geq 1$ be integers, and let $1\leq p<\infty$.
  \begin{enumerate}[label = \textbf{PART \Roman*},
                    labelindent = 0pt,
                    labelsep = 1em,
                    leftmargin = 0pt,
                    labelwidth = 4em,
                    itemindent = *,
                    align = left]
    \item If $\Omega$ satisfies the cone condition and $mp<n$, then the following imbeddings are compact:
      \begin{align}
        & W^{j+m,p}(\Omega) \to W^{j,q}(\Omega_0^k) & & \text{if } 0 < n-mp < k \leq n \text{ and } 
          1\leq q < kp/(n-mp), \label{eq:6.3} \\
        & W^{j+m,p}(\Omega) \to W^{j,q}(\Omega_0^k) & & \text{if } n = mp, 1\leq k \leq n \text{ and }
          1\leq q <\infty. \label{eq:6.4}
      \end{align}
    \item If $\Omega$ satisfies the cone condition and $mp > n$, then the following imbeddings are compact:
      \begin{align}
        & W^{j+m,p}(\Omega) \to C_B^j(\Omega_0) \label{eq:6.5} \\
        & W^{j+m,p}(\Omega) \to W^{j,q}(\Omega_0^k) \qquad \text{if } 1\leq q < \infty. \label{eq:6.6}
      \end{align}
    \item If $\Omega$ satisfies the strong local Lipschitz condition, then the following imbeddings are compact:
      \begin{align}
        W^{j+m,p}(\Omega) \to C^j(\overline{\Omega_0}) \qquad & \text{if } mp > n, \label{eq:6.7} \\
        W^{j+m,p}(\Omega) \to C^{j,\lambda}(\overline{\Omega_0}) \qquad & \text{if } mp > n \geq (m-1)p
          \text{ and } 0 < \lambda < m - (n/p). \label{eq:6.8}
      \end{align}
    \item If $\Omega$ is an arbitrary domain in $\mathbb{R}^n$, the imbeddings \eqref{eq:6.3}--\eqref{eq:6.8}
      are compact provided $W^{j+m,p}(\Omega)$ is replaced by $W^{j+m,p}_0(\Omega)$.
  \end{enumerate}
\end{theorem}