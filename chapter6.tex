\setcounter{chapter}{5}
\chapter{Compact Imbeddings of Sobolev Spaces}

\section{The Rellich-Kondrachov Theorem}

\begin{para}[Restricted Imbeddings]
  Let $\Omega$ be a domain in $\mathbb{R}^n$ and let $\Omega_0$ be a sundomain of $\Omega$.
  Let $X(\Omega)$ denote any of the possible target spaces for imbeddings of $W^{m,p}(\Omega)$, that is,
  $X(\Omega)$ is a space of the form $C_B^j(\Omega)$, $C^j(\overline{\Omega})$,
  $C^{j,\lambda}(\overline{\Omega})$, $L^q(\Omega_k)$, or $W^{j,q}(\Omega_k)$, where $\Omega_k$, $1\leq k\leq n$,
  is the intersection of $\Omega$ with a $k$-dimensional plane in $\mathbb{R}^n$. Since the linear restriction 
  operator $i_{\Omega_0}: u\to u|_{\Omega_0}$ is bounded from $X(\Omega)$ into $X(\Omega_0)$
  (in fact $\|i_{\Omega_0}u\|_{X(\Omega_0)} \leq \|u\|_{X(\Omega)}$) any imbeddings of the form
  \begin{equation}\label{eq:6.1}
    W^{m,p}(\Omega) \to X(\Omega)
  \end{equation}
  can be composed with this restriction to yield the imbedding
  \begin{equation}\label{eq:6.2}
    W^{m,p}(\Omega) \to X(\Omega_0)
  \end{equation}
  and \eqref{eq:6.2} has imbedding constant no larger than \eqref{eq:6.1}.
\end{para}


\begin{para}[Compact Imbeddings]
  Recall that a set $A$ in a normed space is precompact if every sequence of points in $A$ has a subsequence
  converging in norm to an element of the space. An operator between normed spaces is called compact if it
  maps bounded sets into precompact sets, and is called completely continuous if it is continuous and compact.
  (See Paragraph~1.24; for linear operators compactness and complete continuity are equivalent.)
  In this chapter we are concerned with the compactness of imbedding operators which are continuous whenever
  they exist, and so are completely continuous whenever they are compact.

  If $\Omega$ satisfies the hypothesis of the Sobolev imbedding Theorem~4.12 and if $\Omega_0$ is a bounded
  subset of $\Omega$, then, with the exception of certain extreme cases, all the restricted
  imbeddings \eqref{eq:6.1} corresponding to imbeddings asserted in Theorem~4.12 are compact.
  The most important of these compact imbedding results originated in a lemma of Rellich [Re] and was proved
  specifically for Sobolev spaces by Kondrachov [K]. Such compact imbeddings have many important applications
  in analysis, especially to showing that linear elliptic partial differential equations defined over
  bounded domains have discrete spectra. See, for example, [EE] and [ET] for such applications and further
  refinements.
\end{para}

We summarize the various compact imbeddings of $W^{m,p}(\Omega)$ in the following theorem

\begin{theorem}[The Rellich-Kondrachov Theorem]
  Let $\Omega$ be a domain in $\mathbb{R}^n$, let $\Omega_0$ be a bounded subdomain of $\Omega$, and let
  $\Omega_0^k$ be the intersection of $\Omega_0$ with a $k$-dimensional plane in $\mathbb{R}^n$.
  Let $j\geq 0$ and $m\geq 1$ be integers, and let $1\leq p<\infty$.
  \begin{enumerate}[label = \textbf{PART \Roman*},
                    labelindent = 0pt,
                    labelsep = 5pt,
                    leftmargin = 0pt,
                    labelwidth = 3em,
                    itemindent = *,
                    align = left]
    \item If $\Omega$ satisfies the cone condition and $mp<n$, then the following imbeddings are compact:
      \begin{align}
        & W^{j+m,p}(\Omega) \to W^{j,q}(\Omega_0^k) & & \text{if } 0 < n-mp < k \leq n \text{ and } 
          1\leq q < kp/(n-mp), \label{eq:6.3} \\
        & W^{j+m,p}(\Omega) \to W^{j,q}(\Omega_0^k) & & \text{if } n = mp, 1\leq k \leq n \text{ and }
          1\leq q <\infty. \label{eq:6.4}
      \end{align}
    \item If $\Omega$ satisfies the cone condition and $mp > n$, then the following imbeddings are compact:
      \begin{align}
        & W^{j+m,p}(\Omega) \to C_B^j(\Omega_0) \label{eq:6.5} \\
        & W^{j+m,p}(\Omega) \to W^{j,q}(\Omega_0^k) \qquad \text{if } 1\leq q < \infty. \label{eq:6.6}
      \end{align}
    \item If $\Omega$ satisfies the strong local Lipschitz condition, then the following imbeddings are compact:
      \begin{align}
        W^{j+m,p}(\Omega) \to C^j(\overline{\Omega_0}) \qquad & \text{if } mp > n, \label{eq:6.7} \\
        W^{j+m,p}(\Omega) \to C^{j,\lambda}(\overline{\Omega_0}) \qquad & \text{if } mp > n \geq (m-1)p
          \text{ and } 0 < \lambda < m - (n/p). \label{eq:6.8}
      \end{align}
    \item If $\Omega$ is an arbitrary domain in $\mathbb{R}^n$, the imbeddings \eqref{eq:6.3}--\eqref{eq:6.8}
      are compact provided $W^{j+m,p}(\Omega)$ is replaced by $W^{j+m,p}_0(\Omega)$.
  \end{enumerate}
\end{theorem}


\begin{remarks}
  \begin{enumerate}[1.]
    \item Note that if $\Omega$ is bounded, we may have $\Omega_0 = \Omega$ in the statement of the theorem.
    \item If $X$, $Y$, and $Z$ are spaces for which we have the imbeddings $X\to Y$ and $Y\to Z$,
      and one of these imbeddings is compact, then the composite imbedding $X\to Z$ is compact.
      Thus, for example, if $Y\to Z$ is compact, then any sequence $\{u_j\}$ bounded in $X$ will
      be bounded in $Y$ and will therefore have a subsequence $\{u_j'\}$ convergent in $Z$.
    \item Since the extension operator $u\to\tilde{u}$, where $\tilde{u}(x) = u(x)$ if $x\in\Omega$
      and $\tilde{u}(x) = 0$ if $x\notin \Omega$,
      defines an imbedding $W_0^{j+m,p}(\Omega) \to W^{j+m,p}(\mathbb{R}^n)$ by Lemma~3.27, Part IV of Theorem~6.3
      follows from application of Parts I--III to $\mathbb{R}^n$.
    \item In proving the compactness of any of the imbeddings \eqref{eq:6.3}--\eqref{eq:6.8} it is
      sufficient to consider only the case $j=0$. For $j\geq 1$ and $\{u_i\}$ a bounded sequence in
      $W^{j+m,p}(\Omega)$ it is clear that $\{D^\alpha u_i\}$ is bounded in $W^{m,p}(\Omega)$ for each $\alpha$
      such that $|\alpha|\leq j$. Hence $\{D^\alpha u_i|_{\Omega_0^k}\}$ is precompact in $L^q(\Omega_0^k)$
      with $q$ specified as in \eqref{eq:6.3}. It is possible, therefore, to select (by finite induction)
      a subsequence $\{u_i'\}$ of $\{u_i\}$ for which $\{D^\alpha u_i'|_{\Omega_0^k}\}$
      converges in $L^q(\Omega_0^k)$ for each $\alpha$ such that $|\alpha|\leq j$.
      Thus $\{u_i'|_{\Omega_0^k}\}$ converges in $W_0^{j,q}(\Omega_0^k)$ and \eqref{eq:6.3} is compact.
    \item Since $\Omega_0$ is bounded, $C_B^0(\Omega_0^k)\to L^q(\Omega_0^k)$ for $1\leq q\leq \infty$;
      in fact $\|u\|_{0,q,\Omega_0^k} \leq \|u\|_{C_B^0(\Omega_0^k)} \bigl[\vol (\Omega_0^k)\bigr]^{1/q}$.
      Thus the compactness of \eqref{eq:6.6} (for $j=0$) follows from that of \eqref{eq:6.5}.
    \item For the purpose of proving Theorem 6.3 the bounded subdomain $\Omega_0$ of $\Omega$ may be assumed
      to satisfy the cone condition in $\Omega$ does. If $C$ is a finite cone determining the cone condition
      for $\Omega$, let $\widetilde{\Omega}$ be the union of all finite cones congruent to $C$, contained in
      $\Omega$ and having nonempty intersection with $\Omega_0$.
      Then $\Omega_0\subset\widetilde{\Omega}\subset\Omega$ and $\widetilde{\Omega}$ is bounded and satisfies
      the cone condition. If $W^{m,p}(\Omega)\to X(\widetilde{\Omega})$ is compact, then so is
      $W^{m,p}\to X(\Omega_0)$ by restriction.
  \end{enumerate}
\end{remarks}

\begin{para}[Proof of Theorem 6.3, Part III]
  If $mp > n\geq (m-1)p$ and if $0 < \lambda < m-(n/p)$, then there exists $\mu$ such that
  $\lambda < \mu < m-(n/p)$. Since $\Omega_0$ is bounded,
  the imbedding $C^{0,\mu}(\overline{\Omega_0})\to C^{0,\lambda}(\overline{\Omega_0})$ is compact by Theorem~1.34.
  Since $W^{m,p}(\Omega)\to C^{0,\mu}(\overline{\Omega})\to C^{0,\mu}(\overline{\Omega_0})$ by Theorem~4.12
  and restriction, imbedding \eqref{eq:6.8} is compact for $j=0$ by Remark 6.4(2).

  If $mp>n$, let $j^*$ be the nonnegative integer satisfying the inequalities
  $(m-j^*)p > n \geq (m-j^*-1)p$. Then we have the imbedding chain
  \begin{equation}\label{eq:6.9}
    W^{m,p}(\Omega) \to W^{m-j^*,p}(\Omega) \to C^{0,\mu}(\overline{\Omega_0}) \to C(\overline{\Omega_0})
  \end{equation}
  where $0 < \mu < m-j^*-(n/p)$. The last imbeddings in \eqref{eq:6.9} is compact by Theorem 1.34.
  Thus \eqref{eq:6.7} is compact for $j=0$.
\end{para}


\begin{para}[Proof of Theorem 6.3, Part II]
  As noted in Remark 6.4(6), $\Omega_0$ may be assumed to satisfy the cone condition.
  Since $\Omega_0$ is bounded it can, by Lemma~4.22 be written as a finite union,
  $\Omega_0 = \bigcup_{k=1}^M \Omega_k$, where each $\Omega_k$ satisfies the strong local Lipschitz
  condition. If $mp>n$, then
  \[W^{m,p}(\Omega) \to W^{m,p}(\Omega_k) \to C(\overline{\Omega_k}),\]
  the latter imbedding being compact as proved above. If $\{u_i\}$ is a sequence bounded in $W^{m,p}(\Omega)$,
  we may select by finite induction on $k$ a subsequence $\{u_i'\}$ whose restriction to $\Omega_k$
  converges in $C(\overline{\Omega_k})$ for each $k$, $1\leq k\leq M$.
  But this subsequence then converges in $C_B^0(\Omega_0)$, so proving that \eqref{eq:6.5} is compact
  for $j=0$. Therefore \eqref{eq:6.6} is also compact by Remark 6.4(5).
\end{para}