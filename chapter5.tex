\chapter{Interpolation, Extension And Approximation Theorems}

\section{Interpolation on Order of Smoothness}

\begin{para}
  We consider the problem of determining upper bounds for $L^p$ norms of
  derivatives $D^\beta u$, $0<|\beta|<m$, of functions in $W^{m,p}(\Omega)$
  in terms of the $L^p$ norms of $u$ and its partial derivatives of order $m$.
  Such estimates are conveniently expressed in terms of the seminorms $|\cdot|_{j,p}$
  defined in Paragraph~4.29. Theorem~5.2 below provides such an estimate for
  the seminorm $|u|_{j,p}$ in terms of $|u|_{m,p}$ and $\|u\|_p$, as well as
  some elementary consequences of this estimate. Such estimates arose
  in the work of Ehrling [E], Nirenberg [Nr1, Nr2], Gagliardo [Ga1, Ga2],
  and Browder [Br1, Br2], and were frequently proved under the assumption
  that $\Omega$ satisfies the uniform cone condition, at least if $\Omega$
  is unbounded. However, we will prove Theorem~5.2 assuming only the cone
  condition. In fact, even the weak cone condition is sufficient for the proof,
  as is shown in [AF1].
\end{para}


\begin{theorem}
  Let $\Omega$ be a domain in $\mathbb{R}^n$ satisfying the cone condition.
  For each $\epsilon_0>0$ there exist finite constants $K$ and $K'$, each
  depending on $n$, $m$, $p$, $\epsilon_0$ and the dimensions of the cone $C$
  providing the cone condition for $\Omega$ such that if $0<\epsilon\leq\epsilon_0$,
  $0\leq j\leq m$, and $u\in W^{m,p}(\Omega)$, then
  \begin{align}
    |u|_{j,p} & \leq K\bigl(\epsilon |u|_{m,p} + \epsilon^{-j/(m-j)}\|u\|_p\bigr), \label{eq:5.1} \\
    \|u\|_{j,p} & \leq K'\bigl(\epsilon\|u\|_{m,p} + \epsilon^{-j/(m-j)}\|u\|_p\bigr), \label{eq:5.2} \\
    \|u\|_{j,p} & \leq 2K' \|u\|_{m,p}^{j/m} \|u\|_p^{(m-j)/m}. \label{eq:5.3}
  \end{align}
\end{theorem}


\begin{para}
  Inequality~\eqref{eq:5.2} follows from repeated applications of~\eqref{eq:5.1},
  and~\eqref{eq:5.3} by setting $\epsilon_0=1$ in \eqref{eq:5.2} and choosing
  $\epsilon$ in \eqref{eq:5.2} so that the two terms on the right side are equal.
  Furthermore, \eqref{eq:5.1} holds when $\epsilon<\epsilon_0$ if it holds for
  $\epsilon<\epsilon_1$ for any specific positive $\epsilon_1$; to see this just
  replace $\epsilon$ by $\epsilon\epsilon_1/\epsilon_0$ and suitably adjust $K$.
  Thus we need only prove \eqref{eq:5.1}, and that for just one value of $\epsilon_0$.

  We carry out the proof in three lemmas. The first develops a one-dimensional
  version for the case $m=2$, $j=1$. The second establishes \eqref{eq:5.1}
  for $m=2$, $j=1$ for general $\Omega$ satisfying the cone condition. The third
  shows that \eqref{eq:5.1} is valid for general $m\geq 2$ and $1\leq j\leq m-1$
  whenever the case $m=2$, $j=1$ is known to hold.
\end{para}


\begin{lemma}
  If $\rho>0$, $1\leq p<\infty$, $K_p=2^{p-1} 9^p$, and $g\in C^2([0,\rho])$, then
  \begin{equation}\label{eq:5.4}
    |g'(0)|^p \leq \frac{K_p}{\rho} \biggl(\rho^p \int_0^\rho |g''(t)|^p
      \d t + \rho^{-p} \int_0^\rho |g(t)|^p \d t\biggr).
  \end{equation}
\end{lemma}

\begin{proof}
  Let $f\in C^2([0,1])$, let $x\in [0,1/3]$, and let $y\in [2/3,1]$.
  By the mean-value theorem there exists $z\in (x,y)$ such that
  \[ |f'(z)| = \biggl|\frac{f(y)-f(x)}{y-x}\biggr| \leq 3|f(x)| + 3|f(y)|. \]
  Thus
  \begin{align*}
    |f'(0)|
    & = \biggl|f'(z) - \int_0^z f''(t) \d t\biggr| \\
    & \leq 3|f(x)| + 3|f(y)| + \int_0^1 |f''(t)| \d t.
  \end{align*}
  Integration of $x$ over $[0,1/3]$ and $y$ over $[2/3,1]$ yields
  \[ \frac19 |f'(0)| \leq \int_0^{1/3} |f(x)| \d x
      + \int_{2/3}^1 |f(y)|\d y + \frac19 \int_0^1 |f''(t)| \d t. \]
  For $p\geq 1$ we therefore have (using H\"older's inequality if $p>1$)
  \[ |f'(0)|^p \leq K_p \biggl(\int_0^1 |f''(t)|^p \d t
      + \int_0^1 |f(t)|^p \d t\biggr), \]
  where $K_p = 2^{p-1}9^p$.

  Inequality~\eqref{eq:5.4} now follows by substituting $f(t) = g(\rho t)$.
\end{proof}


\begin{lemma}
  If $1\leq p<\infty$ and the domain $\Omega\subset\mathbb{R}^n$ satisfies
  the cone condition, then there exists a constant $K$ depending on $n$, $p$,
  and the height $\rho_0$ and aperture angle $\kappa$ of the cone $C$ providing
  the cone condition for $\Omega$ such that for all $\epsilon$, $0<\epsilon\leq\rho_0$
  and all $u\in W^{2,p}(\Omega)$ we have
  \begin{equation}\label{eq:5.5}
    |u|_{1,p} \leq K (\epsilon |u|_{2,p} + \epsilon^{-1} \|u\|_p).
  \end{equation}
\end{lemma}

\begin{proof}
  Let $\Sigma = \{\sigma\in \mathbb{R}^n : |\sigma|=1\}$ be the unit sphere
  in $\mathbb{R}^n$ with volume element $\d\sigma$ and $(n-1)$-volume
  $K_0 = K_0(n) = \int_\Sigma \d\sigma$. If $x\in\Omega$ let $\sigma_x$
  be the unit vector in the direction of the axis of a cone $C_x\subset\Omega$
  congruent to $C$ and having vertex at $x$, and let $\Sigma_x = \{\sigma\in\Sigma:
  \angle (\sigma,\sigma_x)\leq\kappa/2\}$.

  Let $u\in C^\infty(\Omega)$. If $x\in\Omega$, $\sigma\in\Sigma_x$,
  and $0<\rho\leq\rho_0$, then
  \[ |\sigma\cdot\nabla u(x)|^p \leq \frac{K_p}{\rho} I(\rho,p,u,x,\sigma), \]
  where
  \[ I(\rho,p,u,x,\sigma) = \rho^p \int_0^\rho |D_t^2 u(x+t\sigma)|^p \d t
      + \rho^{-p} \int_0^\rho |u(x+t\sigma)|^p \d t. \]
      There exists a constant $K_1=K_1(n,p,\kappa)$ such that
  \[\int_\Sigma |\sigma\cdot\nabla u(x)|^p \d\sigma
    \geq \int_{\Sigma_x} |\sigma\cdot\nabla u(x)|^p \d\sigma
    \geq K_1 |\nabla u(x)|^p.\]
  Accordingly,
  \[ \int_\Omega |\nabla u(x)|^p \d x
      \leq \frac{K_p}{K_1\rho} \int_\Sigma \d\sigma \int_\Omega I(\rho,p,u,x,\sigma) \d x. \]
  In order to estimate the inner integral on the right, regard $u$
  and its derivatives as extended to all of $\mathbb{R}^n$ so as to be
  identically zero outside $\Omega$. For simplicity, we suppose
  $\sigma = e_n = (0,\ldots,0,1)$ and write $x=(x',x_n)$ with $x'\in\mathbb{R}^{n-1}$.
  We have
  \begin{align*}
    \int_\Omega & I(\rho,p,u,x,e_n) \d x \\
    & = \int_{\mathbb{R}^{n-1}} \d x' \int_{-\infty}^\infty \d x_n
        \int_0^\rho \bigl(\rho^p |D_n^2 u(x',x_n+t)|^p + \rho^{-p} |u(x',x_n+t)|^p\bigr) \d t \\
    & = \int_{\mathbb{R}^{n-1}} \d x' \int_0^\rho \d t
        \int_{-\infty}^\infty (\rho^p |D_n^2 u(x)|^p + \rho^{-p} |u(x)|^p) \d x_n \\
    & \leq \rho \int_\Omega (\rho^p |D_n^2 u(x)|^p + \rho^{-p} |u(x)|^p) \d x.
  \end{align*}
  In general, for $\sigma\in\Sigma$
  \[ \int_\Omega I(\rho,p,u,x,\sigma) \d x \leq \rho
        \int_\Omega (\rho^p |u|_{2,p}^p + \rho^{-p}\|u\|_p^p) \d x, \]
\end{proof}