\chapter{Interpolation, Extension And Approximation Theorems}

\section{Interpolation on Order of Smoothness}

\begin{para}
  We consider the problem of determining upper bounds for $L^p$ norms of
  derivatives $D^\beta u$, $0<|\beta|<m$, of functions in $W^{m,p}(\Omega)$
  in terms of the $L^p$ norms of $u$ and its partial derivatives of order $m$.
  Such estimates are conveniently expressed in terms of the seminorms $|\cdot|_{j,p}$
  defined in Paragraph~4.29. Theorem~5.2 below provides such an estimate for
  the seminorm $|u|_{j,p}$ in terms of $|u|_{m,p}$ and $\|u\|_p$, as well as
  some elementary consequences of this estimate. Such estimates arose
  in the work of Ehrling [E], Nirenberg [Nr1, Nr2], Gagliardo [Ga1, Ga2],
  and Browder [Br1, Br2], and were frequently proved under the assumption
  that $\Omega$ satisfies the uniform cone condition, at least if $\Omega$
  is unbounded. However, we will prove Theorem~5.2 assuming only the cone
  condition. In fact, even the weak cone condition is sufficient for the proof,
  as is shown in [AF1].
\end{para}


\begin{theorem}
  Let $\Omega$ be a domain in $\mathbb{R}^n$ satisfying the cone condition.
  For each $\epsilon_0>0$ there exist finite constants $K$ and $K'$, each
  depending on $n$, $m$, $p$, $\epsilon_0$ and the dimensions of the cone $C$
  providing the cone condition for $\Omega$ such that if $0<\epsilon\leq\epsilon_0$,
  $0\leq j\leq m$, and $u\in W^{m,p}(\Omega)$, then
  \begin{align}
    |u|_{j,p} & \leq K\bigl(\epsilon |u|_{m,p} + \epsilon^{-j/(m-j)}\|u\|_p\bigr), \label{eq:5.1} \\
    \|u\|_{j,p} & \leq K'\bigl(\epsilon\|u\|_{m,p} + \epsilon^{-j/(m-j)}\|u\|_p\bigr), \label{eq:5.2} \\
    \|u\|_{j,p} & \leq 2K' \|u\|_{m,p}^{j/m} \|u\|_p^{(m-j)/m}. \label{eq:5.3}
  \end{align}
\end{theorem}


\begin{para}
  Inequality~\eqref{eq:5.2} follows from repeated applications of~\eqref{eq:5.1},
  and~\eqref{eq:5.3} by setting $\epsilon_0=1$ in \eqref{eq:5.2} and choosing
  $\epsilon$ in \eqref{eq:5.2} so that the two terms on the right side are equal.
  Furthermore, \eqref{eq:5.1} holds when $\epsilon<\epsilon_0$ if it holds for
  $\epsilon<\epsilon_1$ for any specific positive $\epsilon_1$; to see this just
  replace $\epsilon$ by $\epsilon\epsilon_1/\epsilon_0$ and suitably adjust $K$.
  Thus we need only prove \eqref{eq:5.1}, and that for just one value of $\epsilon_0$.

  We carry out the proof in three lemmas. The first develops a one-dimensional
  version for the case $m=2$, $j=1$. The second establishes \eqref{eq:5.1}
  for $m=2$, $j=1$ for general $\Omega$ satisfying the cone condition. The third
  shows that \eqref{eq:5.1} is valid for general $m\geq 2$ and $1\leq j\leq m-1$
  whenever the case $m=2$, $j=1$ is known to hold.
\end{para}


\begin{lemma}
  If $\rho>0$, $1\leq p<\infty$, $K_p=2^{p-1} 9^p$, and $g\in C^2([0,\rho])$, then
  \begin{equation}\label{eq:5.4}
    |g'(0)|^p \leq \frac{K_p}{\rho} \biggl(\rho^p \int_0^\rho |g''(t)|^p
      \d t + \rho^{-p} \int_0^\rho |g(t)|^p \d t\biggr).
  \end{equation}
\end{lemma}

\begin{proof}
  Let $f\in C^2([0,1])$, let $x\in [0,1/3]$, and let $y\in [2/3,1]$.
  By the mean-value theorem there exists $z\in (x,y)$ such that
  \[ |f'(z)| = \biggl|\frac{f(y)-f(x)}{y-x}\biggr| \leq 3|f(x)| + 3|f(y)|. \]
  Thus
  \begin{align*}
    |f'(0)|
    & = \biggl|f'(z) - \int_0^z f''(t) \d t\biggr| \\
    & \leq 3|f(x)| + 3|f(y)| + \int_0^1 |f''(t)| \d t.
  \end{align*}
  Integration of $x$ over $[0,1/3]$ and $y$ over $[2/3,1]$ yields
  \[ \frac19 |f'(0)| \leq \int_0^{1/3} |f(x)| \d x
      + \int_{2/3}^1 |f(y)|\d y + \frac19 \int_0^1 |f''(t)| \d t. \]
  For $p\geq 1$ we therefore have (using H\"older's inequality if $p>1$)
  \[ |f'(0)|^p \leq K_p \biggl(\int_0^1 |f''(t)|^p \d t
      + \int_0^1 |f(t)|^p \d t\biggr), \]
  where $K_p = 2^{p-1}9^p$.

  Inequality~\eqref{eq:5.4} now follows by substituting $f(t) = g(\rho t)$.
\end{proof}


\begin{lemma}
  If $1\leq p<\infty$ and the domain $\Omega\subset\mathbb{R}^n$ satisfies
  the cone condition, then there exists a constant $K$ depending on $n$, $p$,
  and the height $\rho_0$ and aperture angle $\kappa$ of the cone $C$ providing
  the cone condition for $\Omega$ such that for all $\epsilon$, $0<\epsilon\leq\rho_0$
  and all $u\in W^{2,p}(\Omega)$ we have
  \begin{equation}\label{eq:5.5}
    |u|_{1,p} \leq K (\epsilon |u|_{2,p} + \epsilon^{-1} \|u\|_p).
  \end{equation}
\end{lemma}

\begin{proof}
  Let $\Sigma = \{\sigma\in \mathbb{R}^n : |\sigma|=1\}$ be the unit sphere
  in $\mathbb{R}^n$ with volume element $\d\sigma$ and $(n-1)$-volume
  $K_0 = K_0(n) = \int_\Sigma \d\sigma$. If $x\in\Omega$ let $\sigma_x$
  be the unit vector in the direction of the axis of a cone $C_x\subset\Omega$
  congruent to $C$ and having vertex at $x$, and let $\Sigma_x = \{\sigma\in\Sigma:
  \angle (\sigma,\sigma_x)\leq\kappa/2\}$.

  Let $u\in C^\infty(\Omega)$. If $x\in\Omega$, $\sigma\in\Sigma_x$,
  and $0<\rho\leq\rho_0$, then
  \[ |\sigma\cdot\nabla u(x)|^p \leq \frac{K_p}{\rho} I(\rho,p,u,x,\sigma), \]
  where
  \[ I(\rho,p,u,x,\sigma) = \rho^p \int_0^\rho |D_t^2 u(x+t\sigma)|^p \d t
      + \rho^{-p} \int_0^\rho |u(x+t\sigma)|^p \d t. \]
      There exists a constant $K_1=K_1(n,p,\kappa)$ such that
  \[\int_\Sigma |\sigma\cdot\nabla u(x)|^p \d\sigma
    \geq \int_{\Sigma_x} |\sigma\cdot\nabla u(x)|^p \d\sigma
    \geq K_1 |\nabla u(x)|^p.\]
  Accordingly,
  \[ \int_\Omega |\nabla u(x)|^p \d x
      \leq \frac{K_p}{K_1\rho} \int_\Sigma \d\sigma \int_\Omega I(\rho,p,u,x,\sigma) \d x. \]
  In order to estimate the inner integral on the right, regard $u$
  and its derivatives as extended to all of $\mathbb{R}^n$ so as to be
  identically zero outside $\Omega$. For simplicity, we suppose
  $\sigma = e_n = (0,\ldots,0,1)$ and write $x=(x',x_n)$ with $x'\in\mathbb{R}^{n-1}$.
  We have
  \begin{align*}
    \int_\Omega & I(\rho,p,u,x,e_n) \d x \\
    & = \int_{\mathbb{R}^{n-1}} \d x' \int_{-\infty}^\infty \d x_n
        \int_0^\rho \bigl(\rho^p |D_n^2 u(x',x_n+t)|^p + \rho^{-p} |u(x',x_n+t)|^p\bigr) \d t \\
    & = \int_{\mathbb{R}^{n-1}} \d x' \int_0^\rho \d t
        \int_{-\infty}^\infty (\rho^p |D_n^2 u(x)|^p + \rho^{-p} |u(x)|^p) \d x_n \\
    & \leq \rho \int_\Omega (\rho^p |D_n^2 u(x)|^p + \rho^{-p} |u(x)|^p) \d x.
  \end{align*}
  In general, for $\sigma\in\Sigma$
  \[ \int_\Omega I(\rho,p,u,x,\sigma) \d x \leq \rho
        \int_\Omega (\rho^p |u|_{2,p}^p + \rho^{-p}\|u\|_p^p) \d x, \]
  and since $|D_j(u)|\leq |\nabla u|$ and the measure of $\Sigma$ is $K_0$,
  \[ |u|_{1,p}^p \leq \frac{nK_pK_0}{K_1} \bigl(\rho^p |u|_{2,p}^p + \rho^{-p} \|u\|_p^p\bigr). \]
  Inequality~\eqref{eq:5.5} now follows by taking $p$th roots, replacing
  $\rho$ with $\epsilon$, and noting that $C^\infty(\Omega)$ is dense in $W^{2,p}(\Omega)$.
\end{proof}


\begin{lemma}
  Let $m\geq 2$, let $0<\delta_0<\infty$, and let
  $\epsilon_0 = \min\{\delta_0,\delta_0^2,\ldots,\delta_0^{m-1}\}$.
  Suppose that for given $p$, $1\leq p<\infty$, and given $\Omega\subset \mathbb{R}^n$
  there exists a constant $K = K(\delta_0,p,\Omega)$ such that for every $\delta$
  satisfying $0<\delta<\delta_0$ and every $u\in W^{2,p}(\Omega)$, we have
  \begin{equation}\label{eq:5.6}
    |u|_{1,p} \leq K\delta |u|_{2,p} + K\delta^{-1} |u|_{0,p}.
  \end{equation}
  Then there exists a constant $K = K(\epsilon_0,m,p,\Omega)$ such that
  for every $\epsilon$ satisfying $0<\epsilon\leq\epsilon_0$, every integer $j$
  satisfying $0\leq j\leq m-1$, and every $u\in W^{m,p}(\Omega)$, we have
  \begin{equation}\label{eq:5.7}
    |u|_{j,p} \leq K\epsilon |u|_{m,p} + K\epsilon^{-j/(m-j)} |u|_{0,p}.
  \end{equation}
\end{lemma}


\begin{proof}
  Since \eqref{eq:5.7} is trivial for $j=0$, we consider only the case $1\leq j\leq m-1$.
  The proof is accomplished by a double induction on $m$ and $j$.
  The constants $K_1$, $K_2$, $\ldots$ appearing in the argument may depend on $\delta_0$
  (or $\epsilon_0$), $m$, $p$, and $\Omega$. First we prove \eqref{eq:5.7} for $j=m-1$
  by induction on $m$, so that \eqref{eq:5.6} is the special case $m=2$. Assume, therefore,
  that for some $k$, $2\leq k\leq m-1$,
  \begin{equation}\label{eq:5.8}
    |u|_{k-1,p} \leq K_1 \delta |u|_{k,p} + K_1 \delta^{-(k-1)} |u|_{0,p}
  \end{equation}
  holds for all $\delta$, $0<\delta\leq\delta_0$, and all $u\in W^{k,p}(\Omega)$.
  If $u\in W^{k+1,p}(\Omega)$, we prove~\eqref{eq:5.8} with $k+1$ replacing $k$
  (and a different constant $K_1$). If $|\alpha|=k-1$ we obtain from~\eqref{eq:5.6}
  \[ |D^\alpha u|_{1,p} \leq K_2 \delta |D^\alpha u|_{2,p} + K_2 \delta^{-1} |D^\alpha u|_{0,p}. \]
  Combining this inequality with~\eqref{eq:5.8} we obtain, for $0<\eta\leq\delta_0$,
  \begin{align*}
    |u|_{k,p}
    & \leq K_3 \sum_{|\alpha|=k-1} |D^\alpha u|_{1,p} \\
    & \leq K_4 \delta |u|_{k+1,p} + K_4 \delta^{-1} |u|_{k-1,p} \\
    & \leq K_4 \delta |u|_{k+1,p} + K_4 K_1 \delta^{-1} \eta |u|_{k,p}
      + K_4 K_1 \delta^{-1} \eta^{1-k} |u|_{0,p}.
  \end{align*}
  We may assume without prejudice that $2K_1K_4\geq 1$. Therefore,
  we may take $\eta = \delta/(2K_1K_4)$ and so obtain
  \begin{align*}
    |u|_{k,p}
    & \leq 2 K_4 \delta |u|_{k+1,p} + \bigl(\delta / (2K_1K_4)\bigr)^{-k} |u|_{0,p} \\
    & \leq K_5 \delta |u|_{k+1,p} + K_5 \delta^{-k} |u|_{0,p}.
  \end{align*}
  This completes the induction establishing~\eqref{eq:5.8} for $0<\delta\leq\delta_0$
  and hence~\eqref{eq:5.7} for $j=m-1$ and $0<\epsilon\leq\delta_0$.

  We now prove by downward induction on $j$ that
  \begin{equation}\label{eq:5.9}
    |u|_{j,p} \leq K_6 \delta^{m-j} |u|_{m,p} + K_6 \delta^{-j} |u|_{0,p}
  \end{equation}
  holds for $1\leq j\leq m-1$ and $0<\delta\leq\delta_0$.
  Note that~\eqref{eq:5.8} with $k=m$ is the special case $j=m-1$
  of \eqref{eq:5.9}. Assume, therefore, that~\eqref{eq:5.9} holds for
  some $j$, $2\leq j\leq m-1$. We prove that it also holds with $j$
  replaced by $j-1$ (and a different constant $K_6$).
  From~\eqref{eq:5.8} and~\eqref{eq:5.9} we obtain
  \begin{align*}
    |u|_{j-1,p}
    & \leq K_7 \delta |u|_{j,p} + K_7 \delta^{1-j} |u|_{0,p} \\
    & \leq K_7 \delta \bigl(K_6 \delta^{m-j}|u|_{m,p}
      + K_6 \delta^{-j} |u|_{0,p}\bigr) + K_7 \delta^{1-j} |u|_{0,p} \\
    & \leq K_8 \delta^{m-(j-1)} |u|_{m,p} + K_8 \delta^{1-j} |u|_{0,p}.
  \end{align*}
  Thus \eqref{eq:5.9} holds, and \eqref{eq:5.7} follows by setting
  $\delta = \epsilon^{1/(m-j)}$ in~\eqref{eq:5.7} and noting that
  $\epsilon\leq\epsilon_0$ if $\delta\leq\delta_0$.
\end{proof}

This completes the proof of Theorem~5.2

\begin{remark}
  Careful consideration of the proofs of the previous two lemmas
  shows that if the height of the cone providing the cone condition
  for $\Omega$ is infinite, then inequalities~\eqref{eq:5.5}
  and~\eqref{eq:5.7} (and therefore~\eqref{eq:5.1} and~\eqref{eq:5.2})
  hold for all $\epsilon>0$, the corresponding constants $K$
  being independent of $\epsilon$. This is the case, for example,
  if $\Omega=\mathbb{R}^n$ or a half-space like $\mathbb{R}^n_+$.
\end{remark}


\section{Interpolation on Degree of Summability}

The following two interpolation theorems provide sharp estimates
for $L^q$ norms of functions in $W^{m,p}(\Omega)$. Some of these
estimates follow from Theorem~4.12 while others have traditionally
been obtained for regular domains from inbeddings of Sobolev
spaces of fractional order. (See Chapter~7.) We obtain them here
assuming only that the domain satisfies the cone condition. Again,
the weak cone condition would do as well; see [AF1].

\begin{theorem}
  Let $\Omega$ be a domain in $\mathbb{R}^n$ satisfying the cone
  condition. If $mp>n$, let $p\leq q\leq\infty$; if $mp=n$,
  let $p\leq q<\infty$; if $mp<n$, let $p\leq q\leq p^*=np/(n-mp)$.
  Then there exists a constant $K$ depending on $m,n,p,q$ and the
  dimensions of the cone $C$ providing the cone condition for $\Omega$,
  such that for all $u\in W^{m,p}(\Omega)$,
  \begin{equation}\label{eq:5.10}
    \|u\|_q \leq K \|u\|_{m,p}^\theta \|u\|_p^{1-\theta},
  \end{equation}
  where $\theta = (n/mp) - (n/mq)$.
\end{theorem}

\begin{proof}
  The case $mp<n$, $p\leq q\leq p^*$ follows directly from Theorems~2.11
  and~4.12:
  \[ \|u\|_q \leq \|u\|_{p^*}^\theta \|u\|_p^{1-\theta}
      \leq K\|u\|_{m,p}^\theta \|u\|_p^{1-\theta}, \]
  where $1/q = (\theta/p^*) + (1-\theta)/p$ from which it follows that
  $\theta = (n/mp) - (n/mq)$. For the cases $mp=n$, $p\leq q<\infty$,
  and $mp>n$, $p\leq q\leq\infty$ we use the local bound obtained in Lemma~4.15.
  If $0<r\leq\rho$ (the height of the cone $C$), then
  \begin{equation}\label{eq:5.11}
    |u(x)| \leq K_1 \biggl(\sum_{|\alpha|\leq m-1} r^{|\alpha|-n} \chi_r * |D^\alpha u|(x)
      + \sum_{|\alpha|=m} (\chi_r\omega_m)*|D^\alpha u|(x)\biggr),
  \end{equation}
  where $\chi_r$ is the characteristic function of the ball of radius $r$
  centred at the origin in $\mathbb{R}^n$, and $\omega_m(x) = |x|^{m-n}$.
  We estimate the $L^q$ norms of both terms on the right side of~\eqref{eq:5.11}
  using Young's inequality from Corollary~2.25. If $(1/p)+(1/s) = 1+(1/q)$, then
  \begin{align*}
    \|\chi_r * |D^\alpha u|\|_q
    & \leq \|\chi_r\|_s \|D^\alpha u\|_p = K_2 r^{n-(n/p)+(n/q)} \|D^\alpha u\|_p \\
      \|(\chi_r\omega_m) * |D^\alpha u|\|_q
    & \leq \|\chi_r\omega_m\|_s \|D^\alpha u\|_p = K_3 r^{m-(n/p)+(n/q)} \|D^\alpha u\|_p.
  \end{align*}
  (Note that $m-(n/p)+(n/q)>0$ if $q$ satisfies the above restrictions.) Hence
  \[ \|u\|_q \leq K_4 \biggl(\sum_{j=0}^{m-1} r^{j-(n/p)+(n/q)} |u|_{j,p}
      + r^{m-(n/p)+(n/q)} |u|_{m,p}\biggr). \]
  By Theorem~5.2,
  \[ |u|_{j,p} \leq K_5 (r^{m-j} |u|_{m,p} + r^{-j} \|u\|_p), \]
  so
  \[ \|u\|_q \leq K_6 \bigl(r^{m-(n/p)+(n/q)} \|u\|_{m,p}
      + r^{-(n/p+(n/q))} \|u\|_p\bigr). \]
  Adjusting $K_6$ if necessary, we can assume this inequality holds for all $r\leq 1$.
  Choosing $r$ to make the two terms on the right side equal, we obtain~\eqref{eq:5.10}.
\end{proof}

A special case of the above Theorem asserts that if $mp>n$, then
\begin{equation}\label{eq:5.12}
  \|u\|_\infty \leq K \|u\|_{m,p}^{n/mp} \|u\|_p^{1-(n/mp)}.
\end{equation}
A similar inequality with $\|u\|_p$ replaced by a more general $\|u\|_q$
is sometimes useful.

\begin{theorem}
  Let $\Omega$ be a domain in $\mathbb{R}^n$ satisfying the cone condition.
  Let $p>1$ and $mp>n$. Suppose that either $1\leq q\leq p$ or both $q>p$
  and $mp-p<n$. Then there exists a constant $K$ depending on $m$, $n$, $p$,
  $q$ and the
\end{theorem}