\chapter*{Preface}

This monograph presents an introductory study of of the properties of certain Banach spaces of 
weakly differentiable functions of several real variables that arise in connection with numerous 
problems in the theory of partial differential equations, approximation theory, and many other 
areas of pure and applied mathematics. These spaces have become associated with the name of the 
late Russian mathematician S.~L.~Sobolev, although their origins predate his major contributions 
to their development in the late 1930s.

Even by 1975 when the first edition of this monograph was published, there was a great deal of 
material on these spaces and their close relatives, though most of it was available only in 
research papers published in a wide variety of journals. The monograph was written to fill a 
perceived need for a single source where graduate students and researchers in a wide variety of 
disciplines could learn the essential features of Sobolev spaces that they needed for their 
particular applications. No attempt was made even at that time for complete coverage.
To quote from the Preface of the first edition:

\begin{quotation}
  The existing mathematical literature on Sobolev spaces and their generalizations is vast,
  and it would be neither easy nor particularly desirable to include everything that was known 
  about such spaces between the covers of one book. An attempt has been made in this monograph to 
  present all the core material in sufficient generality to cover most applications, to give the 
  reader an overview of the subject that is difficult to obtain by reading research papers, and 
  finally ... to provide a ready reference for someone requiring a result about Sobolev spaces for 
  use in some application.
\end{quotation}

This remains as the purpose and focus of this second edition. During the intervening twenty-seven 
years the research literature has grown exponentially, and there are now several other books in 
English that deal in whole or in part with Sobolev spaces. (For example, see [Ad2], [Bu1], [Mz1], 
[Tr1], [Tr3], and [Tr4].) However, there is still a need for students in other disciplines than 
mathematics, and in other areas of mathematics than just analysis to have available a book that 
describes these spaces and their core properties based only a background in mathematical analysis 
at the senior undergraduate level. We have tried to make this such a book. The organization of 
this book is similar but not identical to that of the first edition:

Chapter~1 remains a potpourri of standard topics from real and functional analysis, included, 
mainly without proofs, because they provide a necessary background for what follows.

Chapter~2 on the Lebesgue Spaces $L^p(\Omega)$ is much expanded and reworked from the previous edition. It provides, in addition to standard results about these spaces, a brief treatment of mixed-norm $L^p$ spaces, weak- $L^p$ spaces, and the Marcinkiewicz interpolation theorem, all of which will be used in a new treatment of the Sobolev Imbedding Theorem in Chapter 4. For the most part, complete proofs are given, as they are for much of the rest of the book.

Chapter~3 provides the basic definitions and properties of the Sobolev spaces $W^{m,p}(\Omega)$ and $W_0^{m, p}(\Omega)$. There are minor changes from the first edition.

Chapter~4 is now completely concerned with the imbedding properties of Sobolev Spaces. The first 
half gives a more streamlined presentation and proof of the various imbeddings of Sobolev spaces 
into $L^p$ spaces, including traces on subspaces of lower dimension, and spaces of continuous and 
uniformly continuous functions. Because the approach to the Sobolev Imbedding Theorem has changed, 
the roles of Chapters 4 and 5 have switched from the first edition. The latter part of Chapter~4 
deals with situations where the regularity conditions on the domain $\Omega$ that are necessary 
for the full Sobolev Imbedding Theorem do not apply, but some weaker imbedding results are still 
possible.

Chapter~5 now deals with interpolation, extension, and approximation results for Sobolev spaces. 
Part of it is expanded from material in Chapter 4 of the first edition with newer results and 
methods of proof.

Chapter~6 deals with establishing compactness of Sobolev imbeddings. It is only slightly changed 
from the first edition.

Chapter~7 is concerned with defining and developing properties of scales of spaces with fractional 
orders of smoothness, rather than the integer orders of the Sobolev spaces themselves. It is 
completely rewritten and bears little resemblance to the corresponding chapter in the first 
edition. Much emphasis is placed on real interpolation methods. The J-method and K-method are 
fully presented and used to develop the theory of Lorentz spaces and Besov spaces and their 
imbeddings, but both families of spaces are also provided with intrinsic characterizations. A
key theorem identifies lower dimensional traces of functions in Sobolev spaces
as constituting certain Besov spaces. Complex interpolation is used to introduce Sobolev spaces of 
fractional order (also called spaces of Bessel potentials) and Fourier transform methods are used 
to characterize and generalize these spaces to yield the Triebel Lizorkin spaces and illuminate 
their relationship with the Besov spaces.

Chapter~8 is very similar to its first edition counterpart. It deals with Orlicz and 
Orlicz-Sobolev spaces which generalize $L^p$ and $W^{m,p}$ spaces by allowing the role of 
the function $t^p$ to be assumed by a more general convex function $A(t)$. An important result 
identifies a certain Orlicz space as a target for an imbedding of $W^{m,p}(\Omega)$ in a limiting 
case where there is an imbedding into $L^p(\Omega)$ for $1 \leq p<\infty$
but not into $L^{\infty}(\Omega)$.

This monograph was typeset by the authors using \TeX{} on
a PC running LinuxMandrake 8.2. The figures were generated using the mathematical 
graphics software package \textit{MG} developed by R.~B.~Israel and R.~A.~Adams.

\begin{flushleft}
  \textit{RAA \& JJFF} \\
  Vancouver, August 2002
\end{flushleft}