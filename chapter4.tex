\chapter{The Sobolev Imbedding Theorem}


\begin{para}
  The imbedding characteristics of Sobolev spaces are essential in their uses in analysis, 
  especially in the study of differential and integral operators. The most important imbedding 
  results for Sobolev spaces are often gathered together into a single ``theorem'' called the 
  Sobolev Imbedding Theorem although they are of several different types and can require different 
  methods of proof. The core results are due to Sobolev [So2] but our statement (Theorem 4.12) 
  also includes refinements due to others, in particular Morrey [Mo] and Gagliardo [Ga1].
  Most of the imbeddings hold for domains $\Omega \subset \mathbb{R}^n$ satisfying some form of 
  ``cone condition'' that enables us to derive pointwise estimates for the value of a function at 
  the vertex of a truncated cone from suitable averages of the values of the function and its 
  derivatives over the cone. Some of the imbeddings require stronger geometric hypotheses which, 
  roughly speaking, force $\Omega$ to have an $(n-1)$-dimensional boundary that is locally the 
  graph of a Lipschitz continuous function and which, like the segment condition described in 
  Paragraph~3.21, requires $\Omega$ to lie on only one side of its boundary. We will discuss these 
  geometric properties of domains prior to the statement of the imbedding theorem itself.
\end{para}


\begin{para}[Targets of the Imbeddings]
  The Sobolev imbedding theorem asserts the existence of imbeddings of $W^{m, p}(\Omega)$
  (or $W_0^{m, p}(\Omega)$ ) into Banach spaces of the following types:
  \begin{enumerate}[(i)]
    \item $W^{j,q}(\Omega)$, where $j \leq m$, and in particular $L^q(\Omega)$.
    \item $W^{j,q}\left(\Omega_k\right)$, where, for $1 \leq k<n$,
      $\Omega_k$ is the intersection of $\Omega$ with a
      $k$-dimensional plane in $\mathbb{R}^n$.
    \item $C_B^j(\Omega)$, the space of functions having bounded, continuous derivatives up to order $j$ on $\Omega$ (see Paragraph 1.27) normed by
    \[
    \left\|u ; C_B^j(\Omega)\right\|=\max _{0 \leq|\alpha| \leq j} \sup _{x \in \Omega}\left|D^\alpha u(x)\right| .
    \]
    \item $C^j(\bar{\Omega})$, the closed subspace of $C_B^j(\Omega)$ consisting of functions 
      having bounded, uniformly continuous derivatives up to order $j$ on $\Omega$
      (see Paragraph 1.28) with the same norm as $C_B^j(\Omega)$ :
      \[
        \left\|\phi ; C^j(\bar{\Omega})\right\|=\max _{0 \leq \alpha \leq j} \sup _{x \in \Omega}\left|D^\alpha \phi(x)\right| .
      \]
      This space is smaller than $C_B^j(\Omega)$ in that its elements must be uniformly continuous 
      on $\Omega$. For example, the function $u$ of Example 3.20 belongs to $C_B^1(\Omega)$ but 
      certainly not to $C^1(\bar{\Omega})$ for the domain $\Omega$ of that example.
    \item $C^{j, \lambda}(\bar{\Omega})$, the closed subspace of $C^j(\bar{\Omega})$ consisting of 
      functions whose derivatives up to order $j$ satisfy Hölder conditions of exponent $\lambda$ 
      in $\Omega$ (see Paragraph 1.29). The norm on $C^{j, \lambda}(\bar{\Omega})$ is
      \[
        \left\|\phi ; C^{j, \lambda}(\bar{\Omega})\right\|=\left\|\phi ; C^j(\bar{\Omega})\right\|+\max _{0 \leq|\alpha| \leq j} \sup _{\substack{x, y \in \Omega \\ x \neq y}} \frac{\left|D^\alpha \phi(x)-D^\alpha \phi(y)\right|}{|x-y|^\lambda} .
      \]
  \end{enumerate}
  Since elements of $W^{m, p}(\Omega)$ are, strictly speaking, not functions defined everywhere on 
  $\Omega$, but rather equivalence classes of such functions defined and equal up to sets of 
  measure zero, we must clarify what is meant by imbeddings of types (ii)--(v).
  What is intended for imbeddings into the continuous function spaces (types (iii)--(v)) is that 
  the ``equivalence class'' $u \in W^{m, p}(\Omega)$ should contain an element that belongs to the 
  continuous function space that is the target of the imbedding and is bounded in that space by a 
  constant times $\|u\|_{m,p,\Omega}$. Hence, for example, existence of the imbedding
  \[
  W^{m, p}(\Omega) \rightarrow C^j(\bar{\Omega})
  \]
  means that each $u \in W^{m, p}(\Omega)$ when considered as a function,
  can be redefined on a subset of $\Omega$ having measure zero to produce
  a new function $u^* \in C^j(\bar{\Omega})$ such that $u^*=u$ in $W^{m, p}(\Omega)$
  (i.e.~$u^*$ and $u$ belong to the same ``equivalence class'' in $W^{m, p}(\Omega)$) and
  \[
  \left\|u^* ; C^j(\bar{\Omega})\right\| \leq K\|u\|_{m, p, \Omega}
  \]
  with $K$ independent of $u$.
  
  Even more care is necessary in interpreting imbeddings into spaces of type (ii):
  \[
  W^{m, p}(\Omega) \rightarrow W^{j, q}\left(\Omega_k\right)
  \]
  where $\Omega_k$ is the intersection of $\Omega$ with a plane of dimension $k<n$
  Each element of $W^{m, p}(\Omega)$ is, by Theorem~3.17, a limit in that space of a sequence 
  $\left\{u_i\right\}$ of functions in $C^{\infty}(\Omega)$.
  The functions $u_i$ have traces on $\Omega_k$ (that is, restrictions to $\Omega_k$) that belong 
  to $C^{\infty}\left(\Omega_k\right)$. The above imbedding signifies that these traces converge 
  in $W^{j, q}\left(\Omega_k\right)$ to a function $u^*$ that is independent of the choice of 
  $\left\{u_i\right\}$ and satisfies
  \[
  \left\|u^*\right\|_{j, q, \Omega_k} \leq K\|u\|_{m, p, \Omega}
  \]
  with $K$ independent of $u$.
\end{para}


\begin{para}
  Let us note as a point of interest, though of no particular use to us later,
  that the imbedding $W^{m, p}(\Omega) \rightarrow W^{j,q}(\Omega)$ is equivalent to the simple 
  containment $W^{m, p}(\Omega) \subset W^{j, q}(\Omega)$. Certainly the former implies the 
  latter. To verify the converse, suppose $W^{m, p}(\Omega) \subset W^{j,q}(\Omega)$, and let $I$ 
  be the linear operator taking $W^{m, p}(\Omega)$ into $W^{j, q}(\Omega)$ defined by $I u=u$ for 
  $u \in W^{m, p}(\Omega)$. If $u_k \rightarrow u$ in $W^{m, p}(\Omega)$
  (and hence in $L^p(\Omega)$) and $I u_k \rightarrow v$ in $W^{j,q}(\Omega)$
  (and hence in $L^q(\Omega)$), then, passing to a subsequence if necessary,
  we have by Corollary 2.17 that
  $u_k(x) \rightarrow u(x)$ a.e.~on $\Omega, u_k(x)=I u_k(x) \rightarrow v(x)$ a.e.~on $\Omega$. 
  Thus $u(x)=v(x)$ a.e.~on $\Omega$, that is, $I u=v$, and $I$ is continuous by the closed graph 
  theorem of functional analysis.
\end{para}


\section{Geometric Properties of Domains}

\begin{para}[Some Definitions]
  Many properties of Sobolev spaces defined on a domain $\Omega$, and in particular the imbedding 
  properties of these spaces, depend on regularity properties of $\Omega$. Such regularity is 
  normally expressed in terms of geometric or analytic conditions that may or may not be satisfied 
  by a given domain. We specify below several such conditions and consider their relationships. 
  First we make some definitions.
  
  Let $v$ be a nonzero vector in $\mathbb{R}^n$, and for each $x \neq 0$ let $\angle(x, v)$ be the 
  angle between the position vector $x$ and $v$. For given such $v, \rho>0$, and $\kappa$ 
  satisfying $0<\kappa \leq \pi$, the set
  \[
    C = \left\{x \in \mathbb{R}^n: x=0 \text { or } 0<|x| \leq \rho, \angle(x, v) \leq \kappa / 2\right\}
  \]
  is called a \emph{finite cone} of height $\rho$, axis direction $v$ and aperture angle $\kappa$ 
  with vertex at the origin. Note that $x+C=\{x+y: y \in C\}$ is a finite cone with vertex at $x$ 
  but the same dimensions and axis direction as $C$ and is obtained by parallel translation of $C$.
  Given $n$ linearly independent vectors $y_1, \ldots, y_n \in \mathbb{R}^n$, the set
  \[
    P = \left\{\sum_{j=1}^n \lambda_j y_j: 0 \leq \lambda_j \leq 1,\, 1 \leq j \leq n\right\}
  \]
  is a \emph{parallelepiped} with one vertex at the origin.
  Similarly, $x+P$ is a parallel translate of $P$ having one vertex at $x$.
  The centre of $x+P$ is the point given by
  $c(x+P)=x+(1 / 2)\left(y_1+\cdots+y_n\right)$. Every parallelepiped with a vertex at $x$ is 
  contained in a finite cone with vertex at $x$ and also contains such a cone.
  
  An open cover $\mathscr{O}$ of a set $S \subset \mathbb{R}^n$ is said to be locally finite
  if any compact set in $\mathbb{R}^n$ can intersect at most finitely many members
  of $\mathscr{O}$. Such locally finite collections of sets must be countable, so their elements 
  can be listed in sequence. If $S$ is closed, then any open cover of $S$ by sets with a uniform 
  bound on their diameters possesses a locally finite subcover.
  
  We now specify six regularity properties that a domain $\Omega \subset \mathbb{R}^n$ may 
  possess. We denote by $\Omega_\delta$ the set of points in $\Omega$ within distance $\delta$ of 
  the boundary of $\Omega$:
  \[
    \Omega_\delta = \{x \in \Omega: \operatorname{dist}(x, \partial\Omega)<\delta\}.
  \]
\end{para}


\begin{para}[The Segment Condition]
  As defined in Paragraph~3.21, a domain $\Omega$ satisfies the segment condition
  if every $x \in\partial\Omega$ has a neighbourhood $U_x$ and a nonzero vector $y_x$
  such that if $z \in \bar{\Omega} \cap U_x$, then $z+t y_x \in \Omega$ for $0<t<1$.
  Since the boundary of $\Omega$ is necessarily closed, we can replace its open cover by the 
  neighbourhoods $U_x$ with a locally finite subcover $\left\{U_1, U_2, \ldots\right\}$ with 
  corresponding vectors $y_1, y_2, \ldots$ such that if $x \in \bar{\Omega} \cap U_j$ for some 
  $j$, then $x+t y_j \in \Omega$ for $0<t<1$.
\end{para}


\begin{para}[The Cone Condition]
  $\Omega$ satisfies the cone condition if there exists a finite cone $C$ such that
  each $x \in \Omega$ is the vertex of a finite cone $C_x$ contained in $\Omega$
  and congruent to $C$. Note that $C_x$ need not be obtained from $C$ by parallel translation,
  but simply by rigid motion.
\end{para}


\begin{para}[The Weak Cone Condition]
  Given $x \in \Omega$, let $R(x)$ consist of all points $y \in \Omega$ such that the line segment from $x$ to $y$ lies in $\Omega$; thus $R(x)$ is a union of rays and line segments emanating from $x$. Let
  \[
  \Gamma(x)=\{y \in R(x):|y-x|<1\}
  \]
  We say that $\Omega$ satisfies the weak cone condition if there exists $\delta>0$ such that
  \[
  \mu_n(\Gamma(x)) \geq \delta \quad \text { for all } x \in \Omega
  \]
  where $\mu_n$ is the Lebesgue measure in $\mathbb{R}^n$. Clearly the cone condition implies the 
  weak cone condition, but there are many domains satisfying the weak cone condition that do not 
  satisfy the cone condition.
\end{para}


\begin{para}[The Uniform Cone Condition]
  $\Omega$ satisfies the uniform cone condition if there exists a locally finite open cover
  $\left\{U_j\right\}$ of the boundary of $\Omega$ and a corresponding sequence
  $\left\{C_j\right\}$ of finite cones, each congruent to some fixed finite cone $C$, such that
  \begin{enumerate}[(i)]
    \item There exists $M<\infty$ such that every $U_j$ has diameter less then $M$.
    \item $\Omega_\delta \subset \bigcup_{j=1}^{\infty} U_j$ for some $\delta>0$.
    \item $Q_j \equiv \bigcup_{x \in \Omega \cap U_j}\left(x+C_j\right) \subset \Omega$
      for every $j$.
    \item For some finite $R$, every collection of $R+1$ of the sets $Q_j$ has empty intersection.
  \end{enumerate}
\end{para}


\begin{para}[The Strong Local Lipschitz Condition]
  $\Omega$ satisfies the strong local Lipschitz condition if there exist positive numbers $\delta$ and $M$, a locally finite open cover $\left\{U_j\right\}$ of bdry $\Omega$, and, for each $j$ a real-valued function $f_j$ of $n-1$ variables, such that the following conditions hold:
  
  \begin{enumerate}[(i)]
    \item For some finite $R$, every collection of $R+1$ of the sets $U_j$ has empty intersection.
    \item For every pair of points $x, y \in \Omega_\delta$ such that $|x-y|<\delta$,
      there exists $j$ such that
      \[
        x, y \in V_j \equiv\left\{x \in U_j: \operatorname{dist}\left(x, \text { bdry } U_j\right)>\delta\right\}
      \]
    \item Each function $f_j$ satisfies a Lipschitz condition with constant $M$:
      that is, if $\xi=\left(\xi_1, \ldots, \xi_{n-1}\right)$ and
      $\rho=\left(\rho_1, \ldots, \rho_{n-1}\right)$ are in $\mathbb{R}^{n-1}$, then
      \[|f(\xi)-f(\rho)| \leq M|\xi-\rho|.\]
    \item For some Cartesian coordinate system $\left(\zeta_{j, 1}, \ldots, \zeta_{j,n}\right)$
      in $U_j, \Omega \cap U_j$ is represented by the inequality
      \[
        \zeta_{j, n}<f_j\left(\zeta_{j, 1}, \ldots, \zeta_{j, n-1}\right)
      \]
  \end{enumerate}
  If $\Omega$ is bounded, the rather complicated set of conditions above reduce to the simple condition that $\Omega$ should have a locally Lipschitz boundary, that is, that each point $x$ on the boundary of $\Omega$ should have a neighbourhood $U_x$ whose intersection with bdry $\Omega$ should be the graph of a Lipschitz continuous function.
\end{para}


\begin{para}[The Uniform $\bm{C^m}$-Regularity Condition]
  $\Omega$ satisfies the uniform $C^m$ regularity condition is there exists a locally finite open cover $\left\{U_j\right\}$ of bdry $\Omega$, and a corresponding sequence $\left\{\Phi_j\right\}$ of $m$-smooth transformations (see Paragraph 3.40) with $\Phi_j$ taking $U_j$ onto the ball $B=\left\{y \in \mathbb{R}^n:|y|<1\right.$ and having inverse $\Psi_j=\Phi_j^{-1}$, such that:
  \begin{enumerate}[(i)]
    \item For some finite $R$. every collection of $R+1$ of the sets $U_j$ has empty intersection.
    \item For some $\delta>0, \Omega_\delta \subset \bigcup_{j=1}^{\infty} \Psi_j\left(\left\{y \in \mathbb{R}^n:|y|<\frac{1}{2}\right\}\right)$.
    \item For each $j, \Phi_j\left(U_j \cap \Omega\right)=\left\{y \in B: y_n>0\right\}$.
    \item If $\left(\phi_{j, 1}, \ldots, \phi_{j, n}\right)$ and $\left(\psi_{j, 1}, \ldots, \psi_{j, n}\right)$ are the components of $\Phi_j$ and $\Psi_j$, then there is a finite constant $M$ such that for every $\alpha$ with $0<|\alpha| \leq m$, every $i, 1 \leq i \leq n$, and every $j$ we have
    \[
    \begin{array}{ll}
    \left|D^\alpha \phi_{j, i}(x)\right| \leq M, & \text { for } x \in U_j, \\
    \left|D^\alpha \psi_{j, i}(y)\right| \leq M, & \text { for } y \in B .
    \end{array}
    \]
  \end{enumerate}
\end{para}


\begin{para}
  Except for the cone condition and the weak cone condition, the other conditions defined above all require that the boundary of $\Omega$ be $(n-1)$-dimensional and that $\Omega$ lie on only one side of its boundary. The domain $\Omega$ of Example 3.20
  satisfies the cone condition (and therefore the weak cone condition), but none of the other four conditions. Among those four we have:
  \begin{align*}
    & \text{the uniform $C^m$-regularity condition $(m \geq 2)$} \\
    & \text{$\Longrightarrow$ the strong local Lipschitz condition} \\
    & \text{$\Longrightarrow$ the uniform cone condition} \\
    & \text{$\Longrightarrow$ the segment condition.}
  \end{align*}
  Also,
  \begin{align*}
    & \text{the uniform cone condition} \\
    & \text{$\Longrightarrow$ the cone condition} \\
    & \text{$\Longrightarrow$ the weak cone condition}
  \end{align*}
  Typically, most of the imbeddings of $W^{m, p}(\Omega)$ have been proven for domains
  satisfying the cone condition. Exceptions are the imbeddings into spaces $C^j(\bar{\Omega})$
  and $C^{j, \lambda}(\bar{\Omega})$ of uniformly continuous functions which, as suggested by 
  Example 3.20, require that $\Omega$ lie on one side of its boundary. These imbeddings are 
  usually proved for domains satisfying the strong local Lipschitz condition. It should be noted, 
  however, that $\Omega$ need not satisfy any of these conditions for appropriate imbeddings of 
  $W_0^{m, p}(\Omega)$ to be valid.
\end{para}

\begin{theorem}[The Sobolev Imbedding Theorem]
  Let $\Omega$ be a domain in $\mathbb{R}^n$ and, for $1\leq k\leq n$,
  let $\Omega_k$ be the intersection of $\Omega$ with a plane of dimension $k$
  in $\mathbb{R}^n$. (If $k=n$, then $\Omega_k = \Omega.$)
  Let $j\geq 0$ and $m\geq 1$ be integers and let $1\leq p < \infty$.

  \noindent\textbf{PART I} Suppose $\Omega$ satisfies the cone condition.

  \noindent\textbf{Case A} If either $mp > n$ or $m=n$ and $p=1$, then
  \begin{equation}
    W^{j+m, p}(\Omega) \to C_B^j(\Omega).
  \end{equation}
  Moreover, if $1\leq k\leq n$, then
  \begin{equation}
    W^{j+m, p}(\Omega) \to W^{j, q}(\Omega_k)\qquad \text{for } p\leq q\leq\infty,
  \end{equation}
  and, in particular,
  \[W^{m,p}(\Omega) \to L^q(\Omega)\qquad \text{for } p\leq q\leq\infty.\]

  \noindent \textbf{Case B}
\end{theorem}


\begin{remarks}

\begin{enumerate}[label = \arabic*.]
  \item Imbeddings (1)-(4) are essentially due to Sobolev [So1, So2], although his original proof 
    did not cover the all cases. Imbeddings (6)-(7) originate in the work of Morrey [Mo].
  \item Imbeddings (2)-(4) involving traces of functions on planes of lower dimension can be      
    extended in a reasonable manner to apply to traces on more general smooth manifolds. For 
    example, see Theorem 5.36.
  \item If $\Omega_k$ (or $\Omega$ ) has finite volume, then imbeddings (2)-(4) also hold for
    $1 \leq q<p$ in addition to the values of $q$ asserted in the statement of the theorem.
    This follows from Theorem 2.14. It will be shown in Theorem 6.43 that no imbedding of the form 
    $W^{m, p}(\Omega) \rightarrow L^q(\Omega)$ where $q<p$ is possible unless $\Omega$ has finite 
    volume.
  \item Part III of the theorem is an immediate consequence of Parts I and II
    applied to $\mathbb{R}^n$ because, by Lemma 3.27 , the operator of zero extension of functions 
    outside $\Omega$ maps $W_0^{m, p}(\Omega)$ isometrically
    into $W^{m, p}\left(\mathbb{R}^n\right)$.
  \item More generally, suppose there exists an operator $E$ mapping $W^{m, p}(\Omega)$
    into $W^{m, p}\left(\mathbb{R}^n\right)$ such that $E u(x)=u(x)$ a.e. in $\Omega$ and such 
    that $\|E u\|_{m, p, \mathbb{R}^n} \leq K_1\|u\|_{m, p, \Omega}$.
    Such an operator is called an $(m, p)$ extension operator for $\Omega$.
    If the imbedding theorem has already been proved for $\mathbb{R}^n$,
    then it must hold for the domain $\Omega$ as well. For example, if
    $W^{m, p}\left(\mathbb{R}^n\right) \rightarrow L^q\left(\mathbb{R}^n\right)$,
    and $u \in W^{m, p}(\Omega)$, then
    \[
      \|u\|_{q, \Omega} \leq\|E u\|_{q, \mathbb{R}^n} \leq K_2\|E u\|_{m, p, \mathbb{R}^n}
      \leq K_2 K_1\|u\|_{m, p, \Omega} .
    \]
    In Chapter 5 we will establish the existence of such extension operators, but only for domains 
    satisfying conditions stronger than the cone condition, so we will not use such a technique to 
    prove Theorem 4.12.
  \item It is sufficient to prove imbeddings (1)-(4), (6)-(7) for the special case $j=0$,
    as the general case follows by applying this special case to derivative $D^\alpha u$ of $u$ 
    for $|\alpha| \leq j$. For example, if the imbedding $W^{m, p}(\Omega)\rightarrow L^q(\Omega)$ 
    has been proven, then for any $u \in W^{j+m, p}(\Omega)$
    we have $D^\alpha u \in W^{m, p}(\Omega)$ for $|\alpha| \leq j$,
    whence $D^\alpha u \in L^q(\Omega)$. Thus $u \in W^{j, q}(\Omega)$ and
    \[
    \begin{aligned}
    \|u\|_{j, q} & =\left(\sum_{|\alpha| \leq j}\left\|D^\alpha u\right\|_{0 . q}^q\right)^{1 / q} \\
    & \leq K_1\left(\sum_{|\alpha| \leq j}\left\|D^\alpha u\right\|_{m, p}^p\right)^{1 / p} \leq K_2\|u\|_{j+m, p} .
    \end{aligned}
    \]
  \item The authors have shown that all of Part I can be proved for domains satisfying only the 
    weak cone condition instead of the cone condition. See [AF1].
\end{enumerate}
\end{remarks}



\begin{para}[Strategy for Proving the Imbedding Theorem]
  We use two overlapping methods to prove the imbeddings in Part I of Theorem 4.12.
  The first, potential theoretic in nature, was used by Sobolev. It works when $p>1$, and gives 
  the right order of growth of imbedding constants as $q \rightarrow \infty$ when $m p=n$;
  this will be useful in Chapter 7. Here we use the potential method to prove Case A and the 
  imbeddings in Cases B and C for $p>1$. The other approach is based on a combinatorial-averaging 
  argument due to Gagliardo [Ga1]. We will use it to establish Cases $\mathrm{B}$ and $\mathrm{C}$ 
  for $p=1$, though it could be adapted (with a bit more difficulty) to prove all of Part I.
  (See, in particular, Theorem 5.10 and the Remark following that theorem.)
  
  Part II of the theorem follows by sharpening certain estimates used in obtaining Case A of Part I.
  
  The entire proof of Theorem 4.12 is fairly lengthy and is broken down into several lemmas. Throughout we use $K$, and occasionally $K_1, K_2, \ldots$, to represent various constants that can depend on parameters of the spaces being imbedded. The values of these constants can change from line to line. While stated for the cone condition, the potential method works verbatim under the weak cone condition as well.
\end{para}


\section{Imbeddings by Potential Arguments}


\begin{lemma}[A Local Estimate]
  Let domain $\Omega \subset \mathbb{R}^n$ satisfy the cone condition. There exists a constant $K$ 
  depending on $m, n$, and the dimensions $\rho$ and $\kappa$ of the cone $C$ specified in the 
  cone condition for $\Omega$ such that for every $u \in C^{\infty}(\Omega)$,
  every $x \in \Omega$, and every $r$ satisfying $0<r \leq \rho$, we have
  \begin{equation}
    \begin{aligned}
      & |u(x)| \leq K\Biggl(\sum_{|\alpha| \leq m-1} r^{|\alpha|-n} \int_{C_{x, r}}\left|D^\alpha u(y)\right| d y \\
      & \quad+\sum_{|\alpha|=m} \int_{C_{x, r}}\left|D^\alpha u(y)\right||x-y|^{m-n} d y\Biggr),
    \end{aligned}
  \end{equation}
  where $C_{x, r}=\left\{y \in C_x:|x-y| \leq r\right\}$. Here $C_x \subset \Omega$ is a cone 
  congruent to $C$ having vertex at $x$.
\end{lemma}

\begin{proof}
  We apply Taylor's formula with integral remainder,
  \[
  f(1)=\sum_{j=0}^{m-1} \frac{1}{j !} f^{(j)}(0)+\frac{1}{(m-1) !} \int_0^1(1-t)^{m-1} f^{(m)}(t) d t
  \]
  to the function $f(t)=u(t x+(1-t) y)$, where $x \in \Omega$ and $y \in C_{x, r}$. Noting that
  \[
  f^{(j)}(t)=\sum_{|\alpha|=j} \frac{j !}{\alpha !} D^\alpha u(t x+(1-t) y)(x-y)^\alpha
  \]
  where $\alpha !=\alpha_{1} ! \cdots \alpha_{n} !$ and $(x-y)^\alpha=\left(x_1-y_1\right)^{\alpha_1} \cdots\left(x_n-y_n\right)^{\alpha_n}$, we obtain
  \[
  \begin{aligned}
  |u(x)| \leq & \sum_{|\alpha| \leq m-1} \frac{1}{\alpha !}\left|D^\alpha u(y)\right||x-y|^{|\alpha|} \\
  & +\sum_{|\alpha|=m} \frac{m}{\alpha !}|x-y|^m \int_0^1(1-t)^{m-1}\left|D^\alpha u(t x+(1-t) y)\right| d t .
  \end{aligned}
  \]
  If $C$ has volume $c \rho^n$, then $C_{x, r}$ has volume $c r^n$.
  Integration of $y$ over $C_{x, r}$ leads to
  \[
  \begin{aligned}
  & c r^n|u(x)| \\
  & \quad \leq \sum_{|\alpha| \leq m-1} \frac{r^{|\alpha|}}{\alpha !} \int_{C_{x, r}}\left|D^\alpha u(y)\right| d y \\
  & \quad+\sum_{|\alpha|=m} \frac{m}{\alpha !} \int_{C_{x, r}}|x-y|^m d y \int_0^1(1-t)^{m-1}\left|D^\alpha u(t x+(1-t) y)\right| d t
  \end{aligned}
  \]
  In the final (double) integral we first change the order of integration, then substitute $z=t x+(1-t) y$, so that $z-x=(1-t)(y-x)$ and $d z=(1-t)^n d y$, to obtain, for that integral,
  \[
  \int_0^1(1-t)^{-n-1} d t \int_{C_{x,(1-t) r}}|z-x|^m\left|D^\alpha u(z)\right| d z
  \]
  A second change of order of integration now gives for the above integral
  \[
  \begin{aligned}
  & \int_{C_{x, r}}|x-z|^m\left|D^\alpha u(z)\right| d z \int_0^{1-(|z-x| / r)}(1-t)^{-n-1} d t \\
  & \quad \leq \frac{r^n}{n} \int_{C_{x, r}}|x-z|^{m-n}\left|D^\alpha u(z)\right| d z
  \end{aligned}
  \]
  Inequality (8) now follows immediately.
\end{proof}


\begin{para}[Proof of Part I, Case A of Theorem 4.12]
  As noted earlier, we can assume that $j=0$. Let $u \in W^{m, p}(\Omega) \cap C^{\infty}(\Omega)$ 
  and let $x \in \Omega$. We must show that
  \[
  |u(x)| \leq K\|u\|_{m, p} .
  \]
  For $p=1$ and $m=n$, this follows immediately from (8). For $p>1$ and $m p>n$, we apply Hölder's inequality to (8) with $r=\rho$ to obtain
  \[
  \begin{aligned}
  |u(x)| & \leq K\left(\sum_{|\alpha| \leq m-1} c^{1 / p^{\prime}} \rho^{|\alpha|-(n / p)}\left\|D^\alpha u\right\|_{p, C_{x, \rho}}\right. \\
  & \left.+\sum_{|\alpha|=m}\left\|D^\alpha u\right\|_{p, C_{x, \rho}}\left[\int_{C_{x, \rho}}|x-y|^{(m-n) p^{\prime}} d y\right]^{1 / p^{\prime}}\right),
  \end{aligned}
  \]
  where $c$ is the volume of $C_{x, 1}$ and $p^{\prime}=p /(p-1)$. The final integral is finite since $(m-n) p^{\prime}>-n$ when $m p>n$. Thus
  \[
  |u(x)| \leq K \sum_{|\alpha| \leq m}\left\|D^\alpha u\right\|_{p, C_{x, \rho}}
  \]
  and (9) follows because $C_{x, \rho} \subset \Omega$.
  
  Next observe that since any $u \in W^{m, p}(\Omega)$ is the limit of a Cauchy sequence of continuous functions by Theorem 3.17, and since (9) implies this Cauchy sequence converges to a continuous function on $\Omega, u$ must coincide with a continuous function a.e. on $\Omega$. Thus $u \in C_B^0(\Omega)$ and imbedding (1) is proved.
  Now let $\Omega_k$ denote the intersection of $\Omega$ with a $k$-dimensional plane $H$, let $\Omega_{k, \rho}=\left\{x \in \mathbb{R}^n: \operatorname{dist}\left(x, \Omega_k\right)<\rho\right\}$, and let $u$ and all its derivatives be extended to be zero outside $\Omega$. Since $C_{x, \rho} \subset B_\rho(x)$, the ball of radius $\rho$ with centre at $x$, we have, using (10) and denoting by $d x^{\prime}$ the $k$-volume element in $H$,
  \[
  \begin{aligned}
  \int_{\Omega_k}|u(x)|^p d x^{\prime} & \leq K \sum_{|\alpha| \leq m} \int_{\Omega_k} d x^{\prime} \int_{B_\rho(x)}\left|D^\alpha u(y)\right|^p d y \\
  & =K \sum_{|\alpha| \leq m} \int_{\Omega_{k, \rho}}\left|D^\alpha u(y)\right|^p d y \int_{H \cap B_p(y)} d x^{\prime} \leq K_1\|u\|_{m, p, \Omega}^p,
  \end{aligned}
  \]
  and $W^{m, p}(\Omega) \rightarrow L^p\left(\Omega_k\right)$. But (9) shows that $W^{m, p}(\Omega) \rightarrow L^{\infty}\left(\Omega_k\right)$ and so imbedding (2) follows by Theorem 2.11.
\end{para}

Let $\chi_r$ be the characteristic function of the ball $B_r(0)=\left\{x \in \mathbb{R}^n:|x|<r\right\}$. In the following discussion we will develop estimates for convolutions of $L^p$ functions with the kernels $\omega_m(x)=|x|^{m-n}$ and
\[
\chi_r \omega_m(x)= \begin{cases}|x|^{m-n} & \text { if }|x|<r, \\ 0 & \text { if }|x| \geq r .\end{cases}
\]
Observe that if $m \leq n$ and $0<r \leq 1$, then
\[
\chi_r(x) \leq \chi_r \omega_m(x) \leq \omega_m(x).
\]


\begin{lemma}
  Let $p \geq 1,1 \leq k \leq n$, and $n-m p<k$. There exists a constant $K$ such that for every $r>0$, every $k$-dimensional plane $H \subset \mathbb{R}^n$, and every $v \in L^p\left(\mathbb{R}^n\right)$, we have $\chi_r \omega_m *|v| \in L^p(H)$ and
  \[
  \left\|\chi_r \omega_m *|v|\right\|_{p, H} \leq K r^{m-(n-k) / p}\|v\|_{p, \mathbb{R}^n}
  \]
  In particular,
  \[
  \left\|\chi_1 *|v|\right\|_{p, H} \leq\left\|\chi_1 \omega_m *|v|\right\|_{p, H} \leq K\|v\|_{p, \mathbb{R}^n}
  \]
\end{lemma}

\begin{proof}
  If $p>1$, then by Hölder's inequality
  \[
  \begin{aligned}
  \chi_r \omega_m *|v|(x) & =\int_{B_r(x)}|v(y)||x-y|^{-s}|x-y|^{s+m-n} d y \\
  & \leq\left(\int_{B_r(x)}|v(y)|^p|x-y|^{-s p} d y\right)^{1 / p}\left(\int_{B_r(x)}|x-y|^{(s+m-n) p^{\prime}} d y\right)^{1 / p^{\prime}} \\
  & =K r^{s+m-(n / p)}\left(\int_{B_r(x)}|v(y)|^p|x-y|^{-s p} d y\right)^{1 / p}
  \end{aligned}
  \]
  provided $s+m-(n / p)>0$. If $p=1$ the same estimate holds provided $s+m-n \geq 0$ without using Hölder's inequality.
  
  Integrating the $p$ th power of the above estimate over $H$ (with volume element $\left.d x^{\prime}\right)$, we obtain
  \[
  \begin{aligned}
  \left\|\chi_r \omega_m *|v|\right\|_{p, H}^p & =\int_H\left|\chi_r \omega_m *\right| v|(x)|^p d x^{\prime} \\
  & \leq K r^{(s+m) p-n} \int_H d x^{\prime} \int_{B_r(x)}|v(y)|^p|x-y|^{-s p} d y \\
  & \leq K r^{(s+m) p-n} r^{k-s p}\|v\|_{p, \mathbb{R}^n}^p=K r^{m p-(n-k)}\|v\|_{p, \mathbb{R}^n}^p
  \end{aligned}
  \]
  provided $k>s p$
  Since $n-m p<k$ there exists $s$ satisfying $(n / p)-m<s<k / p$, so both estimates above are valid and (11) holds.
\end{proof}


\begin{lemma}
  Let $p>1, m p<n, n-m p<k \leq n$, and $p^*=k p /(n-m p)$. There exists a constant $K$ such that for every $k$-dimensional plane $H$ in $\mathbb{R}^n$ and every $v \in L^p\left(\mathbb{R}^n\right)$, we have $\omega_m *|v| \in L^{p^*}(H)$ and
  \[
  \left\|\chi_1 *|v|\right\|_{p^*, H} \leq\left\|\chi_1 \omega_m *|v|\right\|_{p^*, H} \leq\left\|\omega_m *|v|\right\|_{p^*, H} \leq K\|v\|_{p, \mathbb{R}^n}
  \]
\end{lemma}

\begin{proof}
  Only the final inequality of (12) requires proof. Since $m p<n$, for each $x \in \mathbb{R}^n$ Hölder's inequality gives
  \[
  \begin{aligned}
  \int_{\mathbb{R}^n-B_r(x)}|v(y) \| x-y|^{m-n} d y & \leq\|v\|_{p, \mathbb{R}^n}\left(\int_{\mathbb{R}^n-B_r(x)}|x-y|^{(m-n) p^{\prime}} d y\right)^{1 / p^{\prime}} \\
  & =K_1\|v\|_{p, \mathbb{R}^n}\left(\int_r^{\infty} t^{(m-n) p^{\prime}+n-1} d t\right)^{1 / p^{\prime}} \\
  & =K_1 r^{m-(n / p)}\|v\|_{p, \mathbb{R}^n}
  \end{aligned}
  \]
  If $t>0$, choose $r$ so that $K_1 r^{m-(n / p)}\|v\|_{p, \mathbb{R}^n}=t / 2$. If
  \[
  \omega_m *|v|(x)=\int_{\mathbb{R}^n}|v(y)||x-y|^{m-n} d y>t,
  \]
  then
  \[
  \chi_r \omega_m *|v|(x)=\int_{B_r(x)}|v(y)||x-y|^{m-n} d y>t / 2 .
  \]
  Thus
  \[
  \begin{aligned}
  \mu_k\left(\left\{x \in H: \omega_m *|v|(x)>t\right\}\right) & \leq \mu_k\left(\left\{x \in H: \chi_r \omega_m *|v|(x)>t / 2\right\}\right) \\
  & \leq\left(\frac{2}{t}\right)^p\left\|\chi_r \omega_m *|v|\right\|_{p, H}^p \\
  & \leq\left(\frac{r^{(n / p)-m}}{K_1\|v\|_{p, \mathbb{R}^n}}\right)^p K r^{m p-n+k}\|v\|_{p, \mathbb{R}^n}^p=K_2 r^k
  \end{aligned}
  \]
  by inequality (11). But $r^k=\left(2 K_1\|v\|_{p, \mathbb{R}^n} / t\right)^{p^*}$, so
  \[
  \mu_k\left(\left\{x \in H: \omega_m *|v|(x)>t\right\}\right) \leq K_2\left(\frac{2 K_1}{t}\|v\|_{p, \mathbb{R}^n}\right)^{p^*}
  \]
  Thus the mapping $I:\left.v \mapsto\left(\omega_m *|v|\right)\right|_H$ is of weak type $\left(p, p^*\right)$.
  For fixed $m, n, k$, the values of $p$ satisfying the conditions of this lemma constitute an open interval, so there exist $p_1$ and $p_2$ in that interval, and a number $\theta$ satisfying $0<\theta<1$ such that
  \[
  \frac{1}{p}=\frac{1-\theta}{p_1}+\frac{\theta}{p_2},
  \]
  and
  \[
  \frac{1}{p^*}=\frac{n / k}{p}-\frac{m}{k}=\frac{1-\theta}{p_1^*}+\frac{\theta}{p_2^*}.
  \]
  Since $p^*>p$, the Marcinkiewicz interpolation theorem 2.58 assures us that $I$ is bounded from $L^p\left(\mathbb{R}^n\right)$ into $L^{p^*}(H)$, that is, (12) holds.
\end{proof}


\begin{para}[Proof of Part I, Case C of Theorem 4.12 for $\bm{p>1}$]
  We have $mp < n$, $n - mp < k \leq n$, and $p \leq q \leq p *=k p /(n-m p)$.
  Let $u \in C^{\infty}(\Omega)$ and extend $u$ and all its derivatives to be zero
  on $\mathbb{R}^n-\Omega$. Taking $r=\rho$ in Lemma 4.15 and replacing $C_{x, r}$ with the larger 
  ball $B_1(x)$, we obtain
  \[
  |u(x)| \leq K\left(\sum_{|\alpha| \leq m-1} \chi_1 *\left|D^\alpha u\right|(x)+\sum_{|\alpha|=m} \chi_1 \omega_m *\left|D^\alpha u\right|(x)\right)
  \]
  If $1 / q=\theta / p+(1-\theta) / p^*$ where $0 \leq \theta \leq 1$, then by the interpolation inequality of Theorem 2.11 and Lemmas 4.17 and 4.18
  \[
  \begin{aligned}
  \|u\|_{q, \Omega_k} & \leq\|u\|_{p, H}^\theta\|u\|_{p^*, H}^{1-\theta} \\
  & \leq K\left(\sum_{|\alpha| \leq m}\left\|D^\alpha u\right\|_{p, \mathbb{R}^n}\right)^\theta\left(\sum_{|\alpha| \leq m}\left\|D^\alpha u\right\|_{p, \mathbb{R}^n}\right)^{1-\theta} \\
  & \leq K\|u\|_{m, p, \Omega}
  \end{aligned}
  \]
  as required.
\end{para}


\begin{para}[Proof of Part I, Case B of Theorem 4.12 for $\bm{p>1}$]
  We have $m p=n$, $1 \leq k \leq n$, and $p \leq q<\infty$. We can select numbers $p_1, p_2$, and $\theta$ such that $1<p_1<p<p_2, n-m p_1<k, 0<\theta<1$, and
  \[
  \frac{1}{p}=\frac{\theta}{p_1}+\frac{1-\theta}{p_2}, \quad \frac{1}{q}=\frac{\theta}{p_1}
  \]
  As in the above proof of Case $C$ for $p>1$, the maps $\left.v \mapsto\left(\chi_1 *|v|\right)\right|_H$ and $\left.v \mapsto\left(\chi_1 \omega_m *|v|\right)\right|_H$ are bounded from $L^{p_1}\left(\mathbb{R}^n\right)$ into $L^{p_1}\left(\mathbb{R}^k\right)$ and so are of weak type $\left(p_1, p_1\right)$. As in the proof of Case $\mathrm{A}$, these same maps are bounded from $L^{p_2}\left(\mathbb{R}^n\right)$ into $L^{\infty}\left(\mathbb{R}^k\right)$ and so are of weak type $\left(p_2, \infty\right)$. By the Marcinkiewicz theorem again, they are bounded from $L^p\left(\mathbb{R}^n\right)$ into $L^q\left(\mathbb{R}^k\right)$ and
  \[
  \left\|\chi_1 *|v|\right\|_{q, H} \leq\left\|\chi_1 \omega_m *|v|\right\|_{q, H} \leq K\|v\|_{p, \mathbb{R}^n}
  \]
  and the desired result follows by applying these estimates to the various terms of (13).
\end{para}


\section{Imbeddings by Averaging}

\begin{para}
  We still need to prove the imbeddings for Cases $\mathrm{B}$ and $\mathrm{C}$ with $p=1$.
  We first prove that $W^{1,1}(\Omega) \rightarrow L^{n /(n-1)}(\Omega)$ and deduce from this and 
  the imbeddings already established for $p>1$ that all but one of the remaining imbeddings in 
  Cases $\mathrm{B}$ and $\mathrm{C}$ are valid. The remaining imbedding is the special case
  of $\mathrm{C}$ where $k=n-m$, $p=1$, $p^*=1$ which will require a special proof.
  
  We first show that any domain satisfying the cone condition is the union of finitely many 
  subdomains each of which is a union of parallel translates of a fixed parallelepiped.
  Then we establish a special case of a combinatorial lemma estimating a function in terms of
  averages in coordinate directions. Both of these results are due to Gagliardo [Ga1] and
  constitute the foundation on which rests his proof of all of Cases B and C of Part I.
\end{para}


\begin{lemma}[Decomposition of $\bm{\Omega}$]
  Let $\Omega \subset \mathbb{R}^n$ satisfy the cone condition. Then there exists a finite 
  collection $\left\{\Omega_1, \ldots, \Omega_N\right\}$ of open subsets of $\Omega$ such
  that $\Omega=\bigcup_{j=1}^N \Omega_j$, and such that to each $\Omega_j$ there corresponds a 
  subset $A_j \subset \Omega_j$ and an open parallelepiped $P_j$ with one vertex at 0 such that 
  $\Omega_j=\bigcup_{x \in A_j}\left(x+P_j\right)$.
  
  If $\Omega$ is bounded and $\rho>0$ is given, we can accomplish the above decomposition using 
  sets $A_j$ each satisfying $\operatorname{diam}\left(A_j\right)<\rho$.
  
  Finally, if $\Omega$ is bounded and $\rho>0$ is sufficiently small, then each $\Omega_j$ will 
  satisfy the strong local Lipschitz condition.
\end{lemma}

Proof. Let $C$ be the finite cone with vertex at 0 such that any $x \in \Omega$ is the vertex of a finite cone $C_x \subset \Omega$ congruent to $C$. We can select a finite number of finite cones $C_1, \ldots, C_N$ each having vertex at 0 (and each having the same height as $C$ but smaller aperture angle than $C$ ) such that any finite cone congruent to $C$ and having vertex at 0 must contain one of the cones $C_j$. For each $j$, let $P_j$ be an open parallelepiped with one vertex at the origin, contained in $C_j$, and having positive volume. Then for each $x \in \Omega$ there exists $j, 1 \leq j \leq N$, such that
\[
x+P_j \subset x+C_j \subset C_x \subset \Omega .
\]
Since $\Omega$ is open and $\overline{x+P_j}$ is compact, $y+P_j \subset \Omega$ for any $y$ sufficiently close to $x$. Hence every $x \in \Omega$ belongs to $y+P_j$ for some $j$ and some $y \in \Omega$. Let $A_j=\left\{y \in \bar{\Omega}: y+P_j \subset \Omega\right\}$ and let $\Omega_j=\bigcup_{y \in A_j}\left(y+P_j\right)$. Then $\Omega=\bigcup_{j=1}^N \Omega_j$. Now suppose $\Omega$ is bounded and $\rho>0$ is given. If $\operatorname{diam}\left(A_j\right) \geq \rho$ we can decompose $A_j$ into a finite union of sets $A_{j i}$ each with diameter less than $\rho$ and define the corresponding parallelepiped $P_{j i}=P_j$. We then rename the totality of such sets $A_{j i}$ as a single finite family, which we again call $\left\{A_j\right\}$ and define $\Omega_j$ as above.

Figure 2 attempts to illustrate these notions for the domain in $\mathbb{R}^2$ considered in