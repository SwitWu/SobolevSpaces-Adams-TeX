\chapter{The Sobolev Imbedding Theorem}


\begin{para}
  The imbedding characteristics of Sobolev spaces are essential in their uses in analysis, 
  especially in the study of differential and integral operators.
  The most important imbedding results for Sobolev spaces are often gathered together into
  a single ``theorem'' called the Sobolev Imbedding Theorem although they are of several different
  types and can require different methods of proof.
  The core results are due to Sobolev [So2] but our statement (Theorem 4.12) 
  also includes refinements due to others, in particular Morrey [Mo] and Gagliardo [Ga1].
  Most of the imbeddings hold for domains $\Omega \subset \mathbb{R}^n$ satisfying some form of 
  ``cone condition'' that enables us to derive pointwise estimates for the value of a function at 
  the vertex of a truncated cone from suitable averages of the values of the function and its 
  derivatives over the cone. Some of the imbeddings require stronger geometric hypotheses which, 
  roughly speaking, force $\Omega$ to have an $(n-1)$-dimensional boundary that is locally the 
  graph of a Lipschitz continuous function and which, like the segment condition described in 
  Paragraph~3.21, requires $\Omega$ to lie on only one side of its boundary. We will discuss these 
  geometric properties of domains prior to the statement of the imbedding theorem itself.
\end{para}


\begin{para}[Targets of the Imbeddings]
  The Sobolev imbedding theorem asserts the existence of imbeddings of $W^{m,p}(\Omega)$
  (or $W_0^{m, p}(\Omega)$) into Banach spaces of the following types:
  \begin{enumerate}[(i)]
    \item $W^{j,q}(\Omega)$, where $j \leq m$, and in particular $L^q(\Omega)$,
    \item $W^{j,q}(\Omega_k)$, where, for $1 \leq k<n$,
      $\Omega_k$ is the intersection of $\Omega$ with a
      $k$-dimensional plane in $\mathbb{R}^n$.
    \item $C_B^j(\Omega)$, the space of functions having bounded,
      continuous derivatives up to order $j$ on $\Omega$ (see Paragraph~1.27) normed by
      \[
        \|u\|_{C_B^j(\Omega)} = \max_{0 \leq|\alpha| \leq j} \sup_{x \in \Omega}\left|D^\alpha u(x)\right|.
      \]
    \item $C^j(\overline{\Omega})$, the closed subspace of $C_B^j(\Omega)$ consisting of functions
      having bounded, uniformly continuous derivatives up to order $j$ on $\Omega$
      (see Paragraph~1.28) with the same norm as $C_B^j(\Omega)$:
      \[
        \|\phi\|_{C^j(\overline{\Omega})} 
          = \max_{0 \leq \alpha \leq j} \sup_{x \in \Omega} \left|D^\alpha \phi(x)\right|.
      \]
      This space is smaller than $C_B^j(\Omega)$ in that its elements must be uniformly continuous 
      on $\Omega$. For example, the function $u$ of Example~3.20 belongs to $C_B^1(\Omega)$ but 
      certainly not to $C^1(\overline{\Omega})$ for the domain $\Omega$ of that example.
    \item $C^{j, \lambda}(\overline{\Omega})$, the closed subspace of $C^j(\overline{\Omega})$ consisting of 
      functions whose derivatives up to order $j$ satisfy Hölder conditions of exponent $\lambda$ 
      in $\Omega$ (see Paragraph~1.29). The norm on $C^{j, \lambda}(\overline{\Omega})$ is
      \[
        \|\phi\|_{C^{j, \lambda}(\overline{\Omega})}
          = \|\phi\|_{C^j(\overline{\Omega})}
            + \max_{0 \leq|\alpha| \leq j} \sup_{\substack{x, y \in \Omega \\ x \neq y}}
              \frac{\left|D^\alpha \phi(x)-D^\alpha \phi(y)\right|}{|x-y|^\lambda} .
      \]
  \end{enumerate}
  Since elements of $W^{m,p}(\Omega)$ are, strictly speaking,
  not functions defined everywhere on $\Omega$, but rather equivalence classes of such functions defined and equal up to sets of measure zero,
  we must clarify what is meant by imbeddings of types (ii)--(v).
  What is intended for imbeddings into the continuous function spaces (types (iii)--(v)) is that 
  the ``equivalence class'' $u \in W^{m,p}(\Omega)$ should contain an element that belongs to the 
  continuous function space that is the target of the imbedding and is bounded in that space by a 
  constant times $\|u\|_{m,p,\Omega}$. Hence, for example, existence of the imbedding
  \[
  W^{m,p}(\Omega) \rightarrow C^j(\overline{\Omega})
  \]
  means that each $u \in W^{m,p}(\Omega)$ when considered as a function,
  can be redefined on a subset of $\Omega$ having measure zero to produce
  a new function $u^* \in C^j(\overline{\Omega})$ such that $u^*=u$ in $W^{m,p}(\Omega)$
  (i.e.~$u^*$ and $u$ belong to the same ``equivalence class'' in $W^{m,p}(\Omega)$) and
  \[
    \|u^*\|_{C^j(\overline{\Omega})} \leq K\|u\|_{m, p, \Omega}
  \]
  with $K$ independent of $u$.
  
  Even more care is necessary in interpreting imbeddings into spaces of type (ii):
  \[
    W^{m,p}(\Omega) \rightarrow W^{j, q}(\Omega_k)
  \]
  where $\Omega_k$ is the intersection of $\Omega$ with a plane of dimension $k<n$
  Each element of $W^{m,p}(\Omega)$ is, by Theorem~3.17, a limit in that space of a sequence 
  $\{u_i\}$ of functions in $C^{\infty}(\Omega)$.
  The functions $u_i$ have traces on $\Omega_k$ (that is, restrictions to $\Omega_k$) that belong 
  to $C^{\infty}(\Omega_k)$. The above imbedding signifies that these traces converge 
  in $W^{j,q}(\Omega_k)$ to a function $u^*$ that is independent of the choice of 
  $\{u_i\}$ and satisfies
  \[
  \left\|u^*\right\|_{j, q, \Omega_k} \leq K\|u\|_{m, p, \Omega}
  \]
  with $K$ independent of $u$.
\end{para}


\begin{para}
  Let us note as a point of interest, though of no particular use to us later,
  that the imbedding $W^{m,p}(\Omega) \rightarrow W^{j,q}(\Omega)$ is equivalent to the simple 
  containment $W^{m,p}(\Omega) \subset W^{j, q}(\Omega)$. Certainly the former implies the 
  latter. To verify the converse, suppose $W^{m,p}(\Omega) \subset W^{j,q}(\Omega)$, and let $I$ 
  be the linear operator taking $W^{m,p}(\Omega)$ into $W^{j,q}(\Omega)$ defined by $I u=u$ for 
  $u \in W^{m,p}(\Omega)$. If $u_k \rightarrow u$ in $W^{m,p}(\Omega)$
  (and hence in $L^p(\Omega)$) and $I u_k \rightarrow v$ in $W^{j,q}(\Omega)$
  (and hence in $L^q(\Omega)$), then, passing to a subsequence if necessary,
  we have by Corollary 2.17 that
  $u_k(x) \rightarrow u(x)$ a.e.~on $\Omega, u_k(x)=I u_k(x) \rightarrow v(x)$ a.e.~on $\Omega$. 
  Thus $u(x)=v(x)$ a.e.~on $\Omega$, that is, $I u=v$, and $I$ is continuous by the closed graph 
  theorem of functional analysis.
\end{para}


\section{Geometric Properties of Domains}

\begin{para}[Some Definitions]
  Many properties of Sobolev spaces defined on a domain $\Omega$, and in particular the imbedding 
  properties of these spaces, depend on regularity properties of $\Omega$. Such regularity is 
  normally expressed in terms of geometric or analytic conditions that may or may not be 
  satisfied by a given domain. We specify below several such conditions and consider their 
  relationships. First we make some definitions.
  
  Let $v$ be a nonzero vector in $\mathbb{R}^n$, and for each $x \neq 0$ let $\angle(x, v)$ be 
  the angle between the position vector $x$ and $v$. For given such $v, \rho>0$, and $\kappa$ 
  satisfying $0<\kappa \leq \pi$, the set
  \[
    C = \left\{x \in \mathbb{R}^n: x=0 \text { or } 0<|x| \leq \rho, \angle(x, v) \leq \kappa / 2\right\}
  \]
  is called a \emph{finite cone} of height $\rho$, axis direction $v$ and aperture angle $\kappa$ 
  with vertex at the origin. Note that $x+C=\{x+y: y \in C\}$ is a finite cone with vertex at $x$ 
  but the same dimensions and axis direction as $C$ and is obtained by parallel translation of $C$.
  Given $n$ linearly independent vectors $y_1, \ldots, y_n \in \mathbb{R}^n$, the set
  \[
    P = \left\{\sum_{j=1}^n \lambda_j y_j: 0 \leq \lambda_j \leq 1,\, 1 \leq j \leq n\right\}
  \]
  is a \emph{parallelepiped} with one vertex at the origin.
  Similarly, $x+P$ is a parallel translate of $P$ having one vertex at $x$.
  The centre of $x+P$ is the point given by
  $c(x+P)=x+(1 / 2)\left(y_1+\cdots+y_n\right)$. Every parallelepiped with a vertex at $x$ is 
  contained in a finite cone with vertex at $x$ and also contains such a cone.
  
  An open cover $\mathscr{O}$ of a set $S \subset \mathbb{R}^n$ is said to be locally finite
  if any compact set in $\mathbb{R}^n$ can intersect at most finitely many members
  of $\mathscr{O}$. Such locally finite collections of sets must be countable, so their elements 
  can be listed in sequence. If $S$ is closed, then any open cover of $S$ by sets with a uniform 
  bound on their diameters possesses a locally finite subcover.
  
  We now specify six regularity properties that a domain $\Omega \subset \mathbb{R}^n$ may 
  possess. We denote by $\Omega_\delta$ the set of points in $\Omega$ within distance $\delta$ of 
  the boundary of $\Omega$:
  \[
    \Omega_\delta = \{x \in \Omega: \dist(x, \partial\Omega)<\delta\}.
  \]
\end{para}


\begin{para}[The Segment Condition]
  As defined in Paragraph~3.21, a domain $\Omega$ satisfies the segment condition
  if every $x \in\partial\Omega$ has a neighbourhood $U_x$ and a nonzero vector $y_x$
  such that if $z \in \overline{\Omega} \cap U_x$, then $z+t y_x \in \Omega$ for $0<t<1$.
  Since the boundary of $\Omega$ is necessarily closed, we can replace its open cover by the 
  neighbourhoods $U_x$ with a locally finite subcover $\left\{U_1, U_2, \ldots\right\}$ with 
  corresponding vectors $y_1, y_2, \ldots$ such that if $x \in \overline{\Omega} \cap U_j$ for some 
  $j$, then $x+t y_j \in \Omega$ for $0<t<1$.
\end{para}


\begin{para}[The Cone Condition]
  $\Omega$ satisfies the cone condition if there exists a finite cone $C$ such that
  each $x \in \Omega$ is the vertex of a finite cone $C_x$ contained in $\Omega$
  and congruent to $C$. Note that $C_x$ need not be obtained from $C$ by parallel translation,
  but simply by rigid motion.
\end{para}


\begin{para}[The Weak Cone Condition]
  Given $x \in \Omega$, let $R(x)$ consist of all points $y \in \Omega$ such that the line segment
  from $x$ to $y$ lies in $\Omega$; thus $R(x)$ is a union of rays and line segments emanating from $x$. Let
  \[
  \Gamma(x)=\{y \in R(x):|y-x|<1\}
  \]
  We say that $\Omega$ satisfies the weak cone condition if there exists $\delta>0$ such that
  \[
  \mu_n(\Gamma(x)) \geq \delta \quad \text { for all } x \in \Omega
  \]
  where $\mu_n$ is the Lebesgue measure in $\mathbb{R}^n$.
  Clearly the cone condition implies the 
  weak cone condition, but there are many domains satisfying the weak cone condition
  that do not satisfy the cone condition.
\end{para}


\begin{para}[The Uniform Cone Condition]
  $\Omega$ satisfies the uniform cone condition if there exists a locally finite open cover
  $\left\{U_j\right\}$ of the boundary of $\Omega$ and a corresponding sequence
  $\left\{C_j\right\}$ of finite cones, each congruent to some fixed finite cone $C$, such that
  \begin{enumerate}[(i)]
    \item There exists $M<\infty$ such that every $U_j$ has diameter less then $M$.
    \item $\Omega_\delta \subset \bigcup_{j=1}^{\infty} U_j$ for some $\delta>0$.
    \item $Q_j \equiv \bigcup_{x \in \Omega \cap U_j}\left(x+C_j\right) \subset \Omega$
      for every $j$.
    \item For some finite $R$, every collection of $R+1$ of the sets $Q_j$ has empty intersection.
  \end{enumerate}
\end{para}


\begin{para}[The Strong Local Lipschitz Condition]
  $\Omega$ satisfies the strong local Lipschitz condition if there exist positive numbers $\delta$ and $M$, a locally finite open cover $\left\{U_j\right\}$ of $\partial\Omega$, and, for each $j$ a real-valued function $f_j$ of $n-1$ variables, such that the following conditions hold:
  
  \begin{enumerate}[(i)]
    \item For some finite $R$, every collection of $R+1$ of the sets $U_j$ has empty intersection.
    \item For every pair of points $x, y \in \Omega_\delta$ such that $|x-y|<\delta$,
      there exists $j$ such that
      \[
        x, y \in V_j \equiv\left\{x \in U_j: \dist\left(x, \partial U_j\right)>\delta\right\}
      \]
    \item Each function $f_j$ satisfies a Lipschitz condition with constant $M$:
      that is, if $\xi=\left(\xi_1, \ldots, \xi_{n-1}\right)$ and
      $\rho=\left(\rho_1, \ldots, \rho_{n-1}\right)$ are in $\mathbb{R}^{n-1}$, then
      \[|f(\xi)-f(\rho)| \leq M|\xi-\rho|.\]
    \item For some Cartesian coordinate system $\left(\zeta_{j, 1}, \ldots, \zeta_{j,n}\right)$
      in $U_j, \Omega \cap U_j$ is represented by the inequality
      \[
        \zeta_{j, n}<f_j\left(\zeta_{j, 1}, \ldots, \zeta_{j, n-1}\right)
      \]
  \end{enumerate}
  If $\Omega$ is bounded, the rather complicated set of conditions above reduce to the simple condition that $\Omega$ should have a locally Lipschitz boundary, that is, that each point $x$ on the boundary of $\Omega$ should have a neighbourhood $U_x$ whose intersection with $\partial\Omega$ should be the graph of a Lipschitz continuous function.
\end{para}


\begin{para}[The Uniform $\bm{C^m}$-Regularity Condition]
  $\Omega$ satisfies the uniform $C^m$ regularity condition is there exists a locally finite open cover $\left\{U_j\right\}$ of $\partial\Omega$, and a corresponding sequence $\left\{\Phi_j\right\}$ of $m$-smooth transformations (see Paragraph 3.40) with $\Phi_j$ taking $U_j$ onto the ball $B=\left\{y \in \mathbb{R}^n:|y|<1\right.$ and having inverse $\Psi_j=\Phi_j^{-1}$, such that:
  \begin{enumerate}[(i)]
    \item For some finite $R$. every collection of $R+1$ of the sets $U_j$ has empty intersection.
    \item For some $\delta>0, \Omega_\delta \subset \bigcup_{j=1}^{\infty} \Psi_j\left(\left\{y \in \mathbb{R}^n:|y|<\frac{1}{2}\right\}\right)$.
    \item For each $j, \Phi_j\left(U_j \cap \Omega\right)=\left\{y \in B: y_n>0\right\}$.
    \item If $\left(\phi_{j, 1}, \ldots, \phi_{j, n}\right)$ and $\left(\psi_{j, 1}, \ldots, \psi_{j, n}\right)$ 
      are the components of $\Phi_j$ and $\Psi_j$, then there is a finite constant $M$ such that for every  
      $\alpha$ with $0<|\alpha| \leq m$, every $i, 1 \leq i \leq n$, and every $j$ we have
    \[
    \begin{array}{ll}
    \left|D^\alpha \phi_{j, i}(x)\right| \leq M, & \text { for } x \in U_j, \\
    \left|D^\alpha \psi_{j, i}(y)\right| \leq M, & \text { for } y \in B .
    \end{array}
    \]
  \end{enumerate}
\end{para}


\begin{para}
  Except for the cone condition and the weak cone condition,
  the other conditions defined above all require that the boundary of $\Omega$ be $(n-1)$-dimensional
  and that $\Omega$ lie on only one side of its boundary. The domain $\Omega$ of Example~3.20
  satisfies the cone condition (and therefore the weak cone condition), but none of the other four conditions. 
  Among those four we have:
  \begin{align*}
    & \text{the uniform $C^m$-regularity condition $(m \geq 2)$} \\
    & \text{$\Longrightarrow$ the strong local Lipschitz condition} \\
    & \text{$\Longrightarrow$ the uniform cone condition} \\
    & \text{$\Longrightarrow$ the segment condition.}
  \end{align*}
  Also,
  \begin{align*}
    & \text{the uniform cone condition} \\
    & \text{$\Longrightarrow$ the cone condition} \\
    & \text{$\Longrightarrow$ the weak cone condition}
  \end{align*}
  Typically, most of the imbeddings of $W^{m,p}(\Omega)$ have been proven for domains
  satisfying the cone condition. Exceptions are the imbeddings into spaces $C^j(\overline{\Omega})$
  and $C^{j, \lambda}(\overline{\Omega})$ of uniformly continuous functions which, as suggested by 
  Example 3.20, require that $\Omega$ lie on one side of its boundary. These imbeddings are 
  usually proved for domains satisfying the strong local Lipschitz condition. It should be noted, 
  however, that $\Omega$ need not satisfy any of these conditions for appropriate imbeddings of 
  $W_0^{m, p}(\Omega)$ to be valid.
\end{para}

\begin{theorem}[The Sobolev Imbedding Theorem]
  Let $\Omega$ be a domain in $\mathbb{R}^n$ and, for $1\leq k\leq n$,
  let $\Omega_k$ be the intersection of $\Omega$ with a plane of dimension $k$
  in $\mathbb{R}^n$. (If $k=n$, then $\Omega_k = \Omega.$)
  Let $j\geq 0$ and $m\geq 1$ be integers and let $1\leq p < \infty$.

  \noindent\textbf{PART I} Suppose $\Omega$ satisfies the cone condition.

  \noindent\textbf{Case A} If either $mp > n$ or $m=n$ and $p=1$, then
  \begin{equation}
    W^{j+m, p}(\Omega) \to C_B^j(\Omega).
  \end{equation}
  Moreover, if $1\leq k\leq n$, then
  \begin{equation}
    W^{j+m, p}(\Omega) \to W^{j, q}(\Omega_k)\qquad \text{for } p\leq q\leq\infty,
  \end{equation}
  and, in particular,
  \[W^{m,p}(\Omega) \to L^q(\Omega)\qquad \text{for } p\leq q\leq\infty.\]

  \noindent \textbf{Case B}
\end{theorem}


\begin{remarks}

\begin{enumerate}[label = \arabic*.]
  \item Imbeddings (1)-(4) are essentially due to Sobolev [So1, So2], although his original proof 
    did not cover the all cases. Imbeddings (6)-(7) originate in the work of Morrey [Mo].
  \item Imbeddings (2)-(4) involving traces of functions on planes of lower dimension can be      
    extended in a reasonable manner to apply to traces on more general smooth manifolds. For 
    example, see Theorem 5.36.
  \item If $\Omega_k$ (or $\Omega$ ) has finite volume, then imbeddings (2)-(4) also hold for
    $1 \leq q<p$ in addition to the values of $q$ asserted in the statement of the theorem.
    This follows from Theorem 2.14. It will be shown in Theorem 6.43 that no imbedding of the form 
    $W^{m,p}(\Omega) \rightarrow L^q(\Omega)$ where $q<p$ is possible unless $\Omega$ has finite 
    volume.
  \item Part III of the theorem is an immediate consequence of Parts I and II
    applied to $\mathbb{R}^n$ because, by Lemma 3.27 , the operator of zero extension of functions 
    outside $\Omega$ maps $W_0^{m, p}(\Omega)$ isometrically
    into $W^{m,p}\left(\mathbb{R}^n\right)$.
  \item More generally, suppose there exists an operator $E$ mapping $W^{m,p}(\Omega)$
    into $W^{m,p}\left(\mathbb{R}^n\right)$ such that $E u(x)=u(x)$ a.e. in $\Omega$ and such 
    that $\|E u\|_{m, p, \mathbb{R}^n} \leq K_1\|u\|_{m, p, \Omega}$.
    Such an operator is called an $(m, p)$ extension operator for $\Omega$.
    If the imbedding theorem has already been proved for $\mathbb{R}^n$,
    then it must hold for the domain $\Omega$ as well. For example, if
    $W^{m,p}\left(\mathbb{R}^n\right) \rightarrow L^q\left(\mathbb{R}^n\right)$,
    and $u \in W^{m,p}(\Omega)$, then
    \[
      \|u\|_{q, \Omega} \leq\|E u\|_{q, \mathbb{R}^n} \leq K_2\|E u\|_{m, p, \mathbb{R}^n}
      \leq K_2 K_1\|u\|_{m, p, \Omega} .
    \]
    In Chapter 5 we will establish the existence of such extension operators, but only for domains 
    satisfying conditions stronger than the cone condition, so we will not use such a technique to 
    prove Theorem 4.12.
  \item It is sufficient to prove imbeddings (1)-(4), (6)-(7) for the special case $j=0$,
    as the general case follows by applying this special case to derivative $D^\alpha u$ of $u$ 
    for $|\alpha| \leq j$. For example, if the imbedding $W^{m,p}(\Omega)\rightarrow L^q(\Omega)$ 
    has been proven, then for any $u \in W^{j+m, p}(\Omega)$
    we have $D^\alpha u \in W^{m,p}(\Omega)$ for $|\alpha| \leq j$,
    whence $D^\alpha u \in L^q(\Omega)$. Thus $u \in W^{j, q}(\Omega)$ and
    \[
    \begin{aligned}
    \|u\|_{j, q} & =\left(\sum_{|\alpha| \leq j}\left\|D^\alpha u\right\|_{0 . q}^q\right)^{1 / q} \\
    & \leq K_1\left(\sum_{|\alpha| \leq j}\left\|D^\alpha u\right\|_{m,p}^p\right)^{1 / p} \leq K_2\|u\|_{j+m, p} .
    \end{aligned}
    \]
  \item The authors have shown that all of Part I can be proved for domains satisfying only the 
    weak cone condition instead of the cone condition. See [AF1].
\end{enumerate}
\end{remarks}



\begin{para}[Strategy for Proving the Imbedding Theorem]
  We use two overlapping methods to prove the imbeddings in Part I of Theorem 4.12.
  The first, potential theoretic in nature, was used by Sobolev. It works when $p>1$, and gives 
  the right order of growth of imbedding constants as $q \rightarrow \infty$ when $m p=n$;
  this will be useful in Chapter 7. Here we use the potential method to prove Case A and the 
  imbeddings in Cases B and C for $p>1$. The other approach is based on a combinatorial-averaging 
  argument due to Gagliardo [Ga1]. We will use it to establish Cases $\mathrm{B}$ and $\mathrm{C}$ 
  for $p=1$, though it could be adapted (with a bit more difficulty) to prove all of Part I.
  (See, in particular, Theorem 5.10 and the Remark following that theorem.)
  
  Part II of the theorem follows by sharpening certain estimates used in obtaining Case A of Part I.
  
  The entire proof of Theorem 4.12 is fairly lengthy and is broken down into several lemmas.
  Throughout we use $K$, and occasionally $K_1, K_2, \ldots$, to represent various constants that can
  depend on parameters of the spaces being imbedded. The values of these constants can change from line to line. 
  While stated for the cone condition, the potential method works verbatim under the weak cone condition as well.
\end{para}


\section{Imbeddings by Potential Arguments}


\begin{lemma}[A Local Estimate]
  Let domain $\Omega \subset \mathbb{R}^n$ satisfy the cone condition. There exists a constant $K$ 
  depending on $m, n$, and the dimensions $\rho$ and $\kappa$ of the cone $C$ specified in the 
  cone condition for $\Omega$ such that for every $u \in C^{\infty}(\Omega)$,
  every $x \in \Omega$, and every $r$ satisfying $0<r \leq \rho$, we have
  \begin{equation}
    \begin{aligned}
      & |u(x)| \leq K\Biggl(\sum_{|\alpha| \leq m-1} r^{|\alpha|-n} \int_{C_{x,r}}\left|D^\alpha u(y)\right| d y \\
      & \quad+\sum_{|\alpha|=m} \int_{C_{x,r}}\left|D^\alpha u(y)\right||x-y|^{m-n} d y\Biggr),
    \end{aligned}
  \end{equation}
  where $C_{x,r}=\left\{y \in C_x:|x-y| \leq r\right\}$. Here $C_x \subset \Omega$ is a cone 
  congruent to $C$ having vertex at $x$.
\end{lemma}

\begin{proof}
  We apply Taylor's formula with integral remainder,
  \[
  f(1)=\sum_{j=0}^{m-1} \frac{1}{j !} f^{(j)}(0)+\frac{1}{(m-1) !} \int_0^1(1-t)^{m-1} f^{(m)}(t) d t
  \]
  to the function $f(t)=u(t x+(1-t) y)$, where $x \in \Omega$ and $y \in C_{x,r}$. Noting that
  \[
  f^{(j)}(t)=\sum_{|\alpha|=j} \frac{j !}{\alpha !} D^\alpha u(t x+(1-t) y)(x-y)^\alpha
  \]
  where $\alpha !=\alpha_{1} ! \cdots \alpha_{n} !$ and $(x-y)^\alpha=\left(x_1-y_1\right)^{\alpha_1} \cdots\left(x_n-y_n\right)^{\alpha_n}$, we obtain
  \[
  \begin{aligned}
  |u(x)| \leq & \sum_{|\alpha| \leq m-1} \frac{1}{\alpha !}\left|D^\alpha u(y)\right||x-y|^{|\alpha|} \\
  & +\sum_{|\alpha|=m} \frac{m}{\alpha !}|x-y|^m \int_0^1(1-t)^{m-1}\left|D^\alpha u(t x+(1-t) y)\right| d t .
  \end{aligned}
  \]
  If $C$ has volume $c \rho^n$, then $C_{x,r}$ has volume $c r^n$.
  Integration of $y$ over $C_{x,r}$ leads to
  \[
  \begin{aligned}
  & c r^n|u(x)| \\
  & \quad \leq \sum_{|\alpha| \leq m-1} \frac{r^{|\alpha|}}{\alpha !} \int_{C_{x,r}}\left|D^\alpha u(y)\right| d y \\
  & \quad+\sum_{|\alpha|=m} \frac{m}{\alpha !} \int_{C_{x,r}}|x-y|^m d y \int_0^1(1-t)^{m-1}\left|D^\alpha u(t x+(1-t) y)\right| d t
  \end{aligned}
  \]
  In the final (double) integral we first change the order of integration, then substitute $z=t x+(1-t) y$,
  so that $z-x=(1-t)(y-x)$ and $d z=(1-t)^n d y$, to obtain, for that integral,
  \[
  \int_0^1(1-t)^{-n-1} d t \int_{C_{x,(1-t) r}}|z-x|^m\left|D^\alpha u(z)\right| d z
  \]
  A second change of order of integration now gives for the above integral
  \[
  \begin{aligned}
  & \int_{C_{x,r}}|x-z|^m\left|D^\alpha u(z)\right| d z \int_0^{1-(|z-x| / r)}(1-t)^{-n-1} d t \\
  & \quad \leq \frac{r^n}{n} \int_{C_{x,r}}|x-z|^{m-n}\left|D^\alpha u(z)\right| d z
  \end{aligned}
  \]
  Inequality (8) now follows immediately.
\end{proof}


\begin{para}[Proof of Part I, Case A of Theorem 4.12]
  As noted earlier, we can assume that $j=0$. Let $u \in W^{m,p}(\Omega) \cap C^{\infty}(\Omega)$ 
  and let $x \in \Omega$. We must show that
  \[
  |u(x)| \leq K\|u\|_{m,p} .
  \]
  For $p=1$ and $m=n$, this follows immediately from (8). For $p>1$ and $m p>n$,
  we apply Hölder's inequality to (8) with $r=\rho$ to obtain
  \[
  \begin{aligned}
  |u(x)| & \leq K\left(\sum_{|\alpha| \leq m-1} c^{1 / p^{\prime}} \rho^{|\alpha|-(n / p)}\left\|D^\alpha u\right\|_{p, C_{x, \rho}}\right. \\
  & \left.+\sum_{|\alpha|=m}\left\|D^\alpha u\right\|_{p, C_{x, \rho}}\left[\int_{C_{x, \rho}}|x-y|^{(m-n) p^{\prime}} d y\right]^{1 / p^{\prime}}\right),
  \end{aligned}
  \]
  where $c$ is the volume of $C_{x, 1}$ and $p^{\prime}=p /(p-1)$. The final integral is finite since $(m-n) p^{\prime}>-n$ when $m p>n$. Thus
  \[
  |u(x)| \leq K \sum_{|\alpha| \leq m}\left\|D^\alpha u\right\|_{p, C_{x, \rho}}
  \]
  and (9) follows because $C_{x, \rho} \subset \Omega$.
  
  Next observe that since any $u \in W^{m,p}(\Omega)$ is the limit of a Cauchy sequence of continuous functions 
  by Theorem 3.17, and since (9) implies this Cauchy sequence converges to a continuous function on $\Omega, u$ 
  must coincide with a continuous function a.e. on $\Omega$. Thus $u \in C_B^0(\Omega)$ and imbedding (1) is 
  proved.
  Now let $\Omega_k$ denote the intersection of $\Omega$ with a $k$-dimensional plane $H$,
  let $\Omega_{k, \rho}=\left\{x \in \mathbb{R}^n: \dist\left(x, \Omega_k\right)<\rho\right\}$, 
  and let $u$ and all its derivatives be extended to be zero outside $\Omega$.
  Since $C_{x, \rho} \subset B_\rho(x)$, the ball of radius $\rho$ with centre at $x$,
  we have, using (10) and denoting by $d x^{\prime}$ the $k$-volume element in $H$,
  \[
  \begin{aligned}
  \int_{\Omega_k}|u(x)|^p \d x^{\prime} & \leq K \sum_{|\alpha| \leq m} \int_{\Omega_k} \d x^{\prime} \int_{B_\rho(x)}\left|D^\alpha u(y)\right|^p d y \\
  & =K \sum_{|\alpha| \leq m} \int_{\Omega_{k, \rho}}\left|D^\alpha u(y)\right|^p d y \int_{H \cap B_p(y)} \d x^{\prime} \leq K_1\|u\|_{m, p, \Omega}^p,
  \end{aligned}
  \]
  and $W^{m,p}(\Omega) \rightarrow L^p(\Omega_k)$. But (9) shows that $W^{m,p}(\Omega) \rightarrow L^{\infty}(\Omega_k)$ and so imbedding (2) follows by Theorem 2.11.
\end{para}

Let $\chi_r$ be the characteristic function of the ball $B_r(0)=\left\{x \in \mathbb{R}^n:|x|<r\right\}$.
In the following discussion we will develop estimates for convolutions of $L^p$ functions with the kernels 
$\omega_m(x)=|x|^{m-n}$ and
\[
\chi_r \omega_m(x)= \begin{cases}|x|^{m-n} & \text { if }|x|<r, \\ 0 & \text { if }|x| \geq r .\end{cases}
\]
Observe that if $m \leq n$ and $0<r \leq 1$, then
\[
\chi_r(x) \leq \chi_r \omega_m(x) \leq \omega_m(x).
\]


\begin{lemma}
  Let $p \geq 1,1 \leq k \leq n$, and $n-m p<k$. There exists a constant $K$ such that for every $r>0$,
  every $k$-dimensional plane $H \subset \mathbb{R}^n$, and every $v \in L^p\left(\mathbb{R}^n\right)$,
  we have $\chi_r \omega_m *|v| \in L^p(H)$ and
  \[
  \left\|\chi_r \omega_m *|v|\right\|_{p, H} \leq K r^{m-(n-k) / p}\|v\|_{p, \mathbb{R}^n}
  \]
  In particular,
  \[
  \left\|\chi_1 *|v|\right\|_{p, H} \leq\left\|\chi_1 \omega_m *|v|\right\|_{p, H} \leq K\|v\|_{p, \mathbb{R}^n}
  \]
\end{lemma}

\begin{proof}
  If $p>1$, then by Hölder's inequality
  \[
  \begin{aligned}
  \chi_r \omega_m *|v|(x) & =\int_{B_r(x)}|v(y)||x-y|^{-s}|x-y|^{s+m-n} d y \\
  & \leq\left(\int_{B_r(x)}|v(y)|^p|x-y|^{-s p} d y\right)^{1 / p}\left(\int_{B_r(x)}|x-y|^{(s+m-n) p^{\prime}} d y\right)^{1 / p^{\prime}} \\
  & =K r^{s+m-(n / p)}\left(\int_{B_r(x)}|v(y)|^p|x-y|^{-s p} d y\right)^{1 / p}
  \end{aligned}
  \]
  provided $s+m-(n / p)>0$. If $p=1$ the same estimate holds provided $s+m-n \geq 0$
  without using Hölder's inequality.
  
  Integrating the $p$ th power of the above estimate over $H$ (with volume element $\left.d x^{\prime}\right)$, we obtain
  \[
  \begin{aligned}
  \left\|\chi_r \omega_m *|v|\right\|_{p, H}^p & =\int_H\left|\chi_r \omega_m *\right| v|(x)|^p \d x^{\prime} \\
  & \leq K r^{(s+m) p-n} \int_H \d x^{\prime} \int_{B_r(x)}|v(y)|^p|x-y|^{-s p} d y \\
  & \leq K r^{(s+m) p-n} r^{k-s p}\|v\|_{p, \mathbb{R}^n}^p=K r^{m p-(n-k)}\|v\|_{p, \mathbb{R}^n}^p
  \end{aligned}
  \]
  provided $k>s p$
  Since $n-m p<k$ there exists $s$ satisfying $(n / p)-m<s<k / p$, so both estimates above are valid and (11) holds.
\end{proof}


\begin{lemma}
  Let $p>1, m p<n, n-m p<k \leq n$, and $p^*=k p /(n-m p)$. There exists a constant $K$ such that for every $k$-dimensional plane $H$ in $\mathbb{R}^n$ and every $v \in L^p\left(\mathbb{R}^n\right)$, we have $\omega_m *|v| \in L^{p^*}(H)$ and
  \[
  \left\|\chi_1 *|v|\right\|_{p^*, H} \leq\left\|\chi_1 \omega_m *|v|\right\|_{p^*, H} \leq\left\|\omega_m *|v|\right\|_{p^*, H} \leq K\|v\|_{p, \mathbb{R}^n}
  \]
\end{lemma}

\begin{proof}
  Only the final inequality of (12) requires proof. Since $m p<n$, for each $x \in \mathbb{R}^n$ Hölder's inequality gives
  \[
  \begin{aligned}
  \int_{\mathbb{R}^n-B_r(x)}|v(y) \| x-y|^{m-n} d y & \leq\|v\|_{p, \mathbb{R}^n}\left(\int_{\mathbb{R}^n-B_r(x)}|x-y|^{(m-n) p^{\prime}} d y\right)^{1 / p^{\prime}} \\
  & =K_1\|v\|_{p, \mathbb{R}^n}\left(\int_r^{\infty} t^{(m-n) p^{\prime}+n-1} d t\right)^{1 / p^{\prime}} \\
  & =K_1 r^{m-(n / p)}\|v\|_{p, \mathbb{R}^n}
  \end{aligned}
  \]
  If $t>0$, choose $r$ so that $K_1 r^{m-(n / p)}\|v\|_{p, \mathbb{R}^n}=t / 2$. If
  \[
  \omega_m *|v|(x)=\int_{\mathbb{R}^n}|v(y)||x-y|^{m-n} d y>t,
  \]
  then
  \[
  \chi_r \omega_m *|v|(x)=\int_{B_r(x)}|v(y)||x-y|^{m-n} d y>t / 2 .
  \]
  Thus
  \[
  \begin{aligned}
  \mu_k\left(\left\{x \in H: \omega_m *|v|(x)>t\right\}\right) & \leq \mu_k\left(\left\{x \in H: \chi_r \omega_m *|v|(x)>t / 2\right\}\right) \\
  & \leq\left(\frac{2}{t}\right)^p\left\|\chi_r \omega_m *|v|\right\|_{p, H}^p \\
  & \leq\left(\frac{r^{(n / p)-m}}{K_1\|v\|_{p, \mathbb{R}^n}}\right)^p K r^{m p-n+k}\|v\|_{p, \mathbb{R}^n}^p=K_2 r^k
  \end{aligned}
  \]
  by inequality (11). But $r^k=\left(2 K_1\|v\|_{p, \mathbb{R}^n} / t\right)^{p^*}$, so
  \[
  \mu_k\left(\left\{x \in H: \omega_m *|v|(x)>t\right\}\right) \leq K_2\left(\frac{2 K_1}{t}\|v\|_{p, \mathbb{R}^n}\right)^{p^*}
  \]
  Thus the mapping $I:\left.v \mapsto\left(\omega_m *|v|\right)\right|_H$ is of weak type $\left(p, p^*\right)$.
  For fixed $m, n, k$, the values of $p$ satisfying the conditions of this lemma constitute an open interval, so there exist $p_1$ and $p_2$ in that interval, and a number $\theta$ satisfying $0<\theta<1$ such that
  \[
  \frac{1}{p}=\frac{1-\theta}{p_1}+\frac{\theta}{p_2},
  \]
  and
  \[
  \frac{1}{p^*}=\frac{n / k}{p}-\frac{m}{k}=\frac{1-\theta}{p_1^*}+\frac{\theta}{p_2^*}.
  \]
  Since $p^*>p$, the Marcinkiewicz interpolation theorem 2.58 assures us that $I$ is bounded from $L^p\left(\mathbb{R}^n\right)$ into $L^{p^*}(H)$, that is, (12) holds.
\end{proof}


\begin{para}[Proof of Part I, Case C of Theorem 4.12 for $\bm{p>1}$]
  We have $mp < n$, $n - mp < k \leq n$, and $p \leq q \leq p *=k p /(n-m p)$.
  Let $u \in C^{\infty}(\Omega)$ and extend $u$ and all its derivatives to be zero
  on $\mathbb{R}^n-\Omega$. Taking $r=\rho$ in Lemma 4.15 and replacing $C_{x,r}$ with the larger 
  ball $B_1(x)$, we obtain
  \[
  |u(x)| \leq K\left(\sum_{|\alpha| \leq m-1} \chi_1 *\left|D^\alpha u\right|(x)+\sum_{|\alpha|=m} \chi_1 \omega_m *\left|D^\alpha u\right|(x)\right)
  \]
  If $1 / q=\theta / p+(1-\theta) / p^*$ where $0 \leq \theta \leq 1$, then by the interpolation inequality of Theorem 2.11 and Lemmas 4.17 and 4.18
  \[
  \begin{aligned}
  \|u\|_{q, \Omega_k} & \leq\|u\|_{p, H}^\theta\|u\|_{p^*, H}^{1-\theta} \\
  & \leq K\left(\sum_{|\alpha| \leq m}\left\|D^\alpha u\right\|_{p, \mathbb{R}^n}\right)^\theta\left(\sum_{|\alpha| \leq m}\left\|D^\alpha u\right\|_{p, \mathbb{R}^n}\right)^{1-\theta} \\
  & \leq K\|u\|_{m, p, \Omega}
  \end{aligned}
  \]
  as required.
\end{para}


\begin{para}[Proof of Part I, Case B of Theorem 4.12 for $\bm{p>1}$]
  We have $m p=n$, $1 \leq k \leq n$, and $p \leq q<\infty$. We can select numbers $p_1, p_2$, and $\theta$ such that $1<p_1<p<p_2, n-m p_1<k, 0<\theta<1$, and
  \[
  \frac{1}{p}=\frac{\theta}{p_1}+\frac{1-\theta}{p_2}, \quad \frac{1}{q}=\frac{\theta}{p_1}
  \]
  As in the above proof of Case $C$ for $p>1$, the maps $\left.v \mapsto\left(\chi_1 *|v|\right)\right|_H$
  and $\left.v \mapsto\left(\chi_1 \omega_m *|v|\right)\right|_H$ are bounded
  from $L^{p_1}\left(\mathbb{R}^n\right)$ into $L^{p_1}\left(\mathbb{R}^k\right)$ and so are of weak
  type $\left(p_1, p_1\right)$. As in the proof of Case $\mathrm{A}$, these same maps are bounded
  from $L^{p_2}\left(\mathbb{R}^n\right)$ into $L^{\infty}\left(\mathbb{R}^k\right)$ and so are of
  weak type $\left(p_2, \infty\right)$. By the Marcinkiewicz theorem again, they are bounded
  from $L^p\left(\mathbb{R}^n\right)$ into $L^q\left(\mathbb{R}^k\right)$ and
  \[
  \left\|\chi_1 *|v|\right\|_{q, H} \leq\left\|\chi_1 \omega_m *|v|\right\|_{q, H} \leq K\|v\|_{p, \mathbb{R}^n}
  \]
  and the desired result follows by applying these estimates to the various terms of (13).
\end{para}


\section{Imbeddings by Averaging}

\begin{para}
  We still need to prove the imbeddings for Cases $\mathrm{B}$ and $\mathrm{C}$ with $p=1$.
  We first prove that $W^{1,1}(\Omega) \rightarrow L^{n /(n-1)}(\Omega)$ and deduce from this and 
  the imbeddings already established for $p>1$ that all but one of the remaining imbeddings in 
  Cases $\mathrm{B}$ and $\mathrm{C}$ are valid. The remaining imbedding is the special case
  of $\mathrm{C}$ where $k=n-m$, $p=1$, $p^*=1$ which will require a special proof.
  
  We first show that any domain satisfying the cone condition is the union of finitely many 
  subdomains each of which is a union of parallel translates of a fixed parallelepiped.
  Then we establish a special case of a combinatorial lemma estimating a function in terms of
  averages in coordinate directions. Both of these results are due to Gagliardo [Ga1] and
  constitute the foundation on which rests his proof of all of Cases B and C of Part I.
\end{para}


\begin{lemma}[Decomposition of $\bm{\Omega}$]
  Let $\Omega \subset \mathbb{R}^n$ satisfy the cone condition. Then there exists a finite 
  collection $\left\{\Omega_1, \ldots, \Omega_N\right\}$ of open subsets of $\Omega$ such
  that $\Omega=\bigcup_{j=1}^N \Omega_j$, and such that to each $\Omega_j$ there corresponds a 
  subset $A_j \subset \Omega_j$ and an open parallelepiped $P_j$ with one vertex at 0 such that 
  $\Omega_j=\bigcup_{x \in A_j}\left(x+P_j\right)$.
  
  If $\Omega$ is bounded and $\rho>0$ is given, we can accomplish the above decomposition using 
  sets $A_j$ each satisfying $\diam\left(A_j\right)<\rho$.
  
  Finally, if $\Omega$ is bounded and $\rho>0$ is sufficiently small, then each $\Omega_j$ will 
  satisfy the strong local Lipschitz condition.
\end{lemma}

\begin{proof}
  Let $C$ be the finite cone with vertex at 0 such that any $x \in \Omega$ is the vertex of a finite
  cone $C_x \subset \Omega$ congruent to $C$. We can select a finite number of finite cones
  $C_1, \ldots, C_N$ each having vertex at 0 (and each having the same height as $C$ but smaller
  aperture angle than $C$ ) such that any finite cone congruent to $C$ and having vertex at 0
  must contain one of the cones $C_j$. For each $j$, let $P_j$ be an open parallelepiped with
  one vertex at the origin, contained in $C_j$, and having positive volume.
  Then for each $x \in \Omega$ there exists $j, 1 \leq j \leq N$, such that
  \[
  x+P_j \subset x+C_j \subset C_x \subset \Omega .
  \]
  Since $\Omega$ is open and $\overline{x+P_j}$ is compact, $y+P_j \subset \Omega$ for any $y$
  sufficiently close to $x$. Hence every $x \in \Omega$ belongs to $y+P_j$ for some $j$ and some $y \in \Omega$. 
  Let $A_j=\left\{y \in \overline{\Omega}: y+P_j \subset \Omega\right\}$ and
  let $\Omega_j=\bigcup_{y \in A_j}\left(y+P_j\right)$. Then $\Omega=\bigcup_{j=1}^N \Omega_j$.
  Now suppose $\Omega$ is bounded and $\rho>0$ is given. If $\diam\left(A_j\right) \geq \rho$
  we can decompose $A_j$ into a finite union of sets $A_{j i}$ each with diameter less than $\rho$
  and define the corresponding parallelepiped $P_{j i}=P_j$. We then rename the totality of such
  sets $A_{j i}$ as a single finite family, which we again call $\left\{A_j\right\}$ and
  define $\Omega_j$ as above.
  
  Figure 2 attempts to illustrate these notions for the domain in $\mathbb{R}^2$ considered in
  Example 3.20:
  \begin{align*}
    \Omega & = \{(x,y)\in \mathbb{R}^2 : 0 < |x| < 1, 0 < y < 1\}, \\
    C      & = \{(x,y)\in \mathbb{R}^2 : x > 0, y > 0, x^2 + y^2 < 1/4\}, \\
    \rho   & < 0.98.
  \end{align*}
  Finally, we show that if $\rho$ is sufficiently small, then $\Omega_j$
  satisfies the strong local Lipschitz condition. For simplicity of notation,
  let $G = \bigcup_{x\in A} (x+P)$, where $\diam(A)<\rho$ and $P$ is a fixed
  parallelepiped. We show that $G$ satisfies the strong local Lipschitz
  condition if $\rho$ is suitably small. For each vertex $v_j$ of $P$
  let $Q_j = \{y = v_j + \lambda(x-v_j) : x\in P, \lambda>0\}$ be the infinite pyramid
  with vertex $v_j$ generated by $P$. Then $P = \bigcap Q_j$, the intersection
  being taken over all $2^n$ vertices of $P$.
  Let $G_j = \bigcup_{x\in A}(x+Q_j)$. Let $\delta$ be the distance from the centre
  of $P$ to the boundary of $P$ and let $B$ be an arbitrary ball of radius $\sigma = \delta/2$.
  For any fixed $x\in G$, $B$ cannot intersect opposite faces of $x+P$ so we
  may pick a vertexx $v_j$ of $P$ with the property that $x+v_j$ is common
  to all faces of $x+P$ that intersect $B$, if any such faces exist.
  Then $B\cap (x+P) = B\cap (x+Q_j)$. Now let $x,y\in A$ and suppose $B$
  could intersect relatively opposite faces of $x+P$ and $y+P$,
  that is, there exist points $a$ and $b$ on opposite faces of $P$ such that
  $x+a\in B$ and $y+b\in B$. Then
  \begin{align*}
    \rho \geq \dist(x,y)
    & = \dist(x+b, y+b) \\
    & \geq \dist(x+b, x+a) - \dist(x+a, y+b) \\
    & \geq 2\delta - 2\sigma = \delta.
  \end{align*}
  It follows that if $\rho<\delta$, then $B$ cannot intersect relatively opposite faces of $x+P$ and $y+P$ for any $x, y \in A$. Thus $B \cap(x+P)=B \cap\left(x+Q_j\right)$ for some fixed $j$ independent of $x \in A$, whence $B \cap G=B \cap G_j$.
  
  Choose coordinates $\xi=\left(\xi^{\prime}, \xi_n\right)=\left(\xi_1, \ldots, \xi_{n-1}, \xi_n\right)$ in $B$ so that the $\xi_n$-axis lies in the direction of the vector from the centre of $P$ to the vertex $v_j$. Then $B \cap\left(x+Q_j\right)$ is specified in $B$ by an inequality of the form $\xi_n<f_x\left(\xi^{\prime}\right)$ where $f_x$ satisfies a Lipschitz condition with constant independent of $x$. Thus $B \cap G_j$, and hence $B \cap G$, is specified by $\xi_n<f\left(\xi^{\prime}\right)$, where $f\left(\xi^{\prime}\right)=\sup _{x \in A} f_x\left(\xi^{\prime}\right)$ is itself a Lipschitz continuous function. Since this can be done for a neighbourhood $B$ of any point on the boundary of $G$, it follows that $G$ satisfies the strong local Lipschitz condition.
\end{proof}


\begin{lemma}[An Averaging Lemma]
  Let $\Omega$ be a domain in $\mathbb{R}^n$ where $n \geq 2$. Let $k$ be an integer satisfying $1 \leq k \leq n$, and let $\kappa=\left(\kappa_1, \ldots, \kappa_k\right)$ be a $k$-tuple of integers satisfying $1 \leq \kappa_1<\kappa_2<\cdots<\kappa_k \leq n$. Let $S$ be the set of all $\left(\begin{array}{l}n \\ k\end{array}\right)$ such $k$-tuples. Given $x \in R^n$, let $x_\kappa$ denote the point $\left(x_{\kappa_1}, \ldots, x_{\kappa_k}\right)$ in $\mathbb{R}^k$ and let $d x_\kappa=d x_{\kappa_1} \cdots d x_{\kappa_k}$.
  For $\kappa \in S$ let $E_\kappa$ be the $k$-dimensional plane in $\mathbb{R}^n$ spanned by the coordinate axes corresponding to the components of $x_\kappa$ :
  \[
  E_\kappa=\left\{x \in \mathbb{R}^n: x_i=0 \text { if } i \notin \kappa\right\}
  \]
  and let $\Omega_\kappa$ be the projection of $\Omega$ onto $E_\kappa$ :
  \[
  \Omega_\kappa=\left\{x \in E_\kappa: x_\kappa=y_\kappa \text { for some } y \in \Omega\right\} \text {. }
  \]
  Let $F_\kappa\left(x_\kappa\right)$ be a function depending only on the $k$ components of $x_\kappa$ and belonging to $L^\lambda\left(\Omega_\kappa\right)$, where $\lambda=\left(\begin{array}{l}n-1 \\ k-1\end{array}\right)$. Then the function $F$ defined by
  \[
  F(x)=\prod_{\kappa \in S} F_\kappa\left(x_\kappa\right)
  \]
  belongs to $L^1(\Omega)$, and $\|F\|_{1, \Omega} \leq \prod_{\kappa \in S}\left\|F_\kappa\right\|_{\lambda, \Omega_k}$, that is,
  \[
  \left(\int_{\Omega}|F(x)| \d x\right)^\lambda \leq \prod_{\kappa \in S} \int_{\Omega_\kappa}\left|F_\kappa\left(x_\kappa\right)\right|^\lambda \d x_\kappa .
  \]
\end{lemma}

\begin{proof}
  We use the mixed-norm Hölder inequality of Paragraph 2.49 to provide the proof. For each $\kappa \in S$ let $\mathbf{p}_\kappa$ be the $n$-vector whose $i$ th component is $\lambda$ if $i \in \kappa$ and $\infty$ if $i \notin \kappa$. For each $i, 1 \leq i \leq n$, exactly $(k / n)\left(\begin{array}{l}n \\ k\end{array}\right)=\lambda$ of the vectors $\mathbf{p}_\kappa$ have $i$ th component equal to $\lambda$. Therefore, in the notation of Paragraph 2.49
  \[
  \sum_{\kappa \in S} \frac{1}{\mathbf{p}_\kappa}=\frac{1}{\mathbf{w}}
  \]
  where $\mathbf{w}$ is the $n$-vector $(1,1, \ldots, 1)$.
  Let $F_\kappa\left(x_\kappa\right)$ be extended to be zero for $x_\kappa \notin \Omega_\kappa$ and consider $F_\kappa$ to be defined on $\mathbb{R}^n$ but independent of $x_j$ if $j \notin \kappa$. Then $F_\kappa$ is its own supremum over those $x_j$ and
  \[
  \left\|F_\kappa\right\|_{\lambda, \Omega_\kappa}=\left\|F_\kappa\right\|_{\mathbf{p}_\kappa, \mathbb{R}^n}
  \]
  From the mixed-norm Hölder inequality
  \[
  \|F\|_{1, \Omega} \leq\|F\|_{\mathbf{w}, \mathbb{R}^n} \leq \prod_{\kappa \in S}\left\|F_\kappa\right\|_{\mathbf{p}_\kappa, \mathbb{R}^n}=\prod_{\kappa \in S}\left\|F_\kappa\right\|_{\lambda, \Omega_\kappa}
  \]
  as required.
\end{proof}


\begin{lemma}
  If $\Omega$ satisfies the cone condition, then $W^{1,1}(\Omega) \rightarrow L^p(\Omega)$
  for $1 \leq p \leq n /(n-1)$.
\end{lemma}

\begin{proof}
  By Lemma 4.22, $\Omega$ is a finite union of subdomains each of which is a union of parallel 
  translates of a fixed parallelepiped. It is therefore sufficient to prove the imbedding for one 
  such subdomain. Thus we assume $\Omega=\bigcup_{x \in A}(x+P)$ where $P$ is a parallelepiped. 
  There is a linear transformation of $\mathbb{R}^n$ onto itself that
  maps $P$ onto a cube $Q$ of unit edge with edges parallel to the coordinate axes.
  By Theorem 3.41 it is therefore sufficient to prove the lemma
  for $\Omega=\bigcup_{x \in A}(x+Q)$. For $x \in \Omega$ let $\ell$ be the intersection of 
  $\Omega$ with the line through $x$ parallel to the $x_1$-axis. Evidently $\ell$ contains a 
  closed interval of length 1 containing $x_1$, say the interval $\left[\xi_1, \xi_2\right]$.
  If $f \in C^1([0,1])$, then $\left|f\left(t_0\right)\right| \leq|f(t)|+\left|\int_{t_0}^t f^{\prime}(\tau) d \tau\right|$, and integrating over $t$ over $[0,1]$ yields
  \[
  \left|f\left(t_0\right)\right|
    \leq \int_0^1\left(|f(t)|+\left|f^{\prime}(t)\right|\right) \d t.
  \]
  For $u \in C^{\infty}(\Omega)$ we apply this inequality to $u\left(t, \hat{x}_1\right)$ (where $\left.\hat{x}_1=\left(x_2, \ldots, x_n\right)\right)$ to obtain
  \[
  \begin{aligned}
  |u(x)| & \leq \int_{\xi_1}^{\xi_2}\left(\left|u\left(t, \hat{x}_1\right)\right|+\left|D_1 u\left(t, \hat{x}_1\right)\right|\right) d t \\
  & \leq \int_{\ell}\left(\left|u\left(t, \hat{x}_1\right)\right|+\left|D_1 u\left(t, \hat{x}_1\right)\right|\right) d t .
  \end{aligned}
  \]
  Let $\Omega_1$ be the orthogonal projection of $\Omega$ onto the hyperplane of coordinates $\hat{x}_1$, and let
  \[
  u_1\left(\hat{x}_1\right)=\left(\int_{\ell}\left(\left|u\left(t, \hat{x}_1\right)\right|+\left|D_1 u\left(t, \hat{x}_1\right)\right|\right) d t\right)^{1 /(n-1)}
  \]
  (Evidently $u_1\left(\hat{x}_1\right)$ is independent of $x_1$ ) We have
  \[
  \left\|u_1\right\|_{1 /(n-1), \Omega_1}=\int_{\Omega_1}\left|u_1(x)\right|^{n-1} d \hat{x}_1 \leq\|u\|_{1,1, \Omega}
  \]
  Similarly, for $2 \leq j \leq n$ we can define $u_j$ to be independent of $x_j$ and to satisfy $|u(x)| \leq\left(u_j(x)\right)^{1 /(n-1)}$ and
  \[
  \left\|u_j\right\|_{1 /(n-1), \Omega_j} \leq\|u\|_{1,1, \Omega}
  \]
  Since $|u(x)|^{n /(n-1)} \leq \prod_{j=1}^n u_j(x)$, applying inequality (14) with $k=n-1=\lambda$ now gives
  \[
  \int_{\Omega}|u(x)|^{n /(n-1)} \d x \leq\left.\prod_{j=1}^n \int_{\Omega_j}\left|u_j\right|\left(\hat{x}_j\right)\right|^{n-1} d \hat{x}_j \leq\|u\|_{1,1, \Omega}^{n /(n-1)}
  \]
  For the original domain $\Omega$, this will imply that
  \[
  \|u\|_{n /(n-1), \Omega} \leq K\|u\|_{1,1, \Omega}
  \]
  where the constant $\mathrm{K}$ depends on $n$ and the cone $C$ of the cone condition. These determine the number $N$ of subdomains needed, and the size of the determinant of the linear transformation needed to transform the parallelepipeds for each subdomain into $Q$. The imbedding $W^{1,1}(\Omega) \rightarrow L^p(\Omega)$ for $1 \leq p \leq n /(n-1)$ now follows by $L^p$ interpolation (Theorem 2.11)
\end{proof}


\begin{para}[Proof of Part I, Cases B and C of Theorem 4.12 for $\bm{p=1}$, $\bm{k>n-m}$]
  Let $m \leq n$. By the above lemma and previously proved parts of Cases $\mathrm{B}$ and $\mathrm{C}$ for $p>1$, we have
  \[
  W^{m, 1}(\Omega) \rightarrow W^{m-1, p}(\Omega) \quad \text { for } 1 \leq p \leq n /(n-1)
  \]
  Since $k>n-m$, therefore $k \geq n-m+1>n-(m-1) p$ for any $p>1$. Therefore $W^{m-1, p}(\Omega) \rightarrow L^q(\Omega_k)$ holds for
  \[
  1 \leq q \leq p^*=\frac{k p}{n-(m-1) p}=\frac{k n /(n-1)}{n-(m-1) n /(n-1)}=\frac{k}{n-m}
  \]
  Combining these imbeddings we get $W^{m, 1}(\Omega) \rightarrow L^p(\Omega), 1 \leq q \leq k /(n-m)$.
  For $p=1, m=n$ the imbedding $W^{n, 1}(\Omega) \rightarrow L^q(\Omega_k), 1 \leq q \leq \infty, 1 \leq k \leq n$ was already proved under Case $A$.
\end{para}


\begin{para}[Proof of Part I, Case C of Theorem 4.12 for $\bm{p=1}$, $\bm{k=n-m}$]
  In this case we want to show $W^{m, 1}(\Omega) \rightarrow L^1(\Omega_k)$.
  As in the proof in Paragraph 4.24 it is sufficient to establish the imbedding for a domain 
  $\Omega$ that is a union of parallel translates of a unit cube with edges parallel to the 
  coordinate axes. We can also assume that $0 \in \Omega$ and that
  \[
  \Omega_k=\left\{x=\left(x^{\prime}, x^{\prime \prime}\right) \in \Omega: x^{\prime}=0\right\},
  \]
  where $x^{\prime}=\left(x_1, \ldots x_m\right)$ and $x^{\prime \prime}=\left(x_{m+1}, \ldots, x_n\right)$. For $x \in \Omega$ let $\Omega_x$ be the intersection of $\Omega$ with the $m$-plane of variables $x^{\prime}$ passing through $x . \Omega_x$ contains an $m$-cube $Q_x$ of edge 1 containing $x$, and so by Case A of Theorem 4.12 applied to this cube, we have for $u \in C^{\infty}(\Omega)$,
  \[
  |u(x)| \leq K \sum_{|\alpha| \leq m} \int_{\Omega_x}\left|D^\alpha u\left(x^{\prime}, x^{\prime \prime}\right)\right| \d x^{\prime} .
  \]
  Integrating $x^{\prime \prime}$ over $\Omega_k$ then gives
  \[
  \int_{\Omega_k}|u(x)| \d x^{\prime \prime} \leq K \sum_{|\alpha| \leq m} \int_{\Omega}\left|D^\alpha u(x)\right| \d x .
  \]
  The proof of Part I of Theorem 4.12 is now complete.
\end{para}


\section{Imbeddings into Lipschitz Spaces}


\begin{para}
  To prove Part II of Theorem 4.12 , we now assume that the domain $\Omega \subset \mathbb{R}^n$ satisfies the strong local Lipschitz condition defined in Paragraph 4.9 , and that $m p>n \geq(m-1) p$. We shall show that $W^{m,p}(\Omega) \rightarrow C^{0, \lambda}(\overline{\Omega})$ where:
  \begin{enumerate}[(i)]
    \item $0<\lambda \leq m-(n / p)$ if $n>(m-1) p$, or
    \item $0<\lambda<1$ if $n=(m-1) p$ and $p>1$, or
    \item $0<\lambda \leq 1$ if $n=m-1$ and $p=1$.
  \end{enumerate}
  In particular, therefore, $W^{m,p}(\Omega) \rightarrow C^0(\overline{\Omega})$. The imbedding constants may depend on $m, p, n$, and the parameters $\delta$ and $M$ specified in the definition of the strong local Lipschitz condition. Since that condition implies the cone condition, we already know that $W^{m,p}(\Omega) \rightarrow C_B^0(\Omega)$, so if $u \in W^{m,p}(\Omega)$, then
  \[
  \sup _{x \in \Omega}|u(x)| \leq K_1\|u\|_{m, p, \Omega}
  \]
  It is therefore sufficient to establish further that for the appropriate $\lambda$,
  \[
  \sup _{\substack{x, y \in \Omega \\ x \neq y}} \frac{|u(x)-u(y)|}{|x-y|^\lambda} \leq K_2\|u\|_{m, p, \Omega}
  \]
  Since $m p>n \geq(m-1) p$, Cases $\mathrm{B}$ and $\mathrm{C}$ of Part I of Theorem 4.12 yields the imbedding $W^{m,p}(\Omega) \rightarrow W^{1, r}(\Omega)$ where:
  \begin{enumerate}[(i)]
    \item $r=n p /(n-m+1) p$ and so $1-(n / r)=m-(n / p)$ if $n>(m-1) p$, or
    \item $p<r<\infty$ and so $0<1-(n / r)<1$ if $n>(m-1) p$, or
    \item $r=\infty$ and so $1-(n / r)=1$ if $n=m-1$ and $p=1$.
  \end{enumerate}
  It is therefore sufficient to establish the special case $m=1$.
\end{para}


\begin{lemma}
  Let $\Omega$ satisfy the strong local Lipschitz condition. If $u$ belongs to $W^{1, p}(\Omega)$ where $n<p \leq \infty$, and if $0 \leq \lambda \leq 1-(n / p)$, then
  \[
  \sup _{\substack{x, y \in \Omega \\ x \neq y}} \frac{|u(x)-u(y)|}{|x-y|^\lambda} \leq K\|u\|_{1, p, \Omega}
  \]
\end{lemma}

\begin{proof}
  Suppose, for the moment, that $\Omega$ is a cube having unit edge length. For $0<t<1$ let $Q_t$ denote a subset of $\Omega$ that is a closed cube having edge length $t$ and faces parallel to those of $\Omega$. If $x, y \in \Omega$ and $|x-y|=\sigma<1$, then there is a fixed such cube $Q_\sigma$ such that $x, y \in Q_\sigma$.
  Let $u \in C^{\infty}(\Omega)$ If $z \in Q_\sigma$, then
  \[
  u(x)-u(z)=\int_0^1 \frac{d}{d t} u(x+t(z-x)) \d t,
  \]
  so that
  \[
  |u(x)-u(z)| \leq \sigma \sqrt{n} \int_0^1 \mid \grad u((x+t(z-x)) \mid d t
  \]
  It follows that
  \[
  \begin{aligned}
  \left|u(x)-\frac{1}{\sigma^n} \int_{Q_\sigma} u(z) d z\right| & =\left|\frac{1}{\sigma^n} \int_{Q_\sigma}(u(x)-u(z)) d z\right| \\
  & \leq \frac{\sqrt{n}}{\sigma^{n-1}} \int_{Q_\sigma} d z \int_0^1 \mid \grad u((x+t(z-x)) \mid d t \\
  & =\frac{\sqrt{n}}{\sigma^{n-1}} \int_0^1 t^{-n} d t \int_{Q_{t \sigma}}|\grad u(\zeta)| d \zeta \\
  & \leq \frac{\sqrt{n}}{\sigma^{n-1}}\|\grad u\|_{0, p, \Omega} \int_0^1\left(\vol(Q)_{t \sigma}\right)^{1 / p^{\prime}} t^{-n} d t(16) \\
  & \leq K \sigma^{1-(n / p)}\|\grad u\|_{0, p, \Omega},
  \end{aligned}
  \]
  where $K=K(n, p)=\sqrt{n} \int_0^1 t^{-n / p} d t<\infty$. A similar inequality holds with $y$ in place of $x$ and so
  \[
  |u(x)-u(y)| \leq 2 K|x-y|^{1-(n / p)}\|\grad u\|_{0, p, \Omega} .
  \]
  It follows that (15) holds for $0<\lambda \leq 1-(n / p)$ for $\Omega$ a cube,
  and therefore via a nonsingular linear transformation, for $\Omega$ a parallelepiped.
  
  Now suppose that $\Omega$ is an arbitrary domain satisfying the strong local Lipschitz 
  condition. Let $\delta, M, \Omega_\delta, U_j$ and $V_j$ be as specified in the definition of 
  that condition in Paragraph 4.9. There exists a parallelepiped $P$ of diameter $\delta$ whose 
  dimensions depend only on $\delta$ and $M$ such that to each $j$ there corresponds a 
  parallelepiped $P_j$ congruent to $P$ and having one vertex at the origin,
  such that for every $x \in V_j \cap \Omega$ we have $x+P_j \subset \Omega$.
  Furthermore, there exist constants $\delta_0$ and $\delta_1$ depending only on $\delta$
  and $P$, with $\delta_0 \leq \delta$, such that if $x, y \in V_j \cap \Omega$
  and $|x-y|<\delta_0$, then there exists $z \in\left(x+P_j\right) \cap\left(y+P_j\right)$
  with $|x-z|+|y-z| \leq \delta_1|x-y|$. If follows from applications of (15) to $x+P_j$
  and $y+P_j$ that if $u \in C^{\infty}(\Omega)$, then
  \[
  \begin{aligned}
  |u(x)-u(y)| & \leq|u(x)-u(z)|+|u(y)-u(z)| \\
  & \leq K|x-z|^\lambda\|u\|_{1, p, \Omega}+K|y-z|^\lambda\|u\|_{1, p, \Omega} \\
  & \leq K_1|x-y|^\lambda\|u\|_{1, p, \Omega}
  \end{aligned}
  \]
  Now let $x, y$ be arbitrary points in $\Omega$. If $|x-y|<\delta_0 \leq \delta$
  and $x, y \in \Omega_\delta$, then $x, y \in V_j$ for some $j$ and (17) holds.
  If $|x-y|<\delta_0$, $x \in \Omega_\delta$, $y \in \Omega-\Omega_\delta$,
  then $x \in V_j$ for some $j$ and (17) still follows by an applications of (15)
  to $x+P_j$ and $y+P_j$. If $|x-y|<\delta_0, x, y \in \Omega-\Omega_\delta$,
  then (17) follows from applications of (15) to $x+P^{\prime}$ and $y+P^{\prime}$
  where $P^{\prime}$ is any parallelepiped congruent to $p$ and having one vertex at the origin. 
  Finally, if $|x-y| \geq \delta_0$, then
  \[
  |u(x)-u(y)| \leq|u(x)|+|u(y)| \leq K_1\|u\|_{1, p, \Omega} \leq K \delta_0^{-\lambda}|x-y|^\lambda\|u\|_{1, p, \Omega} .
  \]
  Thus (15) holds for all $u \in C^{\infty}(\Omega)$ and, by Theorem~3.17,
  for all $u \in C_B^0(\Omega)$.
\end{proof}

This completes the proof of Part II of Theorem 4.12 and therefore of the whole theorem since,
as remarked earlier, Part III follows from the fact that Parts I and II hold
for $\Omega=\mathbb{R}^n$.


\section{Sobolev's Inequality}


\begin{para}[Seminorms]
  For $1 \leq p<\infty$ and for integers $j, 0 \leq j \leq m$, we introduce functionals $|\cdot|_{j, p}$ on $W^{m,p}(\Omega)$ as follows:
  \[
  |u|_{j, p}=|u|_{j, p, \Omega}=\left(\sum_{|\alpha|=j}\left|D^\alpha u(x)\right|^p d x\right)^{1 / p} .
  \]
  Clearly $|u|_{0, p}=\|u\|_{0, p}=\|u\|_p$ is the norm on $L^p(\Omega)$ and
  \[
  \|u\|_{m,p}=\left(\sum_{j=0}^m|u|_{j, p}^p\right)^{1 / p}
  \]
  If $j \geq 1$, we call $|\cdot|_{j, p}$ a seminorm. It has all the properties of a norm except that $|u|_{j, p}=0$ need not imply $u=0$ in $W^{m,p}(\Omega)$. For example, $u$ may be a nonzero constant function if $\Omega$ has finite volume. Under certain circumstances which we begin to investigate in Paragraph $6.29,|\cdot|_{m,p}$ is a norm on $W_0^{m, p}(\Omega)$ equivalent to the usual norm $\|\cdot\|_{m,p}$. In particular, this is so if $\Omega$ is bounded.

  For now we will confine our attention to these seminorms as they apply to functions in $C_0^{\infty}\left(\mathbb{R}^n\right)$.
\end{para}


\begin{para}
  The Sobolev imbedding theorem tells us that $W_0^{m, p}\left(\mathbb{R}^n\right) \rightarrow L^q\left(\mathbb{R}^n\right)$ for certain finite values of $q$ depending on $m, p$, and $n$; for such $q$ there is a finite constant $K$ such that for all $\phi \in C_0^{\infty}\left(\mathbb{R}^n\right)$ we have
  \[
  \|\phi\|_q \leq K\|\phi\|_{m,p}
  \]
  We now ask whether such an inequality can hold with $|\cdot|_{m,p}$ in place of $\|\cdot\|_{m,p}$. That is, do there exist constants $K<\infty$ and $q \geq 1$ such that for all $\phi \in C_0^{\infty}\left(\mathbb{R}^n\right)$
  \[
  \int_{\mathbb{R}^n}|\phi(x)|^q \d x \leq K^q\left(\sum_{|\alpha|=m} \int_{\mathbb{R}^n}\left|D^\alpha \phi(x)\right|^p \d x\right)^{q / p} ?
  \]
  If so, for any given $\phi \in C_0^{\infty}\left(\mathbb{R}^n\right)$, the inequality must also hold
  for all dilates $\phi_t(x)=\phi(t x), 0<t<\infty$, as these functions also belong
  to $C_0^{\infty}\left(\mathbb{R}^n\right)$. Since $\left\|\phi_t\right\|_q=t^{-n / q}\|\phi\|_q$
  and $\left\|D^\alpha \phi_t\right\|_p=t^{m-(n / p)}\left\|D^\alpha \phi\right\|_p$ if $|\alpha|=m$,
  we must have
  \[
  \int_{\mathbb{R}^n}|\phi(x)|^q \d x \leq K^q t^{n+m q-(n q / p)}\left(\sum_{|\alpha|=m} \int_{\mathbb{R}^n}\left|D^\alpha \phi(x)\right|^p \d x\right)^{q / p}
  \]
  This is clearly not possible for all $t>0$ unless the exponent of $t$ on the right side is zero,
  that is, unless $q=p^*=n p /(n-m p)$. Thus no inequality of the form (18) is possible unless $m p<n$
  and $q=p^*=n p /(n-m p)$. We now show that (18) does hold if these conditions are satisfied.
\end{para}


\begin{theorem}[Sobolev's Inequality]
  When $m p<n$, there exists a finite constant $K$ such that (18) holds for
  every $\phi \in C_0^{\infty}\left(\mathbb{R}^n\right)$ :
  \[
  \|\phi\|_{q, \mathbb{R}^n} \leq K|\phi|_{m, p, \mathbb{R}^n}
  \]
  if and only if $q=p^*=n p /(n-m p)$. This is known as \emph{Sobolev's inequality}.
\end{theorem}

\begin{proof}
  The ``only if'' part was demonstrated above. For the ``if'' part note first that it is sufficient
  to establish the inequality for $m=1$ as its validity for higher $m$ (with $m p<n$ ) can be confirmed
  by induction on $m$. We leave the details to the reader.
  Next, it suffices to prove the case $m=1, p=1$, that is
  \[
  \int_{\mathbb{R}^n}|\phi(x)|^{n /(n-1)} \d x \leq K\left(\sum_{j=1}^n \int_{\mathbb{R}^n}\left|D_j \phi(x)\right| \d x\right)^{n /(n-1)}
  \]
  for if $1<p<n$ and $p^*=n p /(n-p)$ we can apply (20) to $|\phi(x)|^s$ where $s=(n-1) p^* / n$ and obtain, using Hölder's inequality,
  \[
  \begin{aligned}
  \int_{\mathbb{R}^n}|\phi(x)|^{p^*} \d x & \leq K\left(\sum_{j=1}^n s|\phi(x)|^{s-1}\left|D_j \phi(x)\right| d x\right)^{n /(n-1)} \\
  & \leq K_1\left(\sum_{j=1}^n\|\phi\|_{(s-1) p^{\prime}}^{s-1}\left\|D_j \phi\right\|_p\right)^{n /(n-1)}
  \end{aligned}
  \]
  Since $(s-1) p^{\prime}=p^*$ and $p^*-(s-1) n /(n-1)=n /(n-1)$, it follows by cancellation that
  \[\|\phi\|_{p^*} \leq K_2 |\phi|_{1,p}.\]
  It remains, therefore, to prove $(20)$. Let $\phi \in C_0^{\infty}\left(\mathbb{R}^n\right)$ and for $x \in \mathbb{R}^n$ and $1 \leq j \leq n$ let $\hat{x}_j=\left(x_1, \ldots, x_{j-1}, x_{j+1}, \ldots, x_n\right)$. Let
  \[
  u_j\left(\hat{x}_j\right)=\left(\sum_{i=1}^n \int_{-\infty}^{\infty}\left|D_j \phi(x)\right| \d x_j\right)^{1 /(n-1)}
  \]
  which is evidently independent of $x_j$ and satisfies
  \[
  \left(\left\|u_j\right\|_{n-1, \mathbb{R}^{n-1}}\right)^{n-1} \leq|u|_{1,1, \mathbb{R}^n}
  \]
  Since
  \[
  \phi(x)=\int_{-\infty}^{x_1} D_1 \phi\left(t, \hat{x}_1\right) d t
  \]
  we have
  \[
  |\phi(x)| \leq \int_{-\infty}^{\infty}\left|D_1 \phi\left(t, \hat{x}_1\right)\right| d t \leq\left(u_1\left(\hat{x}_1\right)\right)^{n-1}
  \]
  Similarly, $|\phi(x)| \leq\left(u_j\left(\hat{x}_j\right)\right)^{n-1}$. Applying the inequality (14) from Lemma 4.23 with $k=n-1=\lambda$ we obtain
  \[
  \begin{aligned}
  \int_{\mathbb{R}^n}|\phi(x)|^{n /(n-1)} \d x & \leq \int_{\mathbb{R}^n} \prod_{j=1}^n u_j\left(\hat{x}_j\right) \d x \\
  & \leq\left(\prod_{j=1}^n \int_{\mathbb{R}^{n-1}}\left|u_j\left(\hat{x}_j\right)\right|^{n-1} d \hat{x}_j\right)^{1 /(n-1)} \leq|u|_{1,1, \mathbb{R}^n}^{n /(n-1)},
  \end{aligned}
  \]
  which completes the proof of $(20)$ and hence the theorem.
\end{proof}


\begin{remark}
  For the case $m=1,1<p<n$, Talenti [T] and Aubin, as exposed in Section 2.6 of [Au],
  obtained the best constant for the equivalent form of Sobolev's inequality
  \[
  \|\phi\|_{n p /(n-p), \mathbb{R}^n} \leq K\|\grad \phi\|_{p, \mathbb{R}^n}
  \]
  by showing that the ratio
  \[
  \frac{\|\phi\|_{n p /(n-p)}}{\|\grad \phi\|_{1, p}}
  \]
  is maximized if $u$ is a radially symmetric function of the form
  \[
  u(x)=\left(a+b|x|^{p /(p-1)}\right)^{1-(n / p)}
  \]
  which, while not in $C_0^{\infty}\left(\mathbb{R}^n\right)$ is a limit of functions in that space.
  His method involved first showing that replacing an arbitrary function $u$ vanishing at infinity
  with a radially symmetric, non-increasing, equimeasurable rearrangement of $u$ decreased $\|\grad u\|_{p, \mathbb{R}^n}$ while, of course, leaving $\|u\|_{n p /(n-p), \mathbb{R}^n}$ unchanged.
  Talenti's best constant for (21) is
  \[
  K=\pi^{-1 / 2} n^{-1 / p}\left(\frac{p-1}{n-p}\right)^{1 / p^{\prime}}\left(\frac{\Gamma(1+n / 2) \Gamma(n)}{\Gamma(n / p) \Gamma(1+n-(n / p))}\right)^{1 / n}.
  \]
\end{remark}


\section{Variations of Sobolev's Inequality}


\begin{para}
  Mixed-norm $L^p$ estimates of the type considered in Paragraphs 2.48-2.51 and used in the proof of
  Gagliardo's averaging lemma 4.23 can contribute to generalizations of Sobolev's inequality.
  We examine briefly two such generalizations:
  
  \begin{enumerate}[(a)]
    \item anisotropic Sobolev inequalities, in which different $L^p$ norms are used for different 
      partial derivatives on the right side of (19), and
    \item reduced Sobolev inequalities, in which the seminorm $|\phi|_{m, p, \mathbb{R}^n}$ on the 
      right side of (19) is replaced with a similar seminorm involving only a subset of the partial 
      derivatives of order $m$ of $\phi$.
  \end{enumerate}
  
  Questions of this sort are discussed in [BIN1] and [BIN2]. We follow the treatment in [A3] and [A4]
  and most of the details will be omitted here.
\end{para}


\begin{para}[A First-Order Anisotropic Sobolev Inequality]
  If $p_j \geq 1$ for each $j$ with $1 \leq j \leq n$ and $\phi \in C_0^{\infty}\left(\mathbb{R}^n\right)$,
  then an inequality of the form
  \begin{equation}
    \|\phi\|_q \leq K \sum_{j=1}^n\left\|D_j \phi\right\|_{p_j}
  \end{equation}
  is a (first-order) anisotropic Sobolev inequality because different $L^p$ norms are used to estimate the 
  derivatives of $\phi$ in different coordinate directions.
  A dilation argument involving $\phi\left(\lambda_1 x_1, \ldots, \lambda_n x_n\right)$
  for $0<\lambda_j<\infty, 1 \leq j \leq n$ shows that no such anisotropic inequality
  is possible for finite $q$ unless
  \[\sum_{j=1}^n \frac{1}{p_j}>1 \quad \text { and }
    \quad \frac{1}{q}=\frac{1}{n} \sum_{j=1}^n \frac{1}{p_j}-\frac{1}{n}.\]
  If these conditions are satisfied, then (22) does hold. The proof is a generalization of that of
  Theorem 4.31 and uses the mixed-norm Hölder and permutation inequalities. (See [A3] for the details.)
\end{para}


\begin{para}[Higher-Order Anisotropic Sobolev Inequalities]
  The generalization of (22) to an $m$ th order inequality by induction on $m$ is somewhat more 
  problematic.
  The $m$ th order isotropic inequality (19) follows from its special case $m=1$ by simple induction.
  We can also obtain
  \[
  \|\phi\|_q \leq K \sum_{|\alpha|=m}\left\|D^\alpha \phi\right\|_{p_\alpha}
  \]
  where
  \[
  \frac{1}{q}=\frac{1}{n^m} \sum_{|\alpha|=m}\left(\begin{array}{c}
  m \\
  \alpha
  \end{array}\right) \frac{1}{p_\alpha}-\frac{m}{n}, \quad\left(\begin{array}{c}
  m \\
  \alpha
  \end{array}\right)=\frac{m !}{\alpha_{1} ! \alpha_{2} ! \cdots \alpha_{n} !}
  \]
  by induction from (22) under suitable restrictions on the exponents $p_\alpha$, but the restriction
  \[
  \frac{1}{n^m} \sum_{|\alpha|=m}\left(\begin{array}{l}
  m \\
  \alpha
  \end{array}\right) \frac{1}{p_\alpha}>\frac{m}{n}
  \]
  will not suffice in general for the induction even though $\sum_{|\alpha|=m} \binom{m}{\alpha} = n^m$.
  The conditions $m p_\alpha<n$ for each $\alpha$ with $|\alpha|=m$ will suffice, but are stronger than 
  necessary. For any multi-index $\beta$ and $1 \leq j \leq n$, let
  \[
  \beta[j]=\left(\beta_1, \ldots, \beta_{j-1}, \beta_j+1, \beta_{j+1}, \ldots, \beta_n\right)
  \]
  Evidently, $|\beta[j]|=|\beta|+1$ and it can be verified that if the numbers $p_\alpha$ are defined for all $\alpha$ with $|\alpha|=m$, then
  \[
  \sum_{|\beta|=m-1}\left(\begin{array}{c}
  m-1 \\
  \beta
  \end{array}\right) \sum_{j=1}^n \frac{1}{p_{\beta[j]}}=\sum_{|\alpha|=m}\left(\begin{array}{c}
  m \\
  \alpha
  \end{array}\right) \frac{1}{p_\alpha}
  \]
  This provides the induction step necessary to verify the following theorem,
  for which the details can again be found in [A3].
\end{para}


\begin{theorem}
  Let $p_\alpha \geq 1$ for all $\alpha$ with $|\alpha|=m$.
  Suppose that for every $\beta$ with $|\beta|=m-1$ we have
  \[
  \sum_{j=1}^n \frac{1}{p_{\beta[j]}}>m .
  \]
  Then there exists a constant $K$ such that the inequality
  \[
  \|\phi\|_q \leq K \sum_{|\alpha|=m}\left\|D^\alpha \phi\right\|_{p_\alpha}
  \]
  holds for all $\phi \in C_0^{\infty}\left(\mathbb{R}^n\right)$, where
  \[
  \frac{1}{q}=\frac{1}{n^m} \sum_{|\alpha|=m}\left(\begin{array}{c}
  m \\
  \alpha
  \end{array}\right) \frac{1}{p_\alpha}-\frac{m}{n}
  \]
\end{theorem}


\begin{para}[Reduced Sobolev Inequalities]
  Another variation of Sobolev's inequality addresses the question of whether the number of derivatives 
  estimated in the seminorm on the right side of (19) (or, equivalently, (18)) can be reduced without 
  jeopardizing the validity of the inequality for all $\phi \in C_0^{\infty}\left(\mathbb{R}^n\right)$.
  If $m \geq 2$, the answer is yes; only those partial derivatives of order $m$ that are ``completely mixed'' 
  (in the sense that all $m$ differentiations are taken with respect to different variables) need be included
  in the seminorm. Specifically, if we denote
  \[
  \mathcal{M}=\mathcal{M}(n, m)=\left\{\alpha:|\alpha|=m, \quad \alpha_j=0 \text { or } \alpha_j=1 \text { for } 1 \leq j \leq n\right.
  \]
  then the reduced Sobolev inequality
  \[
  \|\phi\|_q \leq K \sum_{\alpha \in \mathcal{M}}\left\|D^\alpha \phi\right\|_p
  \]
  holds for all $\phi \in C_0^{\infty}\left(\mathbb{R}^n\right)$, provided $m p<n$ and $q=n p /(n-m p)$.
  Again the proof depends on mixed-norm estimates; it can be found in [A4] where the possibility of
  further reductions in the number of derivatives estimated on the right side of Sobolev's
  inequality is also considered. See also Section 13 in [BIN1].
\end{para}


\section[$W^{m,p}(\Omega)$ as a Banach Algebra]%
  {$\bm{W^{m,p}(\Omega)}$ as a Banach Algebra}


\begin{para}
  Given $u$ and $v$ in $W^{m,p}(\Omega)$, where $\Omega$ is a domain in $\mathbb{R}^n$,
  one cannot in general expect that their pointwise product $u v$ will belong to $W^{m,p}(\Omega)$.
  The imbedding theorem, however, shows that this is the case provided $m p>n$ and $\Omega$
  satisfies the cone condition. (See [Sr] and [Mz2].)
\end{para}


\begin{theorem}
  Let $\Omega$ be a domain in $\mathbb{R}^n$ satisfying the cone condition.
  If $m p>n$ or $p=1$ and $m \geq n$, then there exists a constant $K^*$ depending on $m, p, n$,
  and the cone $C$ determining the cone condition for $\Omega$,
  such that for $u, v \in W^{m,p}(\Omega)$ the product $u v$, defined pointwise a.e. in $\Omega$, satisfies
  \[
  \|u v\|_{m, p, \Omega} \leq K^*\|u\|_{m, p, \Omega}\|v\|_{m, p, \Omega}
  \]
  In particular, equipped with the equivalent norm $\|\cdot\|_{m, p, \Omega}^*$ defined by
  \[
  \|u\|_{m, p, \Omega}^*=K^*\|u\|_{m, p, \Omega}
  \]
  $W^{m,p}(\Omega)$ is a commutative Banach algebra with respect to pointwise multiplication in that
  \[
  \|u v\|_{m, p, \Omega}^* \leq\|u\|_{m, p, \Omega}^*\|v\|_{m, p, \Omega}^*
  \]
\end{theorem}

\begin{proof}
  \[
  \int_{\Omega}\left|D^\alpha[u(x) v(x)]\right|^p \leq K_\alpha\|u\|_{m, p, \Omega}\|v\|_{m, p, \Omega}
  \]
  where $K_\alpha=K_\alpha(m, p, n, C)$. Let us assume for the moment that $u \in C^{\infty}(\Omega)$. By the Leibniz rule for distributional derivatives, that is,
  \[
  D^\alpha(u v)=\sum_{\beta \leq \alpha}\left(\begin{array}{l}
  \alpha \\
  \beta
  \end{array}\right) D^\beta u D^{\alpha-\beta} v
  \]
  it is sufficient to show that for any $\beta \leq \alpha,|\alpha| \leq m$, we have
  \[
  \int_{\Omega}\left|D^\beta u(x) D^{\alpha-\beta} v(x)\right|^p \d x \leq K_{\alpha, \beta}\|u\|_{m, p, \Omega}^p\|v\|_{m, p, \Omega}^p
  \]
  where $K_{\alpha, \beta}=K_{\alpha, \beta}(m, p, n, C)$. By the imbedding theorem there exists, for any $\beta$ with $|\beta| \leq m$, a constant $K(\beta)=K(\beta, m, p, n, C)$ such that for any $w \in W^{m,p}(\Omega)$
  provided $(m-|\beta|) p \leq n$ and $p \leq r \leq n p /(n-[m-|\beta|] p)[$ or $p \leq r<\infty$ if $(m-|\beta|) p=n]$, or alternatively
  \[
  \left|D^\beta w(x)\right| \leq K(\beta)\|w\|_{m, p . \Omega} \quad \text { a.e. in } \Omega
  \]
  provided $(m-|\beta|) p>n$
  Let $k$ be the largest integer such that $(m-k) p>n$. Since $m p>n$ we have $k \geq 0$. If $|\beta| \leq k$, then $(m-|\beta|) p>n$, so
  \[
  \begin{aligned}
  \int_{\Omega}\left|D^\beta u(x) D^{\alpha-\beta} v(x)\right|^p \d x & \leq K(\beta)^p\|u\|_{m, p . \Omega}^p\left\|D^{\alpha-\beta} v\right\|_{0, p . \Omega}^p \\
  & \leq K(\beta)^p\|u\|_{m, p . \Omega}^p\|v\|_{m . p . \Omega}^p .
  \end{aligned}
  \]
  Similarly, if $|\alpha-\beta| \leq k$, then
  \[
  \int_{\Omega}\left|D^\beta u(x) D^{\alpha-\beta} v(x)\right|^p \d x \leq K(\alpha-\beta)^p\|u\|_{m, p, \Omega}^p\|v\|_{m, p, \Omega}^p .
  \]
  Now if $|\beta|>k$ and $|\alpha-\beta|>k$, then, in fact, $|\beta| \geq k+1$ and $|\alpha-\beta| \geq k+1$ so that $n \geq(m-|\beta|) p$ and $n \geq(m-|\alpha-\beta|) p$. Moreover,
  \[
  \frac{n-(m-|\beta|) p}{n}+\frac{n-(m-|\alpha-\beta|) p}{n}=2-\frac{(2 m-|\alpha|) p}{n}<2-\frac{m p}{n}<1 .
  \]
  Hence there exist positive numbers $r$ and $r^{\prime}$ with $(1 / r)+\left(1 / r^{\prime}\right)=1$ such that
  \[
  p \leq r p<\frac{n p}{n-(m-|\beta|) p}, \quad p \leq r^{\prime} p<\frac{n p}{n-(m-|\alpha-\beta|) p} .
  \]
  Thus by Hölder's inequality and (24) we have
  \[
  \begin{aligned}
  \int_{\Omega}\left|D^\beta u(x) D^{\alpha-\beta} v(x)\right|^p \d x & \leq\left(\int_{\Omega}\left|D^\beta u(x)\right|^{r p} \d x\right)^{1 / r}\left(\int_{\Omega}\left|D^{\alpha-\beta} v(x)\right|^{r^{\prime} p} \d x\right)^{1 / r^{\prime}} \\
  & \leq(K(\beta))^{1 / r}(K(\alpha-\beta))^{1 / r^{\prime}}\|u\|_{m, p, \Omega}^p\|v\|_{m, p, \Omega}^p .
  \end{aligned}
  \]
  This completes the proof of (23) for $u \in C^{\infty}(\Omega), v \in W^{m,p}(\Omega)$.
  If $u \in W^{m,p}(\Omega)$ then by Theorem 3.17 there exists a sequence $\left\{u_j\right\}$ of $C^{\infty}(\Omega)$ functions converging to $u$ in $W^{m,p}(\Omega)$. By the above argument, $\left\{u_j v\right\}$ is a Cauchy sequence in $W^{m,p}(\Omega)$ and so it converges to an element $w$ of that space. Since $m p>n, u$ and $v$ may be assumed to be continuous and bounded on $\Omega$. Thus
  \[
  \begin{aligned}
  \|w-u v\|_{0, p, \Omega} & \leq\left\|w-u_j v\right\|_{0, p, \Omega}+\left\|\left(u_j-u\right) v\right\|_{0, p, \Omega} \\
  & \leq\left\|w-u_j v\right\|_{0, p, \Omega}+\|v\|_{0, \infty, \Omega}\left\|u_j-u\right\|_{0, p, \Omega} \\
  & \rightarrow 0 \quad \text { as } j \rightarrow \infty .
  \end{aligned}
  \]
  Hence $w=u v$ in $L^p(\Omega)$ and so $w=u v$ in the sense of distributions. Therefore, $w=u v$ in $W^{m,p}(\Omega)$ and
  \[
  \|u v\|_{m, p, \Omega}=\|w\|_{m, p, \Omega} \leq \limsup _{j \rightarrow \infty}\left\|u_j v\right\|_{m, p, \Omega} \leq K^*\|u\|_{m, p, \Omega}\|v\|_{m, p, \Omega}
  \]
  as was to be shown.
\end{proof}

We remark that the Banach algebra $W^{m,p}(\Omega)$ has an identity element if an only if $\Omega$ is bounded. 
That is, the function $e(x)=1$ belongs to $W^{m,p}(\Omega)$ if and only if $\Omega$ has finite volume,
but there are no unbounded domains of finite volume that satisfy the cone condition.


\section{Optimality of the Imbedding Theorem}


\begin{para}
  The imbeddings furnished by the Sobolev Imbedding Theorem 4.12 are ``best possible'' in the sense that
  no imbeddings of the types asserted there are possible for any domain for parameter values $m, p, q, \lambda$ 
  etc.\ not satisfying the restrictions imposed in the statement of the theorem.
  We present below a number of examples to illustrate this fact.
  In these examples it is the local behaviour of functions in $W^{m,p}(\Omega)$ rather than their behaviour 
  near the boundary that prevents extending the parameter intervals for imbeddings.
  
  There remains the possibility that a weaker version of Part I of the imbedding theorem may hold for certain 
  domains not nice enough to satisfy the (weak) cone condition.
  We will examine some such possibilities later in this chapter.
\end{para}


\begin{example}
  Let $k$ be an integer such that $1 \leq k \leq n$ and suppose that $m p<n$ and $q>p^*=k p /(n-m p)$. We construct a function $u \in W^{m,p}(\Omega)$ such that $u \notin L^q(\Omega_k)$, where $\Omega_k$ is the intersection of $\Omega$ with a $k$-dimensional plane, thus showing that $W^{m,p}(\Omega)$ does not imbed into $L^q(\Omega_k)$.
  
  Without loss of generality, we can assume that the origin belongs to $\Omega$ and that $\Omega_k=\left\{x \in \Omega: x_{k+1}=\cdots=x_n=0\right\}$. For $R>0$, let $B_R=\left\{x \in \mathbb{R}^n:|x|<R\right\}$. We fix $R$ small enough that $\overline{B_{2 R}} \subset \Omega$. Let $v(x)=|x|^\mu$; the value of $\mu$ will be set later. Evidently $v \in C^{\infty}\left(\mathbb{R}^n-\{0\}\right)$. Let $u \in C^{\infty}\left(\mathbb{R}^n-\{0\}\right)$ be a function satisfying $u(x)=v(x)$ in $B_R$ and $u(x)=0$ outside $B_{2 R}$. The membership of $u$ in $W^{m,p}(\Omega)$ depends only on the behaviour of $v$ near the origin:
  \[
  u \in W^{m,p}(\Omega) \quad \Longleftrightarrow \quad v \in W^{m,p}\left(B_R\right) .
  \]
  It is easily checked by induction on $|\alpha|$ that
  \[
  D^\alpha v(x)=P_\alpha(x)|x|^{\mu-2|\alpha|},
  \]
  where $P_\alpha(x)$ is a polynomial homogeneous of degree $|\alpha|$ in the components of $x$. Thus $\left|D^\alpha v(x)\right| \leq K_\alpha|x|^{\mu-|\alpha|}$ and, setting $\rho=|x|$,
  \[
  \int_{B_R}\left|D^\alpha v(x)\right|^p \d x \leq K_n K_\alpha^P \int_0^R \rho^{(\mu-|\alpha|) p+n-1} d \rho,
  \]
  where $K_n$ is the $(n-1)$-measure of the sphere of radius 1 in $\mathbb{R}^n$. Therefore $v \in W^{m,p}\left(B_R\right)$ and $u \in W^{m,p}(\Omega)$ provided $\mu>m-(n / p)$.
  On the other hand, denoting $\tilde{x}_k=\left(x_1, \ldots, x_k\right)$ and $r=\left|\tilde{x}_k\right|$, we have
  \[
  \int_{\Omega_k}|u(x)|^q d \tilde{x}_k \geq \int_{\left(B_R\right)_k}|v(x)|^q d \tilde{x}_k=K_k \int_0^R r^{\mu q+k-1} d r
  \]
  Thus $u \notin L^q(\Omega_k)$ if $\mu \leq-(k / q)$.
  Since $q>k p /(n-m p)$ we can pick $\mu$ so that $m-(n / p)<\mu \leq-(k / q)$, thus completing the specification of $u$.
\end{example}

Note that $\mu<0$, so $u$ is unbounded near the origin. Hence no imbedding of the form $W^{m,p}(\Omega) \rightarrow C_B^0(\Omega)$ is possible if $m p<n$.


\begin{example}
  Suppose $m p>n>(m-1) p$, and let $\lambda>m-(n / p)$. Fix $\mu$ so that $m-(n / p)<\mu<\lambda$. Then the function $u$ constructed in Example 4.41 continues to belong to $W^{m,p}(\Omega)$. However, if $|x|<R$,
  \[
  \frac{|u(x)-u(0)|}{|x-0|^\lambda}=|x|^{\mu-\lambda} \rightarrow \infty \quad \text { as }|x| \rightarrow 0
  \]
  Thus $u \notin C^{0, \lambda}(\overline{\Omega})$, and the imbedding $W^{m,p}(\Omega) \rightarrow C^{0, \lambda}(\overline{\Omega})$ is not possible.
\end{example}

\begin{example}
  Suppose $p>1$ and $m p=n$. We construct a function $u$ in $W^{m,p}(\Omega)$ such that
  $u \notin L^{\infty}(\Omega)$. Hence the imbedding $W^{m,p}(\Omega) \rightarrow L^q(\Omega)$,
  valid for $p \leq q<\infty$, cannot be extended to yield $W^{m,p}(\Omega) \rightarrow L^{\infty}(\Omega)$
  or $W^{m,p}(\Omega) \rightarrow C^0(\overline{\Omega})$ unless $p=1$ and $m=n$. (See, however, Theorem 8.27.)
  Again we assume $0 \in \Omega$ and define $u(x)$ as in Example 4.41 except with a different function $v(x)$ 
  defined by
  \[
  v(x)=\log (\log (4 R /|x|))
  \]
  Clearly $v$ is not bounded near the origin, so $u \notin L^{\infty}(\Omega)$. It can be checked by induction on $|\alpha|$ that
  \[
  D^\alpha v(x)=\sum_{j=1}^{|\alpha|} P_{\alpha, j}(x)|x|^{-2|\alpha|}(\log (4 R /|x|))^{-j}
  \]
  where $P_{\alpha, j}(x)$ is a polynomial homogeneous of degree $|\alpha|$ in the components of $x$.
  Since $p=n / m$, we have
  \[
  \left|D^\alpha v(x)\right|^p \leq \sum_{j=1}^{|\alpha|} K_{\alpha, j}|x|^{-|\alpha| n / m}(\log (4 R /|x|))^{-j p}
  \]
  so that, setting $\rho=|x|$,
  \[
  \int_{B_R}\left|D^\alpha v(x)\right|^p \d x \leq K \sum_{j=1}^{|\alpha|} \int_0^R(\log (4 R / \rho))^{-j p} \rho^{-|\alpha| n / m+n-1} d \rho
  \]
  The right side of the above inequality is certainly finite if $|\alpha|<m$. If $|\alpha|=m$, we have, setting $\sigma=\log (4 R / \rho)$,
  \[
  \int_{B_R}\left|D^\alpha v(x)\right|^p \d x \leq K \sum_{j=1}^{|\alpha|} \int_{\log 4}^{\infty} \sigma^{-j p} d \sigma
  \]
  which is finite since $p>1$. Thus $v \in W^{m,p}\left(B_R\right)$ and $u \in W^{m,p}(\Omega)$.
  It is interesting that the same function $v$ (and hence $u$ ) works for any choice of $m$ and $p$ with $m p=n$.
\end{example}

\begin{example}
  Suppose $(m-1) p=n$ and $p>1$. We construct $u$ in $W^{m,p}(\Omega)$ such
  that $u \notin C^{0,1}(\overline{\Omega})$.
  Hence the imbedding $W^{m,p}(\Omega) \rightarrow C^{0, \lambda}(\overline{\Omega})$,
  valid for $0<\lambda<1$ whenever $\Omega$ satisfies the strong local Lipschitz condition,
  cannot be extended to yield $W^{m,p}(\Omega) \rightarrow C^{0,1}(\overline{\Omega})$ unless $p=1$ and $m-1=n$. 
  Here $u$ is constructed as in the previous example except using
  \[
  v(x)=|x| \log (\log (4 R /|x|)) .
  \]
  Since $|v(x)-v(0)| /|x-0|=\log (\log (4 R /|x|)) \rightarrow \infty$ as $x \rightarrow 0$
  it is clear that $v \notin C^{0,1}\left(\overline{B_R}\right)$ and
  therefore $u \notin C^{0,1}(\overline{\Omega})$.
  The fact that $v \in W^{m,p}\left(B_R\right)$ and hence $u \in W^{m,p}(\Omega)$
  is shown just as in the previous example.
\end{example}


\section{Nonimbedding Theorems for Irregular Domains}

\begin{para}
  The above examples show that even for very regular domains there can exist no imbeddings of the types 
  considered in Theorem 4.12 except those explicitly stated there. It remains to be seen whether any imbeddings 
  of those types can exist for domains that do not satisfy the cone condition (or at least the weak cone 
  condition). We will show below that Theorem 4.12 can be extended, with weakened conclusions, to certain types 
  of irregular domains, but first we show that no extension is possible if the domain is ``too irregular''.
  This can happen if either the domain is unbounded and too narrow at infinity, or if it has a cusp of 
  exponential sharpness on its boundary.
  An unbounded domain $\Omega \subset \mathbb{R}^n$ may have a smooth boundary and still fail to satisfy the 
  cone condition if it becomes narrow at infinity, that is, if
  \[
  \lim _{\substack{|x| \rightarrow \infty \\ x \in \Omega}} \dist(x, \partial \Omega)=0 .
  \]
  The following theorem shows that Parts I and II of Theorem~4.12 fail completely for any
  unbounded $\Omega$ which has finite volume.
\end{para}


\begin{theorem}
  Let $\Omega$ be an unbounded domain in $\mathbb{R}^n$ having finite volume, and let $q>p$.
  Then $W^{m,p}(\Omega)$ is not imbedded in $L^q(\Omega)$.
\end{theorem}

\begin{proof}
  We construct a function $u(x)$ depending only on distance $\rho=|x|$ of $x$ from the origin whose growth
  as $\rho$ increases is rapid enough to prevent membership in $L^q(\Omega)$ but not so rapid as to prevent 
  membership in $W^{m,p}(\Omega)$.
  Without loss of generality we assume $\vol(\Omega)=1$. Let $A(\rho)$ denote the surface area
  (($n-1)$-measure) of the intersection of $\Omega$ with the surface $|x|=\rho$. Then
  \[
  \int_0^{\infty} A(\rho) d \rho=1
  \]
  Let $r_0=0$ and define $r_k$ for $k=1,2, \ldots$ by
  \[
  \int_{r_k}^{\infty} A(\rho) d \rho=\frac{1}{2^k}=\int_{r_{k-1}}^{r_k} A(\rho) d \rho
  \]
  Since $\Omega$ is unbounded, $r_k$ increases to infinity with $k$. Let $\Delta r_k=r_{k+1}-r_k$
  and fix $\varepsilon$ such that $0<\varepsilon<[1 /(m p)]-[1 /(m q)]$.
  There must exist an increasing sequence $\left\{k_j\right\}_{j=1}^{\infty}$ such that
  $\Delta r_{k_j} \geq 2^{-\varepsilon k_j}$, for otherwise $\Delta r_k<2^{-\varepsilon k}$
  for all but possibly finitely many values of $k$ and we would have $\sum_{k=0}^{\infty} \Delta r_k<\infty$, 
  contradicting $\lim r_k=\infty$. For convenience we assume $k_1 \geq 1$ so $k_j \geq j$ for all $j$.
  Let $a_0=0$, $a_j=r_{k_j+1}$, and $b_j=r_{k_j}$. Note that $a_{j-1} \leq b_j<a_j$
  and $a_j-b_j=\Delta r_{k_j} \geq 2^{-\varepsilon k_j}$.
  Let $f$ be an infinitely differentiable function on $\mathbb{R}$ having the properties:
  \begin{enumerate}[(i)]
    \item $0 \leq f(t) \leq 1$ for all $t$,
    \item $f(t)=0$ if $t \leq 0$ and $f(t)=1$ if $t \geq 1$,
    \item $\left|(d / d t)^\kappa f(t)\right| \leq M$ for all $t$ if $1 \leq \kappa \leq m$.
  \end{enumerate}
  If $x \in \Omega$ and $\rho=|x|$, set
  \[
  u(x)= \begin{cases}2^{k_{j-1} / q} & \text { for } a_{j-1} \leq \rho \leq b_j \\ 2^{k_{j-1} / q}+\left(2^{k_j / q}-2^{k_{j-1} / q}\right) f\left(\frac{\rho-b_j}{a_j-b_j}\right) & \text { for } b_j \leq \rho \leq a_j\end{cases}
  \]
  Clearly $u \in C^{\infty}(\Omega)$. Denoting $\Omega_j=\left\{x \in \Omega: a_{j-1} \leq \rho \leq a_j\right\}$, we have
  \[
  \begin{aligned}
  \int_{\Omega_j}|u(x)|^p \d x & =\left(\int_{a_{j-1}}^{b_j}+\int_{b_j}^{a_j}\right)(u(x))^p A(\rho) d \rho \\
  & \leq 2^{k_{j-1} p / q} \int_{a_{j-1}}^{\infty} A(\rho) d \rho+2^{k_j p / q} \int_{b_j}^{a_j} A(\rho) d \rho \\
  & =\frac{2^{-k_{j-1}(1-p / q)}+2^{-k_j(1-p / q)}}{2} \leq \frac{1}{2^{(j-1)(1-p / q)}} .
  \end{aligned}
  \]
  Since $p<q$, the above inequality forces
  \[
  \int_{\Omega}|u(x)|^p \d x=\sum_{j=1}^{\infty} \int_{\Omega_j}|u(x)|^p \d x<\infty .
  \]
  Also, if $1 \leq \kappa \leq m$, we have
  \[
  \begin{aligned}
  \int_{\Omega_j}\left|\frac{d^\kappa u}{d \rho^\kappa}\right|^p \d x & =\int_{b_j}^{a_j}\left|\frac{d^\kappa u}{d \rho^\kappa}\right|^p A(\rho) d \rho \\
  & \leq M^p 2^{k_j p / q}\left(a_j-b_j\right)^{-\kappa p} \int_{b_j}^{a_j} A(\rho) d \rho \\
  & =\frac{M^p 2^{-k_j(1-p / q-\varepsilon \kappa p)}}{2} \leq \frac{M^p 2^{-C j}}{2},
  \end{aligned}
  \]
  where $C=1-p / q-\varepsilon \kappa p>0$ because of the choice of $\varepsilon$.
  Hence $D^\alpha u \in L^p(\Omega)$ for $|\alpha| \leq m$, that is, $u \in W^{m,p}(\Omega)$.
  However, $u \notin L^q(\Omega)$ because we have for each $j$,
  \[
  \begin{aligned}
  \int_{\Omega_j}|u(x)|^q \d x & \geq 2^{k_{j-1}} \int_{a_{j-1}}^{a_j} A(\rho) d \rho \\
  & =2^{k_{j-1}}\left(2^{-k_{j-1}-1}-2^{-k_j-1}\right) \geq \frac{1}{4} .
  \end{aligned}
  \]
  Therefore $W^{m,p}(\Omega)$ cannot be imbedded in $L^q(\Omega)$.
\end{proof}

The conclusion of the above theorem can be extended to unbounded domains having infinite volume but satisfying
\[\limsup_{N \rightarrow \infty} \vol(\{x \in \Omega: N \leq|x| \leq N+1\})=0.\]
(See Theorem 6.41.)

\begin{para}
  Parts I and II of Theorem 4.12 also fail completely for domains with sufficiently sharp boundary cusps. 
  If $\Omega$ is a domain in $\mathbb{R}^n$ and $x_0$ is a point on its boundary,
  let $B_r=B_r\left(x_0\right)$ denote the open ball of radius $r$ and centre at $x_0$.
  Let $\Omega_r=B_r \cap \Omega$, let $S_r = \partial B_r \cap \Omega$,
  and let $A(r, \Omega)$ be the surface area ($(n-1)$-measure) of $S_r$.
  We shall say that $\Omega$ has a cusp of exponential sharpness
  at its boundary point $x_0$ if for every real number $k$ we have
  \begin{equation}\label{eq:4.25}
    \lim_{r\to 0+} \frac{A(r, \Omega)}{r^k} = 0.
  \end{equation}
\end{para}

\begin{theorem}
  If $\Omega$ is a domain in $\mathbb{R}^n$ having a cusp of exponential sharpness at a point $x_0$
  on its boundary, then $W^{m,p}(\Omega)$ is not imbedded in $L^q(\Omega)$ for any $q>p$.
\end{theorem}

\begin{proof}
  We construct $u \in W^{m,p}(\Omega)$ which fails to belong to $L^q(\Omega)$ because
  it becomes unbounded too rapidly near $x_0$. Without loss of generality we may assume $x_0=0$, 
  so that $r=|x|$. Let $\Omega^*=\left\{x /|x|^2: x \in \Omega,|x|<1\right\}$.
  Then $\Omega^*$ is unbounded and has finite volume by $(25)$, and
  \[
  A\left(r, \Omega^*\right)=r^{2(n-1)} A(1 / r, \Omega)
  \]
  Let $t$ satisfy $p<t<q$. By Theorem 4.46 there exists a
  function $\tilde{v} \in C^m(0, \infty)$ such that
  (i) $\tilde{v}(r)=0$ if $0<r \leq 1$
  (ii) $\int_1^{\infty}\left|\tilde{v}^{(j)}\right|^t A\left(r, \Omega^*\right) d r<\infty$ if $0 \leq j \leq m$
  (iii) $\int_1^{\infty}|\tilde{v}(r)|^q A\left(r, \Omega^*\right) d r=\infty$.
  [Specifically, $v(y)=\tilde{v}(|y|)$ defines $v \in W^{m, t}\left(\Omega^*\right)$ but $v \notin L^q\left(\Omega^*\right)$.] Let $x=y /|y|^2$ so that $\rho=|x|=1 /|y|=1 / r$. Set $\lambda=2 n / q$ and define
  \[
  u(x)=\tilde{u}(\rho)=r^\lambda \tilde{v}(r)=|y|^\lambda v(y)
  \]
  It follows for $|\alpha|=j \leq m$ that
  \[
  \left|D^\alpha u(x)\right| \leq\left|\tilde{u}^{(j)}(\rho)\right| \leq \sum_{i=1}^j c_{i j} r^{\lambda+j+i} \tilde{v}^{(i)}(r)
  \]
  where the coefficients $c_{i j}$ depend only on $\lambda$. Now $u(x)$ vanishes for $|x| \geq 1$ and so
  \[
  \int_{\Omega}|u(x)|^q \d x=\int_0^1|\tilde{u}(\rho)|^q A(\rho, \Omega) d \rho=\int_1^{\infty}|\tilde{v}(r)|^q A\left(r, \Omega^*\right) d r=\infty
  \]
  On the other hand, if $0 \leq|\alpha|=j \leq m$, we have
  \[
  \begin{aligned}
  \int_{\Omega}\left|D^\alpha u(x)\right|^p \d x & \leq \int_0^1\left|\tilde{u}^{(j)}(\rho)\right|^p A(\rho, \Omega) d \rho \\
  & \leq K \sum_{i=0}^j \int_1^{\infty}\left|\tilde{v}^{(i)}(r)\right|^p r^{(\lambda+j+i) p-2 n} A\left(r, \Omega^*\right) d r .
  \end{aligned}
  \]
  If it happens that $(\lambda+2 m) p \leq 2 n$, then, since $p<t$ and $\vol\left(\Omega^*\right)<\infty$, all the integrals in the above sum are finite by Hölder's inequality, and $u \in W^{m,p}(\Omega)$. Otherwise let
  \[
  k=((\lambda+2 m) p-2 n) \frac{t}{t-p}+2 n \text {. }
  \]
  By (25) there exists $a \leq 1$ such that if $\rho \leq a$, then $A(\rho, \Omega) \leq \rho^k$. It follows that if $r \geq 1 / a$, then
  \[
  r^{k-2 n} A\left(r, \Omega^*\right) \leq r^{k-2} \rho^k=r^{-2} .
  \]
  Thus
  \[
  \begin{aligned}
  & \int_1^{\infty}\left|\tilde{v}^{(i)}(r)\right|^p r^{(\lambda+j+i) p-2 n} A\left(r, \Omega^*\right) d r \\
  &=\int_1^{\infty}\left|\tilde{v}^{(i)}(r)\right|^p r^{(k-2 n)(t-p) / t} A\left(r, \Omega^*\right) d r \\
  & \leq\left(\int_1^{\infty}\left|\tilde{v}^{(i)}(r)\right|^t A\left(r, \Omega^*\right) d r\right)^{p / t}\left(\int_1^{\infty} r^{k-2 n} A\left(r, \Omega^*\right) d r\right)^{(t-p) / t}
  \end{aligned}
  \]
  which is finite. Hence $u \in W^{m,p}(\Omega)$ and the proof is complete.
\end{proof}


\section{Imbedding Theorems for Domains with Cusps}


\begin{para}
  Having proved that Theorem 4.12 fails completely for sufficiently irregular domains, we now 
  propose to show that certain imbeddings of the types considered in that theorem do hold for 
  less irregular domains that nevertheless fail to satisfy even the weak cone condition. 
  Questions of this sort have been considered by many writers.
  The treatment here follows that in [A1].
  
  We consider domains $\Omega$ in $\mathbb{R}^n$ whose boundaries consist only
  of $(n-1)$-dimensional surfaces, and it is assumed that $\Omega$ lies on only one side of its 
  boundary. For such domains we shall say, somewhat loosely, that $\Omega$ has a cusp at point 
  $x_0$ on its boundary if no finite open cone of positive volume contained in $\Omega$ can have 
  its vertex at $x_0$. The failure of a domain to have any cusps does not, of course,
  imply that it satisfies the cone condition.
  
  We consider a family of special domains in $\mathbb{R}^n$ that we call standard cusps and that have cusps of power sharpness (less sharp than exponential sharpness).
\end{para}


\begin{para}[Standard Cusps]
  If $1 \leq k \leq n-1$ and $\lambda>1$, let the standard cusp $Q_{k, \lambda}$ be the set of points $x=\left(x_1, \ldots, x_n\right)$ in $\mathbb{R}^n$ that satisfy the inequalities
  \[
  \begin{aligned}
  & x_1^2+\cdots+x_k^2<x_{k+1}^{2 \lambda}, \quad x_{k+1}>0, \ldots, x_n>0 \\
  & \left(x_1^2+\cdots+x_k^2\right)^{1 / \lambda}+x_{k+1}^2+\cdots+x_n^2<a^2
  \end{aligned}
  \]
  where $a$ is the radius of the ball of unit volume in $\mathbb{R}^n$. Note that $a<1$.
  The cusp $Q_{k, \lambda}$ has axial plane spanned by the $x_k, \ldots, x_n$ axes, and verticial 
  plane (cusp plane) spanned by $x_{k+2}, \ldots, x_n$. If $k=n-1$, the origin is the only vertex 
  point of $Q_{k, \lambda}$. The outer boundary surface of $Q_{k, \lambda}$ corresponds to 
  equality in (26) in order to simplify calculations later. A sphere or other suitable surface 
  bounded and bounded away from the origin could be used instead.
  
  Corresponding to the standard cusp $Q_{k, \lambda}$ we consider the associated standard cone $\mathcal{C}_k=Q_{k, 1}$ consisting of points $y=\left(y_1, \ldots, y_n\right)$ in $\mathbb{R}^n$ that satisfy the inequalities
  \[
  \begin{aligned}
  & y_1^2+\cdots+y_k^2<y_{k+1}^2, \quad y_{k+1}>0, \ldots, y_n>0, \\
  & y_1^2+\cdots+y_n^2<a^2 .
  \end{aligned}
  \]
  Figure~3 illustrates the standard cusps $Q_{1.2}$ in $\mathbb{R}^2$, and $Q_{2.2}$
  and $Q_{1.2}$ in $\mathbb{R}^3$, together with their associated standard cones.
  In $\mathbb{R}^3$ the cusp $Q_{2.2}$ has a single cusp point (vertex) at the origin,
  while $Q_{1.2}$ has a cusp line along the $x_3$-axis.
  
  It is convenient to adopt a system of generalized ``cylindrical'' coordinates
  in $\mathbb{R}^n$, $\left(r_k, \phi_1, \ldots, \phi_{k-1}, y_{k+1}, \ldots, y_n\right)$,
  so that $r_k \geq 0,-\pi \leq \phi_1 \leq \pi, 0 \leq \phi_2, \ldots$
  $\phi_{k-1} \leq \pi$, and
  \[
  \begin{aligned}
  & y_1=r_k \sin \phi_1 \sin \phi_2 \cdots \sin \phi_{k-1} \\
  & y_2=r_k \cos \phi_1 \sin \phi_2 \cdots \sin \phi_{k-1} \\
  & y_3=\quad r_k \cos \phi_2 \cdots \sin \phi_{k-1} \\
  & \vdots \\
  & y_k=\quad r_k \cos \phi_{k-1} \text {. } \\
  &
  \end{aligned}
  \]
  In terms of these coordinates, $\mathcal{C}_k$ is represented by
  \[
  \begin{gathered}
  0 \leq r_k<y_{k+1}, \quad y_{k+1}>0, \ldots, y_n>0 \\
  r_k^2+y_{k+1}^2+\cdots+y_n^2<a^2
  \end{gathered}
  \]
  The standard cusp $Q_{k, \lambda}$ may be transformed into the associated cone $\mathcal{C}_k$ by means of the one-to-one transformation
  \[
  \begin{array}{rlr}
  x_1 & =r_k^\lambda \sin \phi_1 \sin \phi_2 \cdots \sin \phi_{k-1} \\
  x_2 & =r_k^\lambda \cos \phi_1 \sin \phi_2 \cdots \sin \phi_{k-1} \\
  x_3 & = & r_k^\lambda \cos \phi_2 \cdots \sin \phi_{k-1} \\
  \vdots & \\
  x_k & = & \\
  x_{k+1} & =y_{k+1} & \\
  \vdots & \\
  x_n & =y_n,
  \end{array}
  \]
  which has Jacobian determinant
  \[
  \left|\frac{\partial\left(x_1, \ldots, x_n\right)}{\partial\left(y_1, \ldots, y_n\right)}\right|=\lambda r_k^{(\lambda-1) k}
  \]
  We now state three theorems extending imbeddings of the types considered in Theorem~4.12 
  (except the trace imbeddings) to domains with boundary irregularities comparable to standard 
  cusps. The proofs of these theorems will be given later in this chapter.
\end{para}


\begin{theorem}
  Let $\Omega$ be a domain in $\mathbb{R}^n$ having the following property:
  There exists a family $\Gamma$ of open subsets of $\Omega$ such that
  \begin{enumerate}[(i)]
    \item $\Omega=\bigcup_{G \in \Gamma} G$
    \item $\Gamma$ has the finite intersection property, that is, there exists a positive
      integer $N$ such that any $N+1$ distinct sets in $\Gamma$ have empty intersection,
    \item at most one set $G \in \Gamma$ satisfies the cone condition,
    \item there exist positive constants $v$ and $A$ such that for each $G \in \Gamma$
      not satisfying the cone condition there exists a one-to-one
      function $\Psi=\left(\psi_1, \ldots, \psi_n\right)$ mapping $G$ onto a standard
      cusp $Q_{k, \lambda}$, where $(\lambda-1) k \leq \nu$, and such that for
      all $i, j,(1 \leq i, j \leq n)$, all $x \in G$, and all $y \in Q_{k, \lambda}$,
      \[
      \left|\frac{\partial \psi_j}{\partial x_i}\right| \leq A \quad \text { and } \quad\left|\frac{\partial\left(\psi^{-1}\right)_j}{\partial y_i}\right| \leq A
      \]
  \end{enumerate}
  If $v>m p-n$, then
  \[
  W^{m,p}(\Omega) \rightarrow L^q(\Omega), \quad \text { for } \quad p \leq q \leq \frac{(\nu+n) p}{v+n-m p} .
  \]
  If $v=m p-n$, then the same imbedding holds for $p \leq q<\infty$, and for $q=\infty$ if $p=1$.
  If $v<m p-n$, then the imbedding holds for $p \leq q \leq \infty$.
\end{theorem}


\begin{theorem}
  Let $\Omega$ be a domain in $\mathbb{R}^n$ having the following property:
  There exist positive constants $v<m p-n$ and $A$ such that for each $x \in \Omega$ there
  exists an open set $G$ with $x \in G \subset \Omega$ and a one-to-one
  mapping $\Psi=\left(\psi_1, \ldots, \psi_n\right)$ mapping $G$ onto a standard cusp $Q_{k, \lambda}$,
  where $(\lambda-1) k \leq \nu$, and such that for all $i, j,(1 \leq i, j \leq n)$, all $x \in G$,
  and all $y \in Q_{k, \lambda}$,
  \[
  \left|\frac{\partial \psi_j}{\partial x_i}\right| \leq A \quad \text { and } \quad\left|\frac{\partial\left(\psi^{-1}\right)_j}{\partial y_i}\right| \leq A .
  \]
  Then
  \[
  W^{m,p}(\Omega) \rightarrow C_B^0(\Omega)
  \]
  More generally, if $v<(m-j) p-n$ where $0 \leq j \leq m-1$, then
  \[
  W^{m,p}(\Omega) \rightarrow C_B^j(\Omega)
  \]
\end{theorem}


\begin{theorem}
  Let $\Omega$ be a domain in $\mathbb{R}^n$ having the following property:
  There exist positive constants $v, \delta$, and $A$ such that for each pair of points $x, y \in \Omega$
  with $|x-y| \leq \delta$ there exists an open set $G$ with $x, y \in G \subset \Omega$ and a one-toone mapping 
  $\Psi=\left(\psi_1, \ldots, \psi_n\right)$ mapping $G$ onto a standard cusp $Q_{k, \lambda}$,
  where $(\lambda-1) k \leq \nu$, and such that for all $i, j,(1 \leq i, j \leq n)$, all $x \in G$,
  and all $y \in Q_{k, \lambda}$,
  \[
  \left|\frac{\partial \psi_j}{\partial x_i}\right| \leq A \quad \text { and } \quad\left|\frac{\partial\left(\psi^{-1}\right)_j}{\partial y_i}\right| \leq A .
  \]
  Suppose that $(m-j-1) p<v+n<(m-j) p$ for some integer $j,(0 \leq j \leq m-1)$. Then
  \[
  W^{m,p}(\Omega) \rightarrow C^{j, \mu}(\overline{\Omega}) \quad \text { for } \quad 0<\mu \leq m-j-\frac{n+\nu}{p} .
  \]
  If $(m-j-1) p=v+n$, then the same imbedding holds for $0<\mu<1$. In either event we have $W^{m,p}(\Omega) \rightarrow C^j(\overline{\Omega})$.
\end{theorem}


\begin{remarks}
  \begin{enumerate}
    \item In these theorems the role played by the parameter $v$ is equivalent to an increase
      in the dimension $n$ in Theorem 4.12, where increasing $n$ results in weaker imbedding results for
      given $m$ and $p$. Since $v \geq(\lambda-1) k$, the sharper the cusp, the greater the equivalent increase 
      in dimension.
    \item The reader may wish to construct examples similar to those of Paragraphs 4.41-4.44 to show that the 
      three theorems above give the best possible imbeddings for the domains and types of spaces considered.
  \end{enumerate}
\end{remarks}


\begin{example}
  To illustrate Theorem 4.51, consider the domain
  \[
  \Omega=\left\{x=\left(x_1, x_2, x_3\right) \in \mathbb{R}^3: x_2>0, x_2^2<x_1<3 x_2^2\right\} .
  \]
  If $a=(4 \pi / 3)^{-1 / 3}$, the radius of the ball of unit volume in $\mathbb{R}^3$, it is readily verified
  that the transformation
  \[
  y_1=x_1+2 x_2^2, \quad y_2=x_2, \quad y_3=x_3-(k / a), \quad k=0, \pm 1, \pm 2, \ldots
  \]
  transforms a subdomain $G_k$ of $\Omega$ onto the standard cusp $Q_{1,2} \subset \mathbb{R}^3$ in the manner 
  required of the transformation $\Psi$ in the statement of the theorem. Moreover, $\{G\}_{k=-\infty}^{\infty}$ 
  has the finite intersection property and covers $\Omega$ up to a set satisfying the cone condition.
  Using $v=1$, we conclude that $W^{m,p}(\Omega) \rightarrow L^q(\Omega)$ for $p \leq q \leq 4 p /(4-m p)$
  if $m p<4$, or for $p \leq q<\infty$ if $m p=4$, or for $p \leq q \leq \infty$ if $m p>4$.
\end{example}


\section{Imbedding Inequalities Involving Weighted Norms}


\begin{para}
  The technique of mapping a standard cusp onto its associated standard cone via (28) and (29) is central to the proof of Theorem 4.51. Such a transformation introduces into any integrals involved a weight factor in the form of the Jacobian determinant (29). Accordingly, we must obtain imbedding inequalities for such standard cones involving $L^p$-norms weighted by powers of distance from the axial plane of the cone. Such inequalities are also useful in extending the imbedding theorem 4.12 to more general Sobolev spaces involving weighted norms. Many authors have treated the subject of weighted Sobolev spaces. We mention, in
  particular, Kufner's monograph [Ku] which focuses on a different class of weights depending on distance from the boundary of $\Omega$.
\end{para}

We begin with some one-dimensional inequalities for functions continuously differentiable on an open interval $(0, T)$ in $\mathbb{R}$.


\begin{lemma}
  Let $v>0$ and $u \in C^1(0, T)$. If $\int_0^T\left|u^{\prime}(t)\right| t^\nu d t<\infty$, then $\lim _{t \rightarrow 0+}|u(t)| t^\nu=0$.
\end{lemma}

\begin{proof}
  Let $\varepsilon>0$ be given and fix $s$ in $(0, T / 2)$ small enough so that for any $t$, $0<t<s$, we have
  \[
  \int_t^s\left|u^{\prime}(\tau)\right| \tau^\nu d \tau<\varepsilon / 3
  \]
  Now there exists $\delta$ in $(0, s)$ such that
  \[
 \delta^\nu\left|u^{\prime}(T / 2)\right|<\varepsilon / 3 \quad \text { and } \quad(\delta / s)^\nu \int_s^{T / 2}\left|u^{\prime}(\tau)\right| \tau^\nu d \tau<\varepsilon / 3
  \]
  If $0<t \leq \delta$, then
  \[
  |u(t)| \leq|u(T / 2)|+\int_t^{T / 2}\left|u^{\prime}(\tau)\right| d \tau
  \]
  so that
  \[
  t^\nu|u(t)| \leq \delta^\nu|u(T / 2)|+\int_t^s\left|u^{\prime}(\tau)\right| \tau^\nu d \tau+(\delta / s)^\nu \int_s^{T / 2}\left|u^{\prime}(\tau)\right| \tau^\nu d \tau<\varepsilon
  \]
  Hence $\lim _{t \rightarrow 0+}|u(t)| t^\nu=0$.
\end{proof}


\begin{lemma}
  Let $v>0, p \geq 1$, and $u \in C^1(0, T)$. Then
  \[
  \int_0^T|u(t)|^p t^{\nu-1} d t \leq \frac{v+1}{v T} \int_0^T|u(t)|^p t^\nu d t+\frac{p}{v} \int_0^T|u(t)|^{p-1}\left|u^{\prime}(t)\right| t^\nu d t .
  \]
\end{lemma}

\begin{proof}
  We may assume without loss of generality that the right side of (30) is finite and that $p=1$. Integration by parts gives
  \[
  \int_0^T|u(t)|\left(v t^{\nu-1}-\frac{v+1}{T} t^\nu\right) d t=-\int_0^T\left(t^\nu-\frac{1}{T} t^{\nu+1}\right) \frac{d}{d t}|u(t)| d t,
  \]
  the previous lemma assuring the vanishing of the integrated term at zero. Transposition and estimation of the term on the right now yields
  \[
  \nu \int_0^T|u(t)| t^{\nu-1} d t \leq \frac{\nu+1}{T} \int_0^T|u(t)| t^\nu d t+\int_0^T\left|u^{\prime}(t)\right| t^\nu d t
  \]
  which is $(30)$ for $p=1$.
\end{proof}


\begin{lemma}
  Let $v>0, p \geq 1$, and $u \in C^1(0, T)$. Then
  \[
  \begin{aligned}
  \sup _{0<t<T}|u(t)|^p & \leq \frac{2}{T} \int_0^T|u(t)|^p d t+p \int_0^T|u(t)|^{p-1}\left|u^{\prime}(t)\right| d t \\
  \sup _{0<t<T}|u(t)|^p t^\nu & \leq \frac{\nu+3}{T} \int_0^T|u(t)|^p t^\nu d t+2 p \int_0^T|u(t)|^{p-1}\left|u^{\prime}(t)\right| t^\nu d t
  \end{aligned}
  \]
\end{lemma}

\begin{proof}
  Again the inequalities need only be proved for $p=1$. If $0<t \leq T / 2$, we obtain by integration by parts
  \[
  \int_0^{T / 2}\left|u\left(t+\frac{T}{2}-\tau\right)\right| d \tau=\frac{T}{2}|u(t)|-\int_0^{T / 2} \tau \frac{d}{d \tau}\left|u\left(t+\frac{T}{2}-\tau\right)\right| d \tau
  \]
  whence
  \[
  |u(t)| \leq \frac{2}{T} \int_0^T|u(\sigma)| d \sigma+\int_0^T\left|u^{\prime}(\sigma)\right| d \sigma
  \]
  For $T / 2 \leq t<T$ the same inequality results from the partial integration of $\int_0^{T / 2}|u(t+\tau-T / 2)| d \tau$. This proves $(31)$ for $p=1$. Replacing $u(t)$ by $u(t) t^\nu$ in this inequality, we obtain
  \[
  \begin{aligned}
  \sup _{0<t<T}|u(t)| t^\nu \leq & \frac{2}{T} \int_0^T|u(t)| t^\nu d t+\int_0^T\left(\left|u^{\prime}(t)\right| t^\nu+v|u(t)| t^{\nu-1}\right) d t \\
  \leq & \frac{2}{T} \int_0^T|u(t)| t^\nu d t+\int_0^T\left|u^{\prime}(t)\right| t^\nu d t \\
  & +v\left(\frac{v+1}{\nu T} \int_0^T|u(t)| t^\nu d t+\frac{1}{v} \int_0^T\left|u^{\prime}(t)\right| t^\nu d t\right)
  \end{aligned}
  \]
  where (30) has been used to obtain the last inequality. This is the desired result (32) for $p=1$.
  
\end{proof}


\begin{para}
  Now we return to $\mathbb{R}^n$ for $n \geq 2$. If $x \in \mathbb{R}^n$, we shall make use of the spherical polar coordinate representation
  \[
  x=(\rho, \phi)=\left(\rho, \phi_1, \ldots, \phi_{n-1}\right)
  \]
  where $\rho \geq 0,-\pi \leq \phi_1 \leq \pi, 0 \leq \phi_2, \ldots, \phi_{n-1} \leq \pi$, and
  \[
  \begin{aligned}
  & x_1=\rho \sin \phi_1 \sin \phi_2 \cdots \sin \phi_{n-1}, \\
  & x_2=\rho \cos \phi_1 \sin \phi_2 \cdots \sin \phi_{n-1} \text {, } \\
  & x_3=\quad \rho \cos \phi_2 \cdots \sin \phi_{n-1} \text {, } \\
  & x_n=\quad \rho \cos \phi_{n-1} \text {. } \\
  &
  \end{aligned}
  \]
  The volume element is
  \[
  d x=d x_1 d x_2 \cdots d x_n=\rho^{n-1} \prod_{j=1}^{n-1} \sin ^{j-1} \phi_j d \rho d \phi,
  \]
  where $d \phi=d \phi_1 \cdots d \phi_{n-1}$.
  We define functions $r_k=r_k(x)$ for $1 \leq k \leq n$ as follows:
  \[
  \begin{aligned}
  & r_1(x)=\rho\left|\sin \phi_1\right| \prod_{j=2}^{n-1} \sin \phi_j, \\
  & r_k(x)=\rho \prod_{j=k}^{n-1} \sin \phi_j, \quad k=2,3, \ldots, n-1, \\
  & r_n(x)=\rho .
  \end{aligned}
  \]
  For $1 \leq k \leq n-1, r_k(x)$ is the distance of $x$ from the coordinate plane spanned by the axes $x_{k+1}, \ldots, x_n$; of course $r_n(x)$ is the distance of $x$ from the origin. In connection with the use of product symbols of the form $P=\prod_{j=k}^m P_j$, we follow the convention that $P=1$ if $m<k$.
  Let $\mathcal{C}$ be an open, conical domain in $\mathbb{R}^n$ specified by the inequalities
  \[
  0<\rho<a, \quad-\beta_1<\phi_1<\beta_1, \quad 0 \leq \phi_j<\beta_j, \quad(2 \leq j \leq n-1)
  \]
  where $0<\beta_i \leq \pi$. (Inequalities ``$<$'' in (33) corresponding to any $\beta_i=\pi$
  are replaced by ``$\leq$''. If all $\beta_i=\pi$, the first inequality is replaced with $0 \leq \rho<a$.)
  Note that any standard cone $\mathcal{C}_k$ (introduced in section 4.50) is of the form (33)
  for some choice of the parameters $\beta_i$, $1 \leq i \leq n-1$.
\end{para}


\begin{lemma}
  Let $\mathcal{C}$ be as specified by (33) and let $p \geq 1$. Suppose that either $m=k=1$, or $2 \leq m \leq n$ and $1 \leq k \leq n$. Let $1-k<v_1 \leq v \leq v_2<\infty$. Then there exists a constant $K=K\left(m, k, n, p, v 1, v 2, \beta_1, \ldots, \beta_{n-1}\right)$ independent of $v$ and $a$, such that for every $u \in C^1(\mathcal{C})$ we have
  \[
  \begin{aligned}
  & \int_{\mathcal{C}}|u(x)|^p\left[r_k(x)\right]^\nu\left[r_m(x)\right]^{-1} \d x \\
  & \leq K \int_{\mathcal{C}}|u(x)|^{p-1}\left(\frac{1}{a}|u(x)|+|\grad u(x)|\right)\left[r_k(x)\right]^\nu \d x .
  \end{aligned}
  \]
\end{lemma}

\begin{proof}
  Once again it is sufficient to establish (34) for $p=1$. Let $\mathcal{C}_{+}$be the set $\left\{x=(\rho, \phi): \phi_1 \geq 0\right\}$ and $\mathcal{C}_{-}$the set $\left\{x=(\rho, \phi): \phi_1 \leq 0\right\}$. Then $\mathcal{C}=\mathcal{C}_{+} \cup \mathcal{C}_{-}$. We prove (34) only for $\mathcal{C}_{+}$(which, however, we continue to call $\mathcal{C}$ ); a similar proof holds for $\mathcal{C}_{-}$, so that (34) holds for the given $\mathcal{C}$. Accordingly, assume $\mathcal{C}=\mathcal{C}_{+}$.
  For $k \leq m$ we may write (34) in the form (taking $p=1$ )
  \[
  \begin{aligned}
  \int_{\mathcal{C}}|u| & \prod_{j=2}^{k-1} \sin ^{j-1} \phi_j \prod_{j=k}^{m-1} \sin ^{\nu+j-1} \phi_j \prod_{j=m}^{n-1} \sin ^{\nu+j-2} \phi_j \rho^{\nu+n-2} d \rho d \phi \\
  & \leq K \int_{\mathcal{C}}\left(\frac{1}{a}|u|+|\grad u|\right) \prod_{j=2}^{k-1} \sin ^{j-1} \phi_j \prod_{j=k}^{n-1} \sin ^{\nu+j-1} \phi_j \rho^{\nu+n-1} d \rho d \phi .
  \end{aligned}
  \]
  For $k>m \geq 2$ we may write (34) in the form
  \[
  \begin{aligned}
  \int_{\mathcal{C}}|u| & \prod_{j=2}^{m-1} \sin ^{j-1} \phi_j \prod_{j=m}^{k-1} \sin ^{j-2} \phi_j \prod_{j=k}^{n-1} \sin ^{\nu+j-2} \phi_j \rho^{\nu+n-2} d \rho d \phi \\
  & \leq K \int_{\mathcal{C}}\left(\frac{1}{a}|u|+|\grad u|\right) \prod_{j=2}^{k-1} \sin ^{j-1} \phi_j \prod_{j=k}^{n-1} \sin ^{\nu+j-1} \phi_j \rho^{\nu+n-1} d \rho d \phi .
  \end{aligned}
  \]
  By virtue of the restrictions placed on $v, m$, and $k$ in the statement of the lemma,
  each of the two inequalities above is a special case of
  \[
  \begin{aligned}
  \int_{\mathcal{C}}|u| & \prod_{j=1}^{i-1} \sin ^{\mu_j} \phi_j \prod_{j=i}^{n-1} \sin ^{\mu_j-1} \phi_j \rho^{\nu+n-2} d \rho d \phi \\
  & \leq K \int_{\mathcal{C}}\left(\frac{1}{a}|u|+|\grad u|\right) \prod_{j=1}^{n-1} \sin ^{\mu_j} \phi_j \rho^{\nu+n-1} d \rho d \phi
  \end{aligned}
  \]
  where $1 \leq i \leq n, \mu_j \geq 0$, and $\mu_j>0$ if $j \geq i$.
  We prove (35) by backwards induction on $i$. For $i=n,(35)$ is obtained by applying Lemma 4.58
  to $u$ considered as a function of $\rho$ on $(0, a)$, and then integrating the remaining 
  variables with the appropriate weights. Assume, therefore, that (35) has been proved for $i=k+1$ 
  where $1 \leq k \leq n-1$. We prove it must also hold for $i=k$.
  If $\beta_k<\pi$, we have
  \[
  \sin \phi_k \leq \phi_k \leq K_1 \sin \phi_k, \quad 0 \leq \phi_k \leq \beta_k
  \]
  where $K_1=K_1\left(\beta_k\right)$. By Lemma 4.58, and since
  \[
  \left|\frac{\partial u}{\partial \phi_k}\right| \leq \rho|\grad u| \prod_{j=k+1}^{n-1} \sin \phi_j,
  \]
  we have
  \begin{align*}
    & \int_0^{\beta_k} |u(\rho,\phi)| \sin^{\mu_k-1} \phi_k \d \phi_k \\
    \leq & \int_0^{\beta_k} |u| \phi_k^{\mu_k-1} \d \phi_k \\
    \leq & K_2 \int_0^{\beta_k} \biggl(|u| 
      + |\grad u| \rho \prod_{j=k+1}^{n-1} \sin\phi_j\biggr) \phi_k^{\mu_k} \d \phi_k \\
    \leq & \int_{0}^{\beta_k} \biggl(|u| 
      + |\grad u| \rho \prod_{j=k+1}^{n-1} \sin\phi_j\biggr) \sin^{\mu_k} \phi_k \d \phi_k.
  \end{align*}
  Note that $K_2$, and hence $K_3$, depends on $\beta_k$ but may be chosen independent of $\mu_k$,
  and hence of $v$, under the conditions of the lemma.
  If $\beta_k=\pi$, we obtain (37) by writing $\int_0^\pi=\int_0^{\pi / 2}+\int_{\pi / 2}^\pi$
  and using the inequalities
  \[
  \begin{array}{ll}
  \sin \phi_k \leq \phi_k \leq(\pi / 2) \sin \phi_k & \text { if } \quad 0 \leq \phi_k \leq \pi / 2 \\
  \sin \phi_k \leq \pi-\phi_k \leq(\pi / 2) \sin \phi_k & \text { if } \quad \pi / 2 \leq \phi_k \leq \pi
  \end{array}
  \]
  We now obtain, using (37) and the induction hypothesis,
  \[
  \begin{aligned}
  & \int_{\mathcal{C}}|u| \prod_{j=1}^{k-1} \sin ^{\mu_j} \phi_j \prod_{j=k}^{n-1} \sin ^{\mu_j-1} \phi_j \rho^{\nu+n-2} d \rho d \phi \\
  & \leq \int_0^a \rho^{\nu+n-2} d \rho \prod_{j=1}^{k-1} \int_0^{\beta_j} \sin ^{\mu_j} \phi_j d \phi_j \\
  & \times \prod_{j=k+1}^{n-1} \int_0^{\beta_j} \sin ^{\mu_j-1} \phi_j d \phi_j \times \int_0^{\beta_k}|u| \sin ^{\mu_k-1} \phi_k d \phi_k \\
  & \leq K_3 \int_{\mathcal{C}}|\grad u| \prod_{j=1}^{n-1} \sin ^{\mu_j} \phi_j \rho^{v+n-1} d \rho d \phi \\
  &+K_3 \int_{\mathcal{C}}|u| \prod_{j=1}^k \sin ^{\mu_j} \phi_j \prod_{j=k+1}^{n-1} \sin ^{\mu_j-1} \phi_j \rho^{\nu+n-2} d \rho d \phi \\
  & \leq K \int_{\mathcal{C}}\left(\frac{1}{a}|u|+\mid \grad^{n \mid}\right) \prod_{j=1}^{n-1} \sin ^{\mu_j} \phi_j \rho^{\nu+n-1} d \rho d \phi
  \end{aligned}
  \]
  This completes the induction establishing (35) and hence the lemma.
\end{proof}

The following lemma provides a weighted imbedding inequality for the $L^q$-norm of a function defined on a 
conical domain of the type (33) in terms of the $W^{m,p_{-}}$ norm, both norms being weighted with a power of 
distance $r_k$ from a coordinate $(n-k)$-plane. It provides the core of the proof of Theorem~4.51.


\begin{lemma}
  Let $\mathcal{C}$ be as specified by (33) and let $p \geq 1$ and $1 \leq k \leq n$.
  Suppose that $\max \{1-k, p-n\}<v_1<v_2<\infty$. Then there exists a constant
  $K=K\left(k, n, p, v_1, v_2, \beta_1, \ldots, \beta_{n-1}\right)$, independent of $a$,
  such that for every $v$ satisfying $v_1 \leq v \leq v_2$ and every function
  $u \in C^1(\mathcal{C}) \cap C(\overline{\mathcal{C}})$ we have
  \[
  \begin{aligned}
  & \left(\int_{\mathcal{C}}|u(x)|^q\left[r_k(x)\right]^\nu \d x\right)^{1 / q} \\
  & \quad \leq K\left(\int_{\mathcal{C}}\left(\frac{1}{a^p}|u(x)|^p+|\grad u(x)|^p\right)\left[r_k(x)\right]^\nu \d x\right)^{1 / p}
  \end{aligned}
  \]
  where $q=(v+n) p /(v+n-p)$.
\end{lemma}

\begin{proof}
  Let $\delta=(v+n-1) p /(v+n-p)$, let $s=(\nu+n-1) / \nu$, and let $s^{\prime}=(v+n-1) /(n-1)$. We have by Hölder's inequality and Lemma 4.61 (the case $m=k)$
  \[
  \begin{aligned}
  \int_{\mathcal{C}}|u(x)|^q\left[r_k(x)\right]^\nu \d x \leq & \left(\int_{\mathcal{C}}|u|^\delta r_k^{\nu-1} \d x\right)^{1 / s}\left(\int_{\mathcal{C}}|u|^{n \delta /(n-1)} r_k^{n \nu /(n-1)} \d x\right)^{1 / s^{\prime}} \\
  \leq & K_1\left(\int_{\mathcal{C}}|u|^{\delta-1}\left(\frac{1}{a}|u|+|\grad u|\right) r_k^\nu \d x\right)^{1 / s} \\
  & \times\left(\int_{\mathcal{C}}|u|^{n \delta /(n-1)} r_k^{n v /(n-1)} \d x\right)^{1 / s^{\prime}}
  \end{aligned}
  \]
  In order to estimate the last integral above we adopt the notation
  \[
  \rho^*=\left(\phi_1, \ldots, \phi_{n-1}\right), \quad \phi_j^*=\left(\rho, \phi_1, \ldots, \hat{\phi}_j, \phi_{j+1}, \ldots, \phi_{n-1}\right), \quad 1 \leq j \leq n-1,
  \]
  where the caret denotes omission of a component. Let
  \[
  \begin{gathered}
  \mathcal{C}_0^*=\left\{\rho^*:\left(\rho, \rho^*\right) \in \mathcal{C} \text { for } 0<\rho<a\right\} \\
  \mathcal{C}_j^*=\left\{\phi_j^*:(\rho, \phi) \in \mathcal{C} \text { for } 0<\phi_j<\beta_j\right\}
  \end{gathered}
  \]
  $\mathcal{C}_0^*$ and $\mathcal{C}_j^*,(1 \leq j \leq n-1)$, are domains in $\mathbb{R}^{n-1}$. We define functions $F_0$ on $\mathcal{C}_0^*$ and $F_j$ on $\mathcal{C}_j^*$ as follows:
  \[
  \begin{aligned}
  \left(F_0\left(\rho^*\right)\right)^{n-1}= & \sup _{0<\rho<a}\left(|u|^\delta \rho^{v+n-1}\right) \prod_{i=k}^{n-1} \sin ^v \phi_i \prod_{i=2}^{n-1} \sin ^{i-1} \phi_i, \\
  \left(F_j\left(\phi_j^*\right)\right)^{n-1}= & \left(\sup _{0<\phi_j<\beta_j}\left(|u|^\delta \sin ^{v+n-1} \phi_j\right)\right) \rho^{v+n-2} \\
  & \times \prod_{i=k}^{n-1} \sin ^v \phi_i \prod_{i=2}^{j-1} \sin ^{i-1} \phi_i \prod_{i=j+1}^{n-1} \sin ^{i-2} \phi_i
  \end{aligned}
  \]
  Then we have
  \[
  |u|^{n \delta /(n-1)} r_k^{n \nu /(n-1)} \rho^{n-1} \prod_{i=2}^{n-1} \sin ^{i-1} \phi_i \leq F_0\left(\rho^*\right) \prod_{j=1}^{n-1} F_j\left(\phi_j^*\right) .
  \]
  Applying the combinatorial lemma 4.23 with $k=n-1=\lambda$ we obtain
  \[
  \begin{aligned}
  & \int_{\mathcal{C}}|u|^{n \delta /(n-1)} r_k^{n \nu /(n-1)} \d x \\
  & \leq \int_{\mathcal{C}} F_0\left(\rho^*\right) \prod_{j=1}^{n-1} F_j\left(\phi_j^*\right) d \rho d \phi \\
  & \leq\left(\int_{\mathcal{C}_0^*}\left(F_0\left(\rho^*\right)\right)^{n-1} d \phi \prod_{j=1}^{n-1} \int_{\mathcal{C}_j^*}\left(F_j\left(\phi_j^*\right)\right)^{n-1} d \rho d \hat{\phi}_j\right)^{1 /(n-1)}
  \end{aligned}
  \]
  Now by Lemma 4.59, and since $|\partial u / \partial \rho| \leq|\grad u|$,
  \[
  \sup _{0<\rho<a}|u|^\delta \rho^{\nu+n-1} \leq K_2 \int_0^a|u|^{\delta-1}\left(\frac{1}{a}|u|+|\grad u|\right) \rho^{\nu+n-1} d \rho,
  \]
  where $K_2$ is independent of $v$ for $1-n<v_1 \leq v \leq v_2<\infty$. It follows that
  \[
  \int_{\mathcal{C}_0^*}\left(F_0\left(\rho^*\right)\right)^{n-1} d \phi \leq K_2 \int_{\mathcal{C}}|u|^{\delta-1}\left(\frac{1}{a}|u|+|\grad u|\right) r_k^\nu \d x
  \]
  Similarly, by making use of (36) or (38) as in Lemma 4.61, we obtain from Lemma 4.59
  \[
  \begin{aligned}
  & \sup _{0<\phi_j<\beta_j}|u|^\delta \sin ^{\nu+j-1} \phi_j \\
  & \leq K_{2, j} \int_0^{\beta_j}|u|^{\delta-1}\left(|u|+\left|\frac{\partial u}{\partial \phi_j}\right|\right) \sin ^{\nu+j-1} \phi_j d \phi_j \\
  & \leq K_{2, j} \int_0^{\beta_j}|u|^{\delta-1}\left(|u|+|\grad u| \rho \prod_{i=j+1}^{n-1} \sin \phi_i\right) \sin ^{v+j-1} \phi_j d \phi_j, \\
  &
  \end{aligned}
  \]
  since $\left|\partial u / \phi_j\right| \leq \rho \prod_{i=j+1}^{n-1} \sin \phi_i$. Hence
  \[
  \begin{aligned}
  \int_{\mathcal{C}_j^*} & \left(F_j\left(\phi_j^*\right)\right)^{n-1} d \rho d \hat{\phi}_j \\
  & \leq K_{2, j} \int_{\mathcal{C}}|\grad u \| u|^{\delta-1} r_k^\nu \d x+K_{2, j} \int_{\mathcal{C}}|u|^\delta r_k^\nu r_{j+1}^{-1} \d x \\
  & \leq K_{3, j} \int_{\mathcal{C}}|u|^{\delta-1}\left(\frac{1}{a}|u|+|\grad u|\right) r_k^\nu \d x,
  \end{aligned}
  \]
  where we have used Lemma 4.61 again to obtain the last inequality. Note that the constants $K_{2, j}$ and $K_{3, j}$ can be chosen independent of $v$ for the values of $v$ allowed. Substitution of (42) and (43) into (41) and then into (40) leads to
  \[
  \begin{aligned}
  \int_{\mathcal{C}}|u|^q r_k^\nu \d x \leq & K_4\left(\int_{\mathcal{C}}|u|^{\delta-1}\left(\frac{1}{a}|u|+|\grad u|\right) r_k^\nu \d x\right)^{1 / s+n /\left((n-1) s^{\prime}\right)} \\
  \leq & K_4\left(\left[\int_{\mathcal{C}}|u|^q r_k^\nu \d x\right]^{(p-1) / p}\right. \\
  & \left.\times\left[2^{p-1} \int_{\mathcal{C}}\left(\frac{1}{a^p}|u|^p+|\grad u|^p\right) r_k^\nu \d x\right]^{1 / p}\right)^{(\nu+n) /(\nu+n-1)}
  \end{aligned}
  \]
  Since $(v+n-1) /(v+n)-(p-1) / p=1 / q$, inequality (39) follows by cancellation for, since $u$ is bounded on $\mathcal{C}$ and $v>1-n, \int_{\mathcal{C}}|u|^q r_k^\nu \d x$ is finite.
\end{proof}


\begin{remarks}
  \item The assumption that $u \in C(\overline{\mathcal{C}})$ was made only to ensure that the above cancellation was justified. In fact, the lemma holds for any $u \in C^1(\mathcal{C})$.
  \item If $1-k<v_1<v_2<\infty$ and $v_1 \leq v \leq v_2$, where $p \geq v+n$, then (39) holds for any $q$ satisfying $1 \leq q<\infty$. It is sufficient to prove this for large $q$. If $q \geq(v+n) /(\nu+n-1)$, then $q=(\nu+n) s /(\nu+n-s)$ for some $s$ satisfying $1 \leq s<p$. Thus
  \[
  \begin{aligned}
  & \left(\int_{\mathcal{C}}|u|^q r_k^v \d x\right)^{s / q} \leq K \int_{\mathcal{C}}\left(\frac{1}{a^s}|u|^s+|\grad u|^s\right) r_k^\nu \d x \\
  & \quad \leq K\left(2^{(p-2) / s} \int_{\mathcal{C}}\left(\frac{1}{a^p}|u|^p+|\grad u|^p\right) r_k^\nu \d x\right)^{s / p}\left(\int_{\mathcal{C}} r_k^\nu \d x\right)^{(p-s) / p}
  \end{aligned}
  \]
  which yields (39) since the last factor is finite.
  \item If $v=m$, a positive integer, then (39) can be obtained very simply as follows. Let $y=(x, z)=\left(x_1, \ldots, x_n, z_1, \ldots, z_m\right)$ denote a point in $\mathbb{R}^{n+m}$ and define $u^*(y)=u(x)$ for $x \in \mathcal{C}$. If
  \[
  \mathcal{C}^*=\left\{y \in \mathbb{R}^{n+m}: y=(x, z), x \in \mathcal{C}, 0<z_j<r_k(x), 1 \leq j \leq m\right\},
  \]
  then $\mathcal{C}^*$ satisfies the cone condition in $\mathbb{R}^{n+m}$, whence by Theorem 4.12 we have, putting $q=(n+m) p /(n+m-p)$,
  \[
  \begin{aligned}
  \left(\int_{\mathcal{C}}|u|^q r_k^m \d x\right)^{1 / q} & =\left(\int_{\mathcal{C}^*}\left|u^*(y)\right|^q d y\right)^{1 / q} \\
  & \leq K\left(\int_{\mathcal{C}^*}\left(\frac{1}{a^p}\left|u^*(y)\right|^p+\left|\grad u^*(y)\right|^p\right) d y\right)^{1 / p} \\
  & =K\left(\int_{\mathcal{C}}\left(\frac{1}{a^p}|u|^p+|\grad u|^p\right) r_k^m \d x\right)^{1 / p}
  \end{aligned}
  \]
  since $\left|\grad u^*(y)\right|=|\grad u(x)|, u^*$ being independent of $z$.
  \item Suppose that $u \in C_0^{\infty}\left(\mathbb{R}^n\right)$, or, more generally, that
  \[
  \int_{\mathbb{R}^n}|u(x)|^p\left[r_k(x)\right]^v \d x<\infty
  \]
  with $v$ as in the above lemma. If we take $\beta_i=\pi, 1 \leq i \leq n-1$, and let $a \rightarrow \infty$ in (39), we obtain
  \[
  \left(\int_{\mathbb{R}^n}|u(x)|^q\left[r_k(x)\right]^v \d x\right)^{1 / q} \leq K\left(\int_{\mathbb{R}^n}|\grad u(x)|^p\left[r_k(x)\right]^v \d x\right)^{1 / p} .
  \]
  This generalizes (the case $m=1$ of) Sobolev's inequality, Theorem 4.31.
\end{remarks}

As final preparations for the proofs of Theorems $4.51-4.53$ we need to obtain weighted analogs of the $L^{\infty}$ and Hölder imbedding inequalities provided by Theorem 4.12. It is convenient here to deal with arbitrary domains satisfying the cone condition rather than the special case $\mathcal{C}$ considered in the lemmas above. The following elementary result will be needed.


\begin{lemma}
  Let $z \in \mathbb{R}^k$ and let $\Omega$ be a domain of finite volume in $\mathbb{R}^k$. If $0 \leq v<k$, then
  \[
  \int_{\Omega}|x-z|^{-v} \d x \leq \frac{K}{k-v}(\vol(\Omega))^{1-v / k}
  \]
  where the constant $K$ depends on $\nu$ and $k$, but not on $z$ or $\Omega$.
\end{lemma}

\begin{proof}
  Let $B$ be the ball in $\mathbb{R}^k$ having centre $z$ and the same volume as $\Omega$. It is easily seen that the left side of the above inequality does not exceed $\int_B|x-z|^{-v} \d x$, and that the inequality holds for $\Omega=B$
\end{proof}


\begin{lemma}
  Let $\Omega \subset \mathbb{R}^n$ satisfy the cone condition. Let $1 \leq k \leq n$ and let $P$ be an $(n-k)$-dimensional plane in $\mathbb{R}^n$. Denote by $r(x)$ the distance from $x$ to $P$. If $0 \leq v<p-n$, then for all $u \in C^1(\Omega)$ we have
  \[
  \sup _{x \in \Omega}|u(x)| \leq K\left(\int_{\Omega}\left(|u(x)|^p+|\grad u|^p\right)[r(x)]^\nu \d x\right)^{1 / p}
  \]
  where the constant $K$ may depend on $v, n, p, k$, and the cone $C$ determining the cone condition for $\Omega$, but not on $u$.
\end{lemma}

\begin{proof}
  Throughout this proof $A_i$ and $K_i$ will denote various constants depending on one or more of the parameters on which $K$ is allowed to depend above. It is
  sufficient to prove that if $C$ is a finite cone contained in $\Omega$ having vertex at, say, the origin, then
  \[
  |u(0)| \leq K\left(\int_C\left(|u(x)|^p+|\grad u|^p\right)[r(x)]^v \d x\right)^{1 / p}
  \]
  For $0 \leq j \leq n$, let $A_j$ denote the supremum of the Lebesgue $j$-dimensional measure of the projection of $C$ onto $\mathbb{R}^j$, taken over all $j$-dimensional subspaces $\mathbb{R}^j$ of $\mathbb{R}^n$. Writing $x=\left(x^{\prime}, x^{\prime \prime}\right)$ where $x^{\prime}=\left(x_1, \ldots, x_{n-k}\right)$ and $x^{\prime \prime}=\left(x_{n-k+1}, \ldots, x_n\right)$, we may assume, without loss of generality, that $P$ is orthogonal to the coordinate axes corresponding to the components of $x^{\prime \prime}$. Define
  \[
  \begin{aligned}
  S & =\left\{x^{\prime} \in \mathbb{R}^{n-k}:\left(x^{\prime}, x^{\prime \prime}\right) \in C \text { for some } x^{\prime \prime} \in \mathbb{R}^k\right\}, \\
  R\left(x^{\prime}\right) & =\left\{x^{\prime \prime} \in \mathbb{R}^k:\left(x^{\prime}, x^{\prime \prime}\right) \in C\right\} \quad \text { for each } x^{\prime} \in S .
  \end{aligned}
  \]
  For $0 \leq t \leq 1$ we denote by $C_t$ the cone $\{t x: x \in C\}$ so that $C_t \subset C$ and $C_t=C$ if $t=1$. For $C_t$ we define the quantities $A_{t, j}, S_t$, and $R_t\left(x^{\prime}\right)$ analogously to the similar quantities defined for $C$. Clearly $A_{t, j}=t^j A_j$. If $x \in C$, we have
  \[
  u(x)=u(0)+\int_0^1 \frac{d}{d t} u(t x) d t
  \]
  so that
  \[
  |u(0)| \leq|u(x)|+|x| \int_0^1|\grad u(t x)| d t
  \]
  Setting $V=\vol(C)$ and $a=\sup _{x \in C}|x|$, and integrating the above inequality over $C$, we obtain
  \[
  \begin{aligned}
  V|u(0)| & \leq \int_C|u(x)| \d x+a \int_C \int_0^1|\grad u(t x)| d t \d x \\
  & =\int_C|u(x)| \d x+a \int_0^1 t^{-n} d t \int_{C_t}|\grad u(x)| \d x .
  \end{aligned}
  \]
  Let $z$ denote the orthogonal projection of $x$ onto $P$. Then $r(x)=\left|x^{\prime \prime}-z^{\prime \prime}\right|$. Since $0 \leq v<p-n$, we have $p>1$, and so by the previous lemma
  \[
  \begin{aligned}
  \int_{C_t}[r(x)]^{-v /(p-1)} \d x & =\int_{S_t} \d x^{\prime} \int_{R_t\left(x^{\prime}\right)}\left|x^{\prime \prime}-z^{\prime \prime}\right|^{-v /(p-1)} \d x^{\prime \prime} \\
  & \leq K_1 \int_{S_t}\left[A_{t, k}\right]^{1-v /(k(p-1))} \d x^{\prime} \\
  & \leq K_1\left[A_{t, k}\right]^{1-v /(k(p-1))}\left[A_{t, n-k}\right]=K_2 t^{n-v /(p-1)}
  \end{aligned}
  \]
  It follows that
  \[
  \begin{aligned}
  & \int_{C_t}|\grad u(x)| \d x \\
  & \quad \leq\left(\int_{C_t}|\grad u(x)|^p[r(x)]^\nu \d x\right)^{1 / p}\left(\int_{C_t}[r(x)]^{-v /(p-1)} \d x\right)^{1 / p^{\prime}} \\
  & \quad \leq K_3 t^{n-(v+n) / p}\left(\int_{C_t}|\grad u(x)|^p[r(x)]^v \d x\right)^{1 / p}
  \end{aligned}
  \]
  Hence, since $v<p-n$,
  \[
  \int_0^1 t^{-n} d t \int_{C_t}|\grad u(x)| \d x \leq K_4\left(\int_C|\grad u(x)|^p[r(x)]^v \d x\right)^{1 / p}
  \]
  Similarly,
  \[
  \begin{aligned}
  \int_C|u(x)| \d x & \leq\left(\int_C|u(x)|^p[r(x)]^\nu \d x\right)^{1 / p}\left(\int_C[r(x)]^{-\nu /(p-1)} \d x\right)^{1 / p^{\prime}} \\
  & \leq K_5\left(\int_C|u(x)|^p[r(x)]^\nu \d x\right)^{1 / p}
  \end{aligned}
  \]
  Inequality (45) now follows from (46), (48), and (49).
\end{proof}


\begin{lemma}
  Suppose all the conditions of the previous lemma are satisfied and, in addition, $\Omega$ satisfies the strong local Lipschitz condition. Then for all $u \in C^1(\Omega)$ we have
  \[
  \sup _{\substack{x, y \in \Omega \\ x \neq y}} \frac{|u(x)-u(y)|}{|x-y|^\mu} \leq K\left(\int_{\Omega}\left(|u(x)|^p+|\grad u(x)|^p\right)[r(x)]^\nu \d x\right)^{1 / p}
  \]
  where $\mu=1-(v+n) / p$ satisfies $0<\mu<1$, and $K$ is independent of $u$.
\end{lemma}

\begin{proof}
  The proof is the same as that given for inequality (15) in Lemma 4.28 except that the inequality
  \[
  \int_{\Omega_{t \sigma}}|\grad u(z)| d z \leq K_1 t^{n-(\nu+n) / p}\left(|\grad u(z)|^p[r(z)]^\nu d z\right)^{1 / p}
  \]
  is used in (16) in place of the special case $v=0$ actually used there. Inequality (51) is obtained in the same way as (47) above.
\end{proof}


\section{Proofs of Theorems 4.51--4.53}


\begin{lemma}
  Let $\bar{v} \geq 0$. If $\bar{v}>p-n$, let $1 \leq q \leq(\bar{v}+n) /(\bar{v}+n-p)$; otherwise, let $1 \leq q<\infty$. There exists a constant $K=K(n, p, \bar{v})$ such that for every standard cusp $Q_{k, \lambda}$ (see Paragraph 4.50) for which $(\lambda-1) k \equiv v \leq \bar{v}$, and every $u \in C^1\left(Q_{k, \lambda}\right)$, we have
  \[
  \|u\|_{0, q, Q_{k, \lambda}} \leq K\|u\|_{1, p, Q_{k, \lambda}} .
  \]
\end{lemma}

\begin{proof}
  Since each $Q_{k, \lambda}$ has the segment property, it suffices to prove (52) for $u \in C^1\left(\overline{Q_{k, \lambda}}\right)$. We first do so for given $k$ and $\lambda$ and then show that $K$ may be chosen to be independent of these parameters.
  First suppose $\bar{v}>p-n$. It suffices to prove (52) for
  \[
  q=(\bar{v}+n) /(\bar{v}+n-p)
  \]
  For $u \in C^1\left(\overline{Q_{k, \lambda}}\right)$ define $\tilde{u}(y)=u(x)$, where $y$ is related to $x$ by (27) and (28). Thus $\tilde{u} \in C^1\left(\mathcal{C}_k\right) \cap C\left(\overline{\mathcal{C}_k}\right)$, where $\mathcal{C}_k$ is the standard cone associated with $Q_{k, \lambda}$. By Lemma 4.62, and since $q \leq(\nu+n) p /(\nu+n-p)$, we have
  \[
  \begin{aligned}
  \|u\|_{0, q, Q_{k, \lambda}} & =\left(\lambda \int_{\mathcal{C}_k}|\tilde{u}(y)|^q\left[r_k(y)\right]^v d y\right)^{1 / q} \\
  & \leq K_1\left(\int_{\mathcal{C}_k}\left(|\tilde{u}(y)|^p+|\grad \tilde{u}(y)|^p\right)\left[r_k(y)\right]^v d y\right)^{1 / q} .
  \end{aligned}
  \]
  Now $x_j=r_k^{\lambda-1} y_j$ if $1 \leq j \leq k$ and $x_j=y_j$ if $k+1 \leq j \leq n$. Since $r_k^2=y_1^2+\cdots+y_k^2$ we have
  \[
  \frac{\partial x_j}{\partial y_i}= \begin{cases}\delta_{i j} r_k^{\lambda-1}+(\lambda-1) r_k^{\lambda-3} y_i y_j & \text { if } 1 \leq i, j \leq k \\ \delta_{i j} & \text { otherwise },\end{cases}
  \]
  where $\delta_{i i}=1$ and $\delta_{i j}=0$ if $i \neq j$. Since $r_k(y) \leq 1$ on $\mathcal{C}_k$ it follows that
  $|\grad \tilde{u}(y)| \leq K_2|\grad u(x)|$.
  Hence (52) follows from (53) in this case. For $\bar{v} \leq p-n$ and arbitrary $q$ the proof is similar, being based on Remark 2 of Paragraph 4.63 .
  In order to show that the constant $K$ in (52) can be chosen independent of $k$ and $\lambda$ provided $v=(\lambda-1) k \leq \bar{v}$, we note that it is sufficient to prove that there is a constant $\tilde{K}$ such that for any such $k, \lambda$ and all $v \in C^1\left(\mathcal{C}_k\right) \cap C\left(\overline{\mathcal{C}_k}\right)$ we have
  \[
  \begin{aligned}
  & \left(\int_{\mathcal{C}_k}|v(y)|^q\left[r_k(y)\right]^\nu d y\right)^{1 / q} \\
  & \quad \leq \tilde{K}\left(\int_{\mathcal{C}_k}\left(|v(y)|^p+|\grad v(y)|^p\right)\left[r_k(y)\right]^\nu d y\right)^{1 / p}
  \end{aligned}
  \]
  In fact, it is sufficient to establish (54) with $\tilde{K}$ depending on $k$ as we can then use the maximum of $\tilde{K}(k)$ over the finitely many values of $k$ allowed. We distinguish three cases.
  Case I $\bar{v}<p-n, 1 \leq q<\infty$. By Lemma 4.65 we have for $0 \leq v \leq \bar{v}$
  \[
  \sup _{y \in \mathcal{C}_k}|v(y)| \leq K(v)\left(\int_{\mathcal{C}_k}\left(|v(y)|^p+|\grad v(y)|^p\right)\left[r_k(y)\right]^\nu d y\right)^{1 / p}
  \]
  Since the integral on the right decreases as $\nu$ increases, we have $K(\nu) \leq K(\bar{\nu})$ and (54) now follows from (55) and the boundedness of $\mathcal{C}_k$.
  
  Case II $\bar{v}>p-n$. Again it is sufficient to deal with $q=(\bar{v}+n) p /(\bar{v}+n-p)$. From Lemma 4.62 we obtain
  where $s=(\nu+n) p /(\nu+n-p) \geq q$ and $K_1$ is independent of $v$ for $p-n<v_0 \leq \bar{v}$. By Hölder's inequality, and since $r_k(y) \leq 1$ on $\mathcal{C}_k$, we have
  \[
  \left(\int_{\mathcal{C}_k}|v|^q r_k^\nu d y\right)^{1 / q} \leq\left(\int_{\mathcal{C}_k}|v|^s r_k^\nu d y\right)^{1 / s}\left(\vol\left(\mathcal{C}_k\right)\right)^{(s-q) / s q}
  \]
  so that if $v_0 \leq v \leq \bar{v}$, then (54) follows from (56).
  If $p-n<0$, we can take $v_0=0$ and be done. Otherwise, $p \geq n \geq 2$. Fixing $v_0=(\bar{v}-n+p) / 2$, we can find $v_1$ such that $0 \leq v_1 \leq p-n$ (or $v_1=0$ if $p=n$ ) such that for $v_1 \leq v \leq v_0$ we have
  \[
  1 \leq t=\frac{(v+n)(\bar{v}+n) p}{(v+n)(\bar{v}+n)+(\bar{v}-v) p} \leq \frac{p}{1+\varepsilon_0}
  \]
  where $\varepsilon_0>0$ and depends only on $\bar{\nu}, n$, and $p$. Because of the latter inequality we may also assume $t-n<v_1$. Since $(v+n) t /(v+n-t)=q$ we have, again by Lemma 4.62 and Hölder's inequality,
  \[
  \begin{gathered}
  \left(\int_{\mathcal{C}_k}|v|^q r_k^\nu d y\right)^{1 / q} \leq K_2\left(\int_{\mathcal{C}_k}\left(|v|^t+|\grad v|^t\right) r_k^\nu d y\right)^{1 / t} \\
  2^{(p-t) / p t} K_2\left(\int_{\mathcal{C}_k}\left(|v|^p+|\grad v|^p\right) r_k^\nu d y\right)^{1 / p}\left(\vol\left(\mathcal{C}_k\right)\right)^{(p-t) / p t}
  \end{gathered}
  \]
  where $K_2$ is independent of $v$ for $v_1 \leq v \leq v_0$.
  In the case $v_1>0$ we can obtain a similar (uniform) estimate for $0 \leq v \leq v_1$ by the method of Case I. Combining this with (56) and (57), we prove (54) for this case.

  Case III $\bar{v}=p-n, 1 \leq q<\infty$. Fix $s \geq \max \{q, n /(n-1)\}$ and let $t=(\nu+n) s /(\nu+n+s)$, so $s=(\nu+n) t /(\nu+n-t)$. Then $1 \leq t \leq p s /(p+s)<p$ for $0 \leq v \leq \bar{v}$. Hence we can select $v_1 \geq 0$ such that $t-n<v_1<p-n$. The rest of the proof is similar to Case II. This completes the proof of the lemma.
\end{proof}


\begin{para}[Proof of Theorem 4.51]
  It is sufficient to prove only the special case $m=1$, for the general case then follows by induction on $m$. Let $q$ satisfy $p \leq q \leq(v+n) p /(v+n-p)$ if $v+n>p$, or $p \leq q<\infty$ otherwise. Clearly $q<n p /(n-p)$ if $n>p$ so in either case we have by Theorem 4.12
  \[
  \|u\|_{0, q, G} \leq K_1\|u\|_{1, p, G}
  \]
  for every $u \in C^1(\Omega)$ and that element $G$ of $\Gamma$ that satisfies the cone condition (if such a $G$ exists). If $G \in \Gamma$ does not satisfy the cone condition, and if $\Psi: G \rightarrow Q_{k, \lambda}$, where $(\lambda-1) k \leq v$, is the 1 -smooth mapping specified in the statement of the theorem. Then by Theorem 3.41 and Lemma 4.67
  \[
  \|u\|_{0, q, G} \leq K_2\left\|u \circ \Psi^{-1}\right\|_{0, q, Q_{k, \lambda}} \leq K_3\left\|u \circ \Psi^{-1}\right\|_{1, p, Q_{k, \lambda}} \leq K_4\|u\|_{1, p, G},
  \]
  where $K_4$ is independent of $G$. Thus, since $q / p \geq 1$,
  \[
  \begin{aligned}
  \|u\|_{0, q, \Omega}^q & \leq \sum_{G \in \Gamma}\|u\|_{0, q, G}^q \leq K_5 \sum_{G \in \Gamma}\left(\|u\|_{1, p, G}^p\right)^{q / p} \\
  & \leq K_5\left(\sum_{G \in \Gamma}\|u\|_{1, p, G}^p\right)^{q / p} \leq K_5 N^{q / p}\|u\|_{1, p, \Omega}^q,
  \end{aligned}
  \]
  where we have used the finite intersection property of $\Gamma$ to obtain the final inequality. The required imbedding inequality now follows by completion.
  
  If $v<m p-n$, we require that $W^{m,p}(\Omega) \rightarrow L^q(\Omega)$ also holds for $q=\infty$. This is a consequence of Theorem 4.52 proved below.
\end{para}


\begin{lemma}
  Let $0 \leq \bar{\nu}<m p-n$. Then there exists a constant $K=K(m, p, n, \bar{\nu})$ 
  such that if $Q_{k, \lambda}$ is any standard cusp domain for which $(\lambda-1) k=v \leq \bar{v}$
  and if $u \in C^m\left(Q_{k, \lambda}\right)$, then
  \[
  \sup _{x \in Q_{k, \lambda}}|u(x)| \leq K\|u\|_{m, p, Q_{k, \lambda}}
  \]
\end{lemma}

\begin{proof}
  Again it is sufficient to prove the lemma for the case $m=1$.
  If $u$ belongs to $C^1\left(Q_{k, \lambda}\right)$ where $(\lambda-1) k=v \leq \bar{v}$,
  then we have by Lemma 4.65 and via the
  method of the second paragraph of the proof of Lemma 4.67,
  \[
  \begin{aligned}
  \sup _{x \in Q_{k, \lambda}}|u(x)| & =\sup _{y \in \mathcal{C}_k}|\tilde{u}(y)| \\
  & \leq K_1\left(\int_{\mathcal{C}_k}\left(|\tilde{u}(y)|^p+|\grad \tilde{u}(y)|^p\right)\left[r_k(y)\right]^v d y\right)^{1 / p} \\
  & \leq K_2\left(\int_{Q_{k, \lambda}}\left(|u(x)|^p+|\grad u(x)|^p\right) \d x\right)^{1 / p} .
  \end{aligned}
  \]
  Since $r_k(y) \leq 1$ for $y \in \mathcal{C}_k$ it is evident that $K_1$, and hence $K_2$,
  can be chosen independent of $k$ and $\lambda$ provided $0 \leq v=(\lambda-1) k \leq \bar{\nu}$.
\end{proof}


\begin{para}[Proof of Theorem 4.52]
  It is sufficient to prove that
  \[
  W^{m,p}(\Omega) \rightarrow C_B^0(\Omega) .
  \]
  Let $u \in C^{\infty}(\Omega)$. If $x \in \Omega$, then $x \in G \subset \Omega$ for some domain $G$
  for which there exists a 1-smooth transformation $\Psi: G \rightarrow Q_{k, \lambda},(\lambda-1) k \leq \nu$, 
  as specified in the statement of the theorem. Thus
  \[
  \begin{aligned}
  |u(x)| & \leq \sup _{x \in G}|u(x)|=\sup _{y \in Q_{k, \lambda}}\left|u \circ \Psi^{-1}(y)\right| \\
  & \leq K_1\left\|u \circ \Psi^{-1}\right\|_{m, p, Q_{k, \lambda}} \leq K_2\|u\|_{m, p, G} \\
  & \leq K_2\|u\|_{m, p, \Omega},
  \end{aligned}
  \]
  where $K_1$ and $K_2$ are independent of $G$. The rest of the proof is similar
  to the second paragraph of the proof in Paragraph~4.16.
\end{para}

\begin{para}[Proof of Theorem 4.53]
  As in Lemma 4.28 it is sufficient to prove that
  \[
  W^{1, p}(\Omega) \rightarrow C^{0, \mu}(\overline{\Omega}) \quad \text { if } \quad 0<\mu \leq 1-\frac{n+v}{p}
  \]
  that is, that
  \[
  \sup _{\substack{x, y \in \Omega \\ x \neq y}} \frac{|u(x)-u(y)|}{|x-y|^\mu} \leq K\|u\|_{1, p, \Omega}
  \]
  holds when $v+n<p$ and $0<\mu \leq 1-(v+n) / p$. For $x, y \in \Omega$ satisfying $|x-y| \geq \delta$,
  (61) holds by virtue of (60). If $|x-y|<\delta$, then there exists $G \subset \Omega$ with $x, y \in G$,
  and a 1-smooth transformation $\Psi$ from $G$ onto a standard cusp $Q_{k, \lambda}$
  with $(\lambda-1) k \leq v$, satisfying the conditions of the theorem. Inequality (61)
  can then be derived from Lemma~4.66 by the same method used in the proof of Lemma~4.69.
  The details are left to the reader.
\end{para}