\chapter[The Sobolev Spaces $W^{k,p}(\varOmega)$]%
  {The Sobolev Spaces \boldmath{$W^{m,p}(\varOmega)$}}

In this chapter we introduce Sobolev spaces of integer order and establish
some of their most important properties. These spaces are defines over an arbitrary
domain $\Omega\subset \mathbb{R}^n$ and are subspaces of various Lebesgue
spaces $L^p(\Omega)$.


\section*{Definitions and Basic Properties}

\begin{para}[The Sobolev Norms]
  We define a functional $\|\cdot\|_{m,p}$, where $m$
  is a positive integer and $1\leq p\leq \infty$, as follows:
  \begin{align}
    \|u\|_{m,p} & = \biggl(\sum_{0\leq |\alpha|\leq m} \|D^\alpha u\|_p^p\biggr)^{1/p}
      \quad\text{if}\quad 1\leq p<\infty, \\
    \|u\|_{m,\infty} & = \max_{0\leq |\alpha| \leq m} \|D^\alpha u\|_{\infty}
  \end{align}
  for any function $u$ for which the right side makes sense,
  $\|\cdot\|_p$ beging, of course, the norm in $L^p(\Omega)$.
  In some situations where confusion of domains may occur
  we will use $\|u\|_{m,p,\Omega}$ in place of $\|u\|_{m,p}$.
  Evidently (1) or (2) defines a norm on 
  any vector space of functions on which the right side takes finite values provided 
  functions are identified in the space if they are equal almost everywhere in $\Omega$.
\end{para}

\begin{para}[Sobolev Spaces]
  For any positive integer $m$ and $1 \leq p \leq \infty$ we consider three vector spaces 
  on which $\|\cdot\|_{m,p}$ is a norm:
  \begin{enumerate}[(a)]
    \item $H^{m, p}(\Omega) \equiv$ the completion of $\left\{u \in C^m(\Omega):\|u\|_{m,p}<\infty\right\}$ with respect to the norm $\|\cdot\|_{m,p}$,
    \item $W^{m,p}(\Omega) \equiv\left\{u \in L^p(\Omega): D^\alpha u \in L^p(\Omega)\right.$ for $\left.0 \leq|\alpha| \leq m\right\}$, where $D^\alpha u$ is the weak (or distributional) partial derivative of Paragraph 1.62, and
    \item $W_0^{m, p}(\Omega) \equiv$ the closure of $C_0^{\infty}(\Omega)$ in the space $W^{m,p}(\Omega)$.
  \end{enumerate}
\end{para}

Equipped with the appropriate norm (1) or (2) these are called \emph{Sobolev spaces}
over $\Omega$. Clearly $W^{0,p}(\Omega)=L^p(\Omega)$,
and if $1\leq p<\infty$, $W_0^{0, p}(\Omega)=L^p(\Omega)$
because $C_0^{\infty}(\Omega)$ is dense in $L^p(\Omega)$.
(See Paragraph 2.30.) For any $m$, we have the obvious chain of imbeddings
\[W_0^{m, p}(\Omega) \rightarrow W^{m,p}(\Omega) \rightarrow L^p(\Omega).\]
We will show in Theorem 3.17 that $H^{m, p}(\Omega)=W^{m,p}(\Omega)$ for every domain 
$\Omega$. This result, published in 1964 by Meyers and Serrin [MS] ended much confusion 
about the relationship of these spaces that existed in the literature before that time.
It is surprising that this elementary result remained undiscovered for so long.

The spaces $W^{m,p}(\Omega)$ were introduced by Sobolev [So1, So2].
Many related spaces were being studied by other writers, in particular Morrey [Mo]
and Deny and Lions [DL].
Many different symbols ($W^{m,p}, H^{m, p}, P^{m, p}, L_n^m$, etc.) have been used to
denote these spaces and their variants, and before they became generally associated with 
the name of Sobolev they were sometimes referred to under other names, for example, as 
Beppo Levi spaces.

Numerous generalizations and specializions of the basic spaces $W^{m,p}(\Omega)$ have been 
constructed. Much of this literature originated in the Soviet Union. In particular, there 
are extensions that allow arbitrary real values of $m$ (see Chapter 7) which are interpreted as corresponding to fractional orders of differentiation.
There are weighted spaces that introduce weight functions into the $L^p$ norms;
see Kufner [Ku]. There are spaces of vector fields that are annihilated by differential 
operators like curl and divergence; see [DaL]. Other generalizations involve different 
orders of differentiation and different $L^p$ norms in different coordinate directions 
(anisotropic spaces---see [BIN1, BIN2]), and Orlicz-Sobolev spaces (see Chapter 8) modeled 
on the generalizations of $L^p$ spaces known as Orlicz spaces. Finally, there has been much 
work on the interaction between Sobolev spaces and differential geometry $[\mathrm{Hb}]$ 
and a flurry of recent activity on Sobolev spaces on metric spaces [Hn, HK].

We will not be able to investigate the most of these generalizations here.

\begin{theorem}
  $W^{m,p}(\Omega)$ is a Banach space.
\end{theorem}

\begin{proof}
  Let $\{u_n\}$ be a Cauchy sequence in $W^{m,p}(\Omega)$.
  Then $\left\{D^\alpha u\right\}$ is a Cauchy sequence in $L^p(\Omega)$
  for $0 \leq|\alpha| \leq m$.
  Since $L^p(\Omega)$ is complete there exist functions $u$ and $u_\alpha, 0 \leq|\alpha| \leq m$, 
  such that $u_n \rightarrow u$ and $D^\alpha u_n \rightarrow u_\alpha$ in $L^p(\Omega)$ as 
  $n\to\infty$. Now $L^p(\Omega) \subset L_{\loc}^1(\Omega)$ and so $u_n$
  determines a distribution $T_{u_n} \in \mathscr{D}^{\prime}(\Omega)$
  as in Paragraph 1.58. For any $\phi \in \mathscr{D}(\Omega)$ we have
  \[
  \left|T_{u_n}(\phi)-T_u(\phi)\right| \leq \int_{\Omega}|u_n(x)-u(x)| |\phi(x)| \d x
    \leq \|\phi\|_{p'} \| u_n - u \|_p
  \]
  by Hölder's inequality, where $p'$ is the exponent conjugate to $p$.
  Therefore $T_{u_n}(\phi) \rightarrow T_u(\phi)$ for every $\phi \in \mathscr{D}(\Omega)$
  as $n \rightarrow \infty$.
  Similarly, $T_{D^\alpha u_n}(\phi) \rightarrow T_{u_\alpha}(\phi)$
  for every $\phi \in \mathscr{D}(\Omega)$. It follows that
  \[
  T_{u_\alpha}(\phi)=\lim _{n \rightarrow \infty} T_{D^\alpha u_n}(\phi)=\lim _{n \rightarrow \infty}(-1)^{|\alpha|} T_{u_n}\left(D^\alpha \phi\right)=(-1)^{|\alpha|} T_u\left(D^\alpha \phi\right)
  \]
  for every $\phi \in \mathscr{D}(\Omega)$. Thus $u_\alpha=D^\alpha u$
  in the distributional sense on $\Omega$ for $0 \leq|\alpha| \leq m$,
  whence $u \in W^{m,p}(\Omega)$.
  Since $\lim _{n \rightarrow \infty}\left\|u_n-u\right\|_{m,p}=0$,
  the space $W^{m,p}(\Omega)$ is complete.
\end{proof}


\begin{corollary}
  $H^{m, p}(\Omega) \subset W^{m,p}(\Omega)$.
\end{corollary}

\begin{proof}
  Distributional and classical partial derivatives coincide whenever the latter exist and are continuous on $\Omega$. Therefore the set
  \[
  S=\left\{\phi \in C^m(\Omega):\|\phi\|_{m,p}<\infty\right\}
  \]
  is contained in $W^{m,p}(\Omega)$. Since $W^{m,p}(\Omega)$ is complete, the identity operator on $S$ extends to an isometric isomorphism between $H^{m, p}(\Omega)$, the completion of $S$, and the closure of $S$ in $W^{m,p}(\Omega)$. We can identify $H^{m, p}(\Omega)$ with this closure.
\end{proof}


\begin{para}
  Several important properties of the spaces $W^{m,p}(\Omega)$ can be easily obtained by 
  regarding $W^{m,p}(\Omega)$ as a closed subspace of an $L^p$ space on a union of disjoint 
  copies of $\Omega$.
  
  Given integers $n \geq 1$ and $m \geq 0$, let $N \equiv N(n, m)$ be the number of multi-indices 
  $\alpha=\left(\alpha_1, \ldots, \alpha_n\right)$ such that $|\alpha| \leq m$. For each such 
  multi-index $\alpha$ let $\Omega_\alpha$ be a copy of $\Omega$ in a different copy of
  $\mathbb{R}^n$, so that the $N$ domains $\Omega_\alpha$ are \emph{de facto} mutually disjoint.
  Let $\Omega^{(m)}$ be the union of these $N$ domains;
  $\Omega^{(m)}=\bigcup_{|\alpha| \leq m} \Omega_\alpha$.
  Given a function $u$ in $W^{m,p}(\Omega)$, let $U$ be the function on $\Omega^{(m)}$ that 
  coincides with $D^\alpha u$ on $\Omega_\alpha$. It is easy to check that the map $P$ taking $u$ 
  to $U$ is an isometry from $W^{m,p}(\Omega)$ into $L^p(\Omega^{(m)})$.
  Since $W^{m,p}(\Omega)$ is complete, the range $W$ of the isometry $P$ is a closed subspace of 
  $L^p(\Omega^{(m)})$. It follows that $W$ is separable if $1 \leq p<\infty$,
  and is uniformly convex and reflexive if $1<p<\infty$. The same conclusions must therefore hold 
  for $W^{m,p}(\Omega)=P^{-1}(W)$.
\end{para}


\begin{theorem}
  $W^{m,p}(\Omega)$ is separable if $1 \leq p<\infty$, and is uniformly convex and reflexive if 
  $1<p<\infty$. In particular, $W^{m, 2}(\Omega)$ is a separable Hilbert space with inner product
  \[(u, v)_m=\sum_{0 \leq|\alpha| \leq m}\left(D^\alpha u, D^\alpha v\right),\]
  where $(u, v)=\int_{\Omega} u(x) \overline{v(x)} \d x$ is the inner product on $L^2(\Omega)$.
\end{theorem}


\section[Duality and the Spaces $W^{-m,p'}(\Omega)$]%
  {Duality and the Spaces $\bm{W^{-m,p'}(\Omega)}$}


\begin{para}
  In this section we shall take, for fixed $\Omega$, $m$, and $p$, the number $N$,
  the spaces $L^p(\Omega^{(m)})$ and $W$, and the operator $P$ to be specified as in 
  Paragraph~3.5. We also define
  \[
  \langle u, v\rangle = \int_{\Omega} u(x) v(x) \d x
  \]
  for any functions $u$, $v$ for which the right side makes sense.
  For given $p$ let us agree that $p'$ always denotes the conjugate exponent:
  \[
  p' = 
    \begin{cases}
      \infty & \text { if } p=1 \\
      p /(p-1) & \text { if } 1<p<\infty \\
      1 & \text { if } p=\infty
    \end{cases}
  \]
  First we extend the Riesz Representation Theorem to the space $W^{m,p}(\Omega)$.
  Then, we identify the dual of $W_0^{m, p}(\Omega)$ with a subspace of $\mathscr{D}'(\Omega)$. 
  Finally, we show that if $1<p<\infty$, the dual of $W_0^{m, p}(\Omega)$ can also be identified 
  with the completion of $L^{p'}(\Omega)$ with respect to a norm weaker than the
  usual $L^{p'}$ norm.
\end{para}


\begin{para}[The Dual of $\bm{L^p(\Omega^{(m)})}$]
  To every $L\in\left(L^p(\Omega^{(m)})\right)^{\prime}$, where $1 \leq p<\infty$, 
  there corresponds a unique $v \in L^{p'}(\Omega^{(m)})$ such that for every 
  $u \in L^p(\Omega^{(m)})$,
  \[
   L(u)=\int_{\Omega^{(m)}} u(x) v(x) d x=\sum_{|\alpha| \leq m} \int_{\Omega_\alpha} u_\alpha(x)v_\alpha(x) d x=\sum_{|\alpha| \leq m}\left\langle u_\alpha, v_\alpha\right\rangle,
  \]
  where $u_\alpha$ and $v_\alpha$ are the restrictions of $u$ and $v$, respectively, to $\Omega_\alpha$.
  Moreover, $\bigl\|L ;\left(L^p(\Omega^{(m)})\right)^{\prime}\bigr\|
    = \bigl\|v ; L^{p'}(\Omega^{(m)})\bigr\|$.
  Thus $\left(L^p(\Omega^{(m)})\right)^{\prime} \equiv L^{p'}(\Omega^{(m)})$.
  This is valid because $L^p(\Omega^{(m)})$ is,
  after all, an $L^p$ space, albeit one defined on an unusual domain.
\end{para}

\begin{theorem}[The Dual of $\bm{W^{m,p}(\Omega)}$]
  Let $1 \leq p<\infty$. For every $L \in\left(W^{m,p}(\Omega)\right)^{\prime}$ there exist 
  elements $v \in L^{p'}(\Omega^{(m)})$ such that if the restriction of $v$ to 
  $\Omega_\alpha$ is $v_\alpha$, we have for all $u \in W^{m,p}(\Omega)$
  \begin{equation}\label{eq:3.3}
  L(u)=\sum_{0 \leq|\alpha| \leq m}\left\langle D^\alpha u, v_\alpha\right\rangle .
  \end{equation}
  Moreover
  \begin{equation}\label{eq:3.4}
    \|L\|_{(W^{m,p}(\Omega))'} = \inf \|v\|_{L^{p'}(\Omega^{(m)})}
    = \min \|v\|_{L^{p'}(\omega^{(m)})},
  \end{equation}
  the infimum being taken over, and \emph{attained} on the set of all $v\in L^{p'}(\Omega^{(m)})$
  for which (3) holds for every $u\in W^{m,p}(\Omega)$.

  If $1<p<\infty$, the element $v \in L^{p'}(\Omega^{(m)})$ satisfying (3) and (4) is unique.
\end{theorem}


\begin{proof}
  A linear functional $L^*$ is defined as follows on the range $W$ of the operator $P$
  defined in Paragraph~3.5:
  \[
  L^*(Pu) = L(u), \quad u \in W^{m,p}(\Omega) .
  \]
  Since $P$ is an isometric isomorphism, $L^* \in W'$ and
  \[
  \left\|L^* ; W'\right\|=\left\|L ;\left(W^{m,p}(\Omega)\right)^{\prime}\right\| .
  \]
  By the Hahn-Banach Theorem~1.13 there exists a norm preserving extension $\hat{L}$ of $L^*$
  to all of $L^p(\Omega^{(m)})$, and, as observed in Paragraph~3.8
  there exists $v \in L^{p'}(\Omega^{(m)})$ such that if $u \in L^p(\Omega^{(m)})$, then
  \[
  \hat{L}(u) = \sum_{0 \leq|\alpha| \leq m}\left\langle u_\alpha, v_\alpha\right\rangle .
  \]
  Thus, for $u \in W^{m,p}(\Omega)$ we obtain
  \[
  L(u)=L^*(P u)=\hat{L}(P u)=\sum_{0 \leq|\alpha| \leq m}\left\langle D^\alpha u, v_\alpha\right\rangle
  \]
  Moreover,
  \[
  \left\|L ;\left(W^{m,p}(\Omega)\right)^{\prime}\right\|
    = \left\|L^* ; W'\right\|
    = \left\|\hat{L} ;\left(L^p(\Omega^{(m)})\right)^{\prime}\right\|
    = \left\|v ; L^{p'}(\Omega^{(m)})\right\| .
  \]
  Now (4) must hold because any element $v \in L^{p'}(\Omega^{(m)})$ for which (3) holds for every $u \in W^{m,p}(\Omega)$ corresponds to an extension $L$ of $L^*$ and so will have norm $\left\|v ; L^{p'}(\Omega^{(m)})\right\|$ not less than $\left\|L ;\left(W^{m,p}(\Omega)\right)^{\prime}\right\|$.
  The uniqueness of $v$ if $1<p<\infty$ follows from the uniform convexity of $L^p(\Omega^{(m)})$ and $L^{p'}(\Omega^{(m)})$ by an argument similar to that in Lemma 2.43.
\end{proof}


\begin{para}
  If $1 \leq p<\infty$ every element $L$ of $\left(W^{m,p}(\Omega)\right)^{\prime}$ is an 
  extension to $W^{m,p}(\Omega)$ of a distribution $T \in \mathscr{D}^{\prime}(\Omega)$. To see what form this distribution takes, suppose $L$ is given by (3) for some $v \in L^{p'}(\Omega^{(m)})$ and define $T$ and $T_{v_\alpha}$ on $\mathscr{D}(\Omega)$ by
  \begin{equation}\label{eq:3.5}
    T=\sum_{0 \leq|\alpha| \leq m}(-1)^{|\alpha|} D^\alpha T_{v_\alpha}, \quad T_{v_\alpha}(\phi)=\left\langle\phi, v_\alpha\right\rangle . \quad 0 \leq|\alpha| \leq m,
  \end{equation}
  For every $\phi \in \mathscr{D}(\Omega) \subset W^{m,p}(\Omega)$ we have $T(\phi)=\sum_{0 \leq|\alpha| \leq m} T_{v_\alpha}\left(D^\alpha \phi\right)=L(\phi)$ so that $L$ is clearly an extension of $T$. Moreover, by (4)
  \[
  \left\|L ;\left(W^{m,p}(\Omega)\right)^{\prime}\right\|=\min \left\{\left\|v ; L^{p'}(\Omega^{(m)})\right\|: L \text { extends } T \text { given by }(5)\right\} .
  \]
\end{para}

These remarks also hold for $L \in\left(W_0^{m, p}(\Omega)\right)^{\prime}$ since any such functional possesses a norm-preserving extension to $W^{m,p}(\Omega)$.

\begin{para}
  Now suppose $T$ is any element of $\mathscr{D}^{\prime}(\Omega)$ having the form (5) for some $v \in L^{p'}(\Omega^{(m)})$, where $1 \leq p' \leq \infty$. Then $T$ possesses (possibly non-unique) continuous extensions to $W^{m,p}(\Omega)$. However, $T$ possesses a unique continuous extension to $W_0^{m, p}(\Omega)$. To see this, for $u \in W_0^{m, p}(\Omega)$ let $\left\{\phi_n\right\}$ be a sequence in $C_0^{\infty}(\Omega)=\mathscr{D}(\Omega)$ converging to $u$ in norm in $W_0^{m, p}(\Omega)$. Then
  \[
  \begin{aligned}
  \left|T\left(\phi_k\right)-T\left(\phi_n\right)\right| & \leq \sum_{0 \leq|\alpha| \leq m}\left|T_{v_\alpha}\left(D^\alpha \phi_k-D^\alpha \phi_n\right)\right| \\
  & \leq \sum_{0 \leq|\alpha| \leq m}\left\|D^\alpha\left(\phi_k-\phi_n\right)\right\|_p\left\|v_\alpha\right\|_{p'} \\
  & \leq\left\|\phi_k-\phi_n\right\|_{m,p}\left\|v ; L^{p'}(\Omega^{(m)})\right\| \rightarrow 0 \quad \text { as } k, n \rightarrow \infty .
  \end{aligned}
  \]
  Thus $\left\{T\left(\phi_n\right)\right\}$ is a Cauchy sequence in $\mathbb{C}$ and so converges to a limit that we can denote by $L(u)$ since it is clear that if also $\left\{\psi_n\right\} \subset \mathscr{D}(\Omega)$ and $\left\|\psi_n-u\right\|_{m,p} \rightarrow 0$, then $T\left(\phi_n\right)-T\left(\psi_n\right) \rightarrow 0$ as $n \rightarrow \infty$. The functional $L$ thus defined is linear and belongs to $\left(W_0^{m, p}(\Omega)\right)^{\prime}$, for if $u=\lim _{n \rightarrow \infty} \phi_n$ as above, then
  \[
  |L(u)|=\lim _{n \rightarrow \infty}\left|T\left(\phi_n\right)\right| \leq \lim _{n \rightarrow \infty}\left\|\phi_n\right\|_{m,p}\left\|v ; L^{p'}(\Omega^{(m)})\right\|=\|u\|_{m,p}\left\|v ; L^{p'}(\Omega^{(m)})\right\| .
  \]
\end{para}

We have therefore proved the following theorem.


\begin{theorem}[The Normed Dual of $\bm{W_0^{m, p}(\Omega)}$]
  If $1 \leq p<\infty$, $p'$ is the exponent conjugate to $p$, and $m \geq 1$, the dual space $\left(W_0^{m, p}(\Omega)\right)^{\prime}$ is isometrically isomorphic to the Banach space $W^{-m, p'}(\Omega)$ consisting of those distributions $T \in \mathscr{D}^{\prime}(\Omega)$ that satisfy (5) and having norm
  \[
  \|T\|=\min \left\{\left\|v ; L^{p'}(\Omega^{(m)})\right\|: v \text { satisfies }(5)\right\} .
  \]
\end{theorem}

The completeness of this space is a consequence of the isometric isomorphism. Evidently $W^{-m, p'}(\Omega)$ is separable and reflexive if $1<p<\infty$.

When $W_0^{m, p}(\Omega)$ is a proper subset of $W^{m,p}(\Omega)$, continuous linear functionals on $W^{m,p}(\Omega)$ are not fully determined by their restrictions to $C_0(\Omega)$, and so are not determined by distributions $T$ given by (5).


\begin{para}[The $\bm{(-m, p')}$ norm on $\bm{L^{p'}(\Omega)}$]
  There is another way of characterizing the dual of $W_0^{m, p}(\Omega)$ if $1<p<\infty$.
  Each element $v \in L^{p'}(\Omega)$ determines an element $L_v$
  of $\left(W_0^{m, p}(\Omega)\right)^{\prime}$ by means of $L_v(u)=\langle u, v\rangle$,
  because
  \[
  \left|L_v(u)\right|=|\langle u, v\rangle| \leq\|v\|_{p'}\|u\|_p \leq\|v\|_{p'}\|u\|_{m,p} .
  \]
  We define the $\left(-m, p'\right)$-norm of $v \in L^{p'}(\Omega)$ to be the norm of $L_v$, that is
  \[
  \|v\|_{-m, p'}
    = \left\|L_v ;\left(W_0^{m, p}(\Omega)\right)^{\prime}\right\|
    = \sup_{u \in W_0^{m, p}(\Omega),\|u\|_{m,p} \leq 1}|\langle u, v\rangle| .
  \]
  Clearly $\|v\|_{-m, p'} \leq\|v\|_{p'}$ and for any $u \in W_0^{m, p}(\Omega)$ and $v \in L^{p'}(\Omega)$ we have
  \begin{equation}\label{eq:3.6}
  |\langle u, v\rangle|=\|u\|_{m,p}\left|\left\langle\frac{u}{\|u\|_{m,p}}, v\right\rangle\right| \leq\|u\|_{m,p}\|v\|_{-m, p'}
  \end{equation}
  which is a generalization of Hölder's inequality.

  Let $V = \{L_v: v \in L^{p'}(\Omega)\}$, which is a vector subspace of
  $\left(W_0^{m, p}(\Omega)\right)^{\prime}$.
  We show that $V$ is dense in $\left(W_0^{m, p}(\Omega)\right)^{\prime}$.
  To this end it is sufficient to show that if
  $F \in\left(W_0^{m, p}(\Omega)\right)^{\prime \prime}$ satisfies $F\left(L_v\right)=0$
  for every $L_v \in V$, then $F=0$ in $\left(W_0^{m, p}(\Omega)\right)^{\prime \prime}$.
  But since $W_0^{m, p}(\Omega)$ is reflexive, there exists $f \in W_0^{m, p}(\Omega)$ corresponding to $F \in\left(W_0^{m, p}(\Omega)\right)^{\prime \prime}$
  such that $\langle f, v\rangle=L_v(f)=F\left(L_v\right)=0$ for every $v \in L^{p'}(\Omega)$.
  But then $f(x)$ must be zero a.e.~in $\Omega$. Hence $f=0$ in $W_0^{m, p}(\Omega)$ and 
  $F=0$ in $(W_0^{m, p}(\Omega))''$.
  
  Let $H^{-m, p'}(\Omega)$ denote the completion of $L^{p'}(\Omega)$
  with respect to the norm $\|\cdot\|_{-m, p'}$. Then we have
  \[
  H^{-m, p'}(\Omega) \equiv\left(W_0^{m, p}(\Omega)\right)^{\prime} \equiv W^{-m, p'}(\Omega) \text {. }
  \]
  In particular, corresponding to each $v \in H^{-m, p'}(\Omega)$, there exists a 
  distribution $T_v \in W^{-m, p'}(\Omega)$ such that
  $T_v(\phi)=\lim _{n \rightarrow \infty}\left\langle\phi, v_n\right\rangle$
  for every $\phi \in \mathscr{D}(\Omega)$ and every sequence
  $\left\{v_n\right\} \subset L^{p'}(\Omega)$ for which
  $\lim _{n \rightarrow \infty}\left\|v_n-v\right\|_{-m, p'}=0$.
  Conversely, any $T \in W^{-m, p'}(\Omega)$ satisfies $T=T_v$ for some such $v$.
  Moreover, by (6), $\left|T_v(\phi)\right| \leq\|\phi\|_{m,p}\|v\|_{-m, p'}$
\end{para}

\begin{para}
  A similar argument to that above shows that the dual space $\left(W^{m,p}(\Omega)\right)^{\prime}$ can be characterized for $1<p<\infty$ as the completion of $L^{p'}(\Omega)$ with respect to the norm
  \[
  \|v\|_{-m, p'}^*
    = \sup _{u \in W^{m,p}(\Omega),\|u\|_{m,p} \leq 1}|\langle u, v\rangle|.
  \]
\end{para}


\section[Approximation by Smooth Functions on $\Omega$]%
  {Approximation by Smooth Functions on $\bm{\Omega}$}

We wish to prove that $\left\{\phi \in C^{\infty}(\Omega):\|\phi\|_{m,p}<\infty\right\}$
is dense in $W^{m,p}(\Omega)$. For this we require the following existence theorem for
infinitely differentiable partitions of unity.

\begin{theorem}[Partitions of Unity]
  Let $A$ be an arbitrary subset of $\mathbb{R}^n$ and let $\mathscr{O}$ be a collection of open 
  sets in $\mathbb{R}^n$ which cover $A$, that is, $A \subset \bigcup_{U \in \mathscr{O}} U$.
  Then there exists a collection $\Psi$ of functions $\psi \in C_0^{\infty}\left(\mathbb{R}^n\right)$ having the following properties:
  \begin{enumerate}[label = (\roman*)]
    \item For every $\psi \in \Psi$ and every $x \in \mathbb{R}^n$, $0 \leq \psi(x) \leq 1$.
    \item If $K \subset\subset A$, all but finitely many $\psi \in \Psi$ vanish identically on $K$.
    \item For every $\psi \in \Psi$ there exists $U \in \mathscr{O}$ such that $\supp(\psi) \subset U$.
    \item For every $x \in A$, we have $\sum_{\psi \in \Psi} \psi(x)=1$.
  \end{enumerate}
  Such a collection $\Psi$ is called a $C^{\infty}$-partition of unity for $A$ subordinate to $\mathscr{O}$.
\end{theorem}

\begin{proof}
  Since the proof can be found in many texts, we give only an outline of it.
  First suppose that $A$ is compact. Then there is a finite collection of sets in $\mathscr{O}$ 
  that cover $A$, say $A \subset \bigcup_{j=1}^N U_j$.
  Compact sets $K_1 \subset U_1, \ldots, K_N \subset U_N$ can then be constructed so that
  $A \subset \bigcup_{j=1}^N K_j$. For each $j$ a nonnegative-valued function
  $\phi_j \in C_0^{\infty}\left(U_j\right)$ can be found such that $\phi_j(x)>0$
  for $x \in K_j$. A function $\phi$ in $C^{\infty}\left(\mathbb{R}^n\right)$ can then be constructed so that $\phi(x)>0$ on $\mathbb{R}^n$ and $\phi(x)=\sum_{j=1}^N \phi_j(x)$
  for $x \in A$. Now $\Psi=\left\{\psi_n: \psi_j(x)=\phi_j(x) / \phi(x), 1 \leq j \leq N\right\}$ 
  has the required properties.
  If $A$ is an arbitrary open set. Then $A=\bigcup_{j=1}^{\infty} A_j$, where
  \[
  A_j=\{x \in A:|x| \leq j \text { and } \dist(x, \partial A) \geq 1 / j\}
  \]
  is compact. Taking $A_0=A_{-1}=\emptyset$, for each $j \geq 1$ the collection
  covers $A_j$ and so there exists a finite $C^{\infty}$-partition of unity $\Psi_j$ for $A_j$ subordinate to $\mathscr{O}_j$. The sum $\sigma(x)=\sum_{j=1}^{\infty} \sum_{\phi \in \Psi_j} \phi(x)$ involves only finitely many nonzero terms at each $x \in A$. The collection $\Psi=\{\psi: \psi(x)=\phi(x) / \sigma(x)$ for some $\phi$ in some $\Psi_j$ if $x \in A, \psi(x)=0$ if $\left.x \notin A\right\}$ has the prescribed properties.
  
  Finally, if $A$ is arbitrary, then $A \subset B$ where $B$ is the union of all $U \in \mathscr{O}$ and is an open set. Any partition of unity for $B$ will do for $A$ as well.
\end{proof}

\begin{lemma}[Mollification in $\bm{W^{m,p}(\Omega)}$]
  Let $J_\varepsilon$ be defined as in Paragraph~2.28 and let $1 \leq p<\infty$ and $u \in W^{m,p}(\Omega)$. If $\Omega'$ is a subdomain with compact closure in $\Omega$, then $\lim _{\varepsilon \rightarrow 0+} J_\varepsilon * u=u$ in $W^{m,p}\left(\Omega'\right)$.
\end{lemma}

\begin{proof}
  Let $\varepsilon<\dist (\Omega', \partial\Omega)$ 
  and $\tilde{u}$ be the zero extension of $u$ outside $\Omega$.
  If $\phi \in \mathscr{D}\left(\Omega'\right)$
  \[
  \begin{aligned}
    \int_{\Omega'} J_\varepsilon * u(x) D^\alpha \phi(x) d x
    & = \int_{\mathbb{R}^n} \int_{\mathbb{R}^n} \tilde{u}(x-y) J_\varepsilon(y) 
        D^\alpha \phi(x) d x d y \\
    & = (-1)^{|\alpha|} \int_{\mathbb{R}^n} \int_{\Omega'}
        D_x^\alpha u(x-y) J_\varepsilon(y) \phi(x) d x d y \\
    & = (-1)^{|\alpha|} \int_{\Omega'} J_\varepsilon * D^\alpha u(x) \phi(x) d x .
  \end{aligned}
  \]
  Thus $D^\alpha J_\varepsilon * u=J_\varepsilon * D^\alpha u$ in the distributional sense
  in $\Omega'$. Since $D^\alpha u \in L^p(\Omega)$ for $0 \leq|\alpha| \leq m$
  we have by Theorem~2.29(c)
  \[
  \lim _{\varepsilon \rightarrow 0+}\left\|D^\alpha J_\varepsilon * u-D^\alpha u\right\|_{p, \Omega'}=\lim _{\varepsilon \rightarrow 0+}\left\|J_\varepsilon * D^\alpha u-D^\alpha u\right\|_{p, \Omega'}=0 .
  \]
  Thus $\lim _{\varepsilon \rightarrow 0+}\left\|J_\varepsilon u-u\right\|_{m, p, \Omega'}=0$.
\end{proof}


\begin{theorem}[$\bm{H} = \bm{W}$]
  (See [MS].) If $1 \leq p<\infty$, then
  \[
  H^{m, p}(\Omega)=W^{m,p}(\Omega)
  \]
\end{theorem}

\begin{proof}
  By Corollary~3.4 it is sufficient to show that $W^{m,p}(\Omega) \subset H^{m, p}(\Omega)$,
  that is, that $\left\{\phi \in C^m(\Omega):\|\phi\|_{m,p}<\infty\right\}$
  is dense in $W^{m,p}(\Omega)$. If $u \in W^{m,p}(\Omega)$ and $\varepsilon>0$,
  we in fact show that there exists $\phi \in C^{\infty}(\Omega)$
  such that $\|\phi-u\|_{m,p}<\varepsilon$, so that $C^{\infty}(\Omega)$ is dense
  in $W^{m,p}(\Omega)$. For $k=1,2, \ldots$ let
  \[
  \Omega_k=\{x \in \Omega:|x|<k \text { and } \dist(x, \partial\Omega)>1/k\}
  \]
  and let $\Omega_0=\Omega_{-1}=\emptyset$, the empty set. Then
  \[
  \mathscr{O} = \left\{U_k: U_k=\Omega_{k+1} \cap\left(\overline{\Omega_{k-1}}\right)^c, k=1,2, \ldots\right\}
  \]
  is a collection of open subsets of $\Omega$ that covers $\Omega$.
  Let $\Psi$ be a $C^{\infty}$-partition or unity for $\Omega$ subordinate to $\mathscr{O}$.
  Let $\psi_k$ denote the sum of the finitely many functions $\psi \in \Psi$ whose supports are 
  contained in $U_k$.
  Then $\psi_k \in C_0^{\infty}\left(U_k\right)$ and $\sum_{k=1}^{\infty} \psi_k(x)=1$ on $\Omega$.
  
  If $0<\varepsilon<1 /(k+1)(k+2)$, then $J_\varepsilon *\left(\psi_k u\right)$ has support in the 
  intersection $V_k=\Omega_{k+2} \cap\left(\Omega_{k-2}\right)^c \subset\subset \Omega$.
  Since $\psi_k u \in W^{m,p}(\Omega)$ we may choose $\varepsilon_k$,
  satisfying $0<\varepsilon_k<1 /(k+1)(k+2)$, such that
  \[
    \left\|J_{\varepsilon_k} *\left(\psi_k u\right)-\psi_k u\right\|_{m, p, \Omega}=\left\|J_{\varepsilon_k} *\left(\psi_k u\right)-\psi_k u\right\|_{m, p, V_k}<\varepsilon /(2^k) .
  \]
  Let $\phi=\sum_{k=1}^{\infty} J_{\varepsilon_k} *\left(\psi_k u\right)$.
  On any $\Omega' \subset\subset \Omega$ only finitely many terms in the sum can be nonzero.
  Thus $\phi \in C^{\infty}(\Omega)$. For $x \in \Omega_k$, we have
  \[
    u(x)=\sum_{j=1}^{k+2} \psi_j(x) u(x), \quad \text { and } \quad \phi(x)=\sum_{j=1}^{k+2} J_{\varepsilon_j} *\left(\psi_j u\right)(x)
  \]
  Thus
  \[
    \|u-\phi\|_{m, p, \Omega_k} \leq \sum_{j=1}^{k+2}\left\|J_{\varepsilon_j} *\left(\psi_j u\right)-\psi_j u\right\|_{m, p, \Omega}<\varepsilon .
  \]
  By the monotone convergence theorem~1.48, $\|u-\phi\|_{m, p, \Omega}<\varepsilon$.
\end{proof}


\begin{example}
  Theorem 3.17 can not be extended to the case $p=\infty$.
  For instance, if $\Omega=\left\{x \in \mathbb{R}:-1<x<1\right.$,
  and $u(x)=|x|$, then $u^{\prime}(x)=x /|x|$ for $x \neq 0$ and so $u \in W^{1, \infty}(\Omega)$. 
  But $u \notin H^{1,\infty}(\Omega)$. In fact, if $0<\varepsilon<1 / 2$,
  there exists no function $\phi \in C^1(\Omega)$ such that $\left\|\phi^{\prime}-u^{\prime}\right\|_{\infty}<\varepsilon$.
\end{example}


\section{Approximation by Smooth Functions on $\mathbb{R}^n$}


\begin{para}
  Having shown that an element of $W^{m,p}(\Omega)$ can always be approximated by functions 
  smooth on $\Omega$ we now ask whether the approximation can in fact be done with bounded 
  functions having bounded derivatives of all orders, or at least of all orders up to and 
  including at least $m$. That is, we are asking whether, for any values of $k \geq m$, the space 
  $C^k(\overline{\Omega})$ is dense in $W^{m,p}(\Omega)$. The following example shows that the answer 
  may be negative.
\end{para}


\begin{example}
  Let $\Omega=\left\{(x, y) \in \mathbb{R}^2: 0<|x|<1,0<y<1\right\}$. Then the function defined on $\Omega$ by
  \[
  u(x, y)= \begin{cases}1 & \text { if } x>0 \\ 0 & \text { if } x<0\end{cases}
  \]
  evidently belongs to $W^{1, p}(\Omega)$. However, if $\varepsilon>0$ is sufficiently small, 
  there can exist no $\phi \in C^1(\overline{\Omega})$ such that
  $\|u-\phi\|_{1, p, \Omega}<\varepsilon$. To see this, suppose there exists such a $\phi$.
  If $L=\{(x, y):-1 \leq x \leq 0,0 \leq y \leq 1\}$ and $R=\{(x, y): 0 \leq x \leq 1$, $0 \leq y \leq 1\}$, then $\overline{\Omega}=L \cup R$.
  We have $\|\phi\|_{1, L} \leq\|\phi\|_{p, L}<\varepsilon$ and
  similarly $\|1-\phi\|_{1, R}<\varepsilon$ from which we obtain $\|\phi\|_{1, R}>1-\varepsilon$. 
  If
  \[
  \Phi(x)=\int_0^1 \phi(x, y) d y,
  \]
  then there exist $a$ and $b$ with $-1 \leq a<0$ and $0<b \leq 1$ such
  that $\Phi(a)<\varepsilon$ and $\Phi(b)>1-\varepsilon$. If $0<\varepsilon<1 / 2$, then
  \[
  \begin{aligned}
  1-2 \varepsilon<\Phi(b)-\Phi(a)=\int_a^b \Phi^{\prime}(x) d x & \leq \int_{\overline{\Omega}}\left|D_x \phi(x, y)\right| d x d y \\
  & \leq 2^{1 / p^{\prime}}\left\|D_x \phi\right\|_{p, \Omega}<2^{1 / p^{\prime}} \varepsilon
  \end{aligned}
  \]
  Thus $1<\varepsilon\left(2+2^{1 / p^{\prime}}\right)$, which is not possible for small $\varepsilon$.
  The difficulty with the domain in this example is that it lies on both sides of part of its 
  boundary, namely the line segment $x=0,0 \leq y \leq 1$. We now formulate a condition on a 
  domain $\Omega$ that prevents this from happening and guarantees that for any $k$
  and $m, C^k(\overline{\Omega})$ is dense in $W^{m,p}(\Omega)$ provided $1 \leq p<\infty$.
\end{example}


\begin{para}[The Segment Condition]
  We say that a domain $\Omega$ satisfies the segment condition if every $x \in\partial\Omega$
  has a neighbourhood $U_x$ and a nonzero vector $y_x$ such that if $z \in \overline{\Omega} \cap U_x$, 
  then $z+t y_x \in \Omega$ for $0<t<1$.

  If nonempty, the boundary of a domain satisfying this condition must be $(n-1)$ dimensional,
  and the domain cannot lie on both sides of any part of its boundary.
\end{para}


\begin{theorem}
  If $\Omega$ satisfies the segment condition, then the set of restrictions to $\Omega$
  of functions in $C_0^{\infty}\left(\mathbb{R}^n\right)$ is dense in $W^{m,p}(\Omega)$
  for $1 \leq p<\infty$.
\end{theorem}

\begin{proof}
  Let $f$ be a fixed function in $C_0^{\infty}\left(\mathbb{R}^n\right)$ satisfying
  \begin{enumerate}[label = (\roman*)]
    \item $f(x)=1$ if $|x| \leq 1$,
    \item $f(x)=0$ if $|x| \geq 2$,
    \item $\left|D^\alpha f(x)\right| \leq M$ (constant) for all $x$ and $0 \leq|\alpha| \leq m$.
  \end{enumerate}
  For $\varepsilon>0$ let $f_\varepsilon(x)=f(\varepsilon x)$.
  Then $f_\varepsilon(x)=1$ for $|x| \leq 1 / \varepsilon$
  and also $\left|D^\alpha f_\varepsilon(x)\right| \leq M \varepsilon^{|\alpha|} \leq M$
  if $\varepsilon \leq 1$. If $u \in W^{m,p}(\Omega)$, then $u_\varepsilon=f_\varepsilon u$
  belongs to $W^{m,p}(\Omega)$ and has bounded support.
  For $0<\varepsilon \leq 1$ and $|\alpha| \leq m$
  \[
  \left|D^\alpha u_\varepsilon(x)\right|=\left|\sum_{\beta \leq \alpha}\left(\begin{array}{c}
  \alpha \\
  \beta
  \end{array}\right) D^\beta u(x) D^{\alpha-\beta} f_\varepsilon(x)\right| \leq M \sum_{\beta \leq \alpha}\left(\begin{array}{c}
  \alpha \\
  \beta
  \end{array}\right)\left|D^\beta u(x)\right| .
  \]
  Therefore, setting $\Omega_\varepsilon=\{x \in \Omega:|x|>1 / \varepsilon\}$, we have
  \[
  \begin{aligned}
  \left\|u-u_\varepsilon\right\|_{m, p, \Omega} & =\left\|u-u_\varepsilon\right\|_{m, p, \Omega_\varepsilon} \\
  & \leq\|u\|_{m, p, \Omega_\varepsilon}+\left\|u_\varepsilon\right\|_{m, p, \Omega_\varepsilon} \leq \text { const }\|u\|_{m, p, \Omega_\varepsilon} .
  \end{aligned}
  \]
  The right side approaches zero as $\varepsilon \rightarrow 0+$.
  Thus any $u \in W^{m,p}(\Omega)$ can be approximated in that space by functions
  with bounded supports.
  We now, therefore, assume that $K=\{x \in \Omega: u(x) \neq 0\}$ is bounded.
  The set $F=\overline{K}-\left(\bigcup_{x \in\partial\Omega} U_x\right)$
  is thus compact and contained in $\Omega$, $\left\{U_x\right\}$
  being the collection of open sets 
  referred to in the definition of the segment condition.
  There exists an open set $U_0$ such that $F \subset\subset U_0 \subset\subset \Omega$.
  Since $\overline{K}$ is compact, there exists finitely many of the sets $U_x$,
  let us rename them $U_1, \ldots, U_k$,
  such that $\overline{K} \subset U_0 \cup U_1 \cup \ldots \cup U_k$.
  Moreover, there are other open sets $V_0, V_1, \ldots, V_k$
  such that $V_j \subset\subset U_j$ for $0 \leq j \leq k$
  but still $\overline{K} \subset V_0 \cup V_1 \cup \cdots \cup V_k$.
  Let $\Psi$ be a $C^{\infty}$-partition of unity subordinate to $\left\{V_j: 0 \leq j \leq k\right\}$, and let $\psi_j$ be the sum of the finitely many functions $\psi \in \Psi$ whose supports lie in $V_j$. Let $u_j=\psi_j u$. Suppose that for each $j$ we can find $\phi_j \in C_0^{\infty}\left(\mathbb{R}^n\right)$ such that
  \begin{equation}\label{eq:3.7}
    \left\|u_j-\phi_j\right\|_{m, p, \Omega}<\varepsilon /(k+1) .
  \end{equation}
  Then, putting $\phi=\sum_{j=0}^k \phi_j$, we would obtain
  \[
  \|u-\phi\|_{m, p, \Omega} \leq \sum_{j=0}^k\left\|u_j-\phi_j\right\|_{m, p, \Omega}<\varepsilon .
  \]
  A function $\phi_0 \in C_0^{\infty}\left(\mathbb{R}^n\right)$ satisfying (7) for $j=0$ can be 
  found via Lemma 3.16 since $\supp\left(u_0\right) \subset V_0 \subset\subset \Omega$.
  It remains, therefore, to find $\phi_j$ satisfying (7) for $1 \leq j \leq k$.
  For fixed such $j$ we extend $u_j$ to be identically zero outside $\Omega$.
  Thus $u_j \in W^{m,p}\left(\mathbb{R}^n-\Gamma\right)$,
  where $\Gamma=\overline{V_j} \cap\partial\Omega$.
  Let $y$ be the nonzero vector associated with the set $U_j$ in the definition of the segment 
  condition. (See Fig.~1.) Let $\Gamma_t=\{x-t y: x \in \Gamma\}$, where $t$ is so chosen that
  \[
  0<t<\min \left\{1, \dist\left(V_j, \mathbb{R}^n-U_j\right) /|y|\right\}.
  \]
  Then $\Gamma_t \subset U_j$ and $\Gamma_t \cap \overline{\Omega}$ is empty by the segment condition. 
  Let us define $u_{j, t}(x)=u_j(x+t y)$.
  Then $u_{j, t} \in W^{m,p}\left(\mathbb{R}^n-\Gamma_t\right)$.
  Translation is continuous in $L^p(\Omega)$ (see the proof of Theorem~2.32)
  so $D^\alpha u_{j, t} \rightarrow D^\alpha u_j$ in $L^p(\Omega)$
  as $t \rightarrow 0+$ for $|\alpha| \leq m$.
  Thus $u_{j, t} \rightarrow u_j$ in $W^{m,p}(\Omega)$ as $t \rightarrow 0+$,
  and so it is sufficient to find $\phi_j \in C_0^{\infty}\left(\mathbb{R}^n\right)$
  such that $\left\|u_{j, t}-\phi_j\right\|_{m,p}$ is sufficiently small.
  However, $\Omega \cap U_j \subset\subset \mathbb{R}^n-\Gamma_t$,
  and so by Lemma~3.16 we can take $\phi_j=J_\delta * u_{j, t}$ for suitably small $\delta>0$. 
  This completes the proof.
\end{proof}


\begin{corollary}
  $\quad W_0^{m, p}\left(\mathbb{R}^n\right)=W^{m,p}\left(\mathbb{R}^n\right)$.
\end{corollary}


\section[Approximation by Functions in $C_0^\infty (\Omega)$]%
  {Approximation by Functions in $\bm{C_0^\infty (\Omega)}$}


\begin{para}
  Corollary 3.23 suggests the question: For what domains $\Omega$ is it true
  that $W^{m,p}(\Omega)=W_0^{m, p}(\Omega)$, that is, when is $C_0^{\infty}(\Omega)$
  dense in $W^{m,p}(\Omega)$ ? A partial answer to this question can be formulated in terms of 
  the nature of the distributions belonging to $W^{-m, p^{\prime}}\left(\mathbb{R}^n\right)$.
  The approach below is due to Lions [Lj]. Throughout this discussion we assume $1<p<\infty$
  and $p^{\prime}$ is the conjugate exponent $p^{\prime}=p /(p-1)$.
\end{para}

\begin{para}[$\bm{(m, p')}$-Polar sets]
  Let $F$ be a closed subset of $\mathbb{R}^n$.
  A distribution $T \in \mathscr{D}^{\prime}\left(\mathbb{R}^n\right)$ is said to have support
  in $F(\supp(T) \subset F)$ provided that $T(\phi)=0$ for every
  $\phi \in \mathscr{D}\left(\mathbb{R}^n-F\right)$.
  We say that the closed set $F$ is $\left(m, p^{\prime}\right)$-polar if the only
  distribution $T \in W^{-m, p^{\prime}}\left(\mathbb{R}^n\right)$ having support in $F$ is
  the zero distribution $T=0$. If $F$ has positive measure,
  it cannot be $\left(m, p^{\prime}\right)$-polar because the characteristic function of any 
  compact subset of $F$ having positive measure belongs
  to $L^{p^{\prime}}\left(\mathbb{R}^n\right)$ and hence
  to $W^{-m, p^{\prime}}\left(\mathbb{R}^n\right)$.
\end{para}

We shall show later that if $m p>n$,
then $W^{m,p}\left(\mathbb{R}^n\right) \rightarrow C\left(\mathbb{R}^n\right)$ in the sense that 
if $u \in W^{m,p}\left(\mathbb{R}^n\right)$, then there exists $v \in C\left(\mathbb{R}^n\right)$ 
such that $u(x)=v(x)$ a.e. in $\mathbb{R}^n$ and
\[
|v(x)| \leq \text { const }\|u\|_{m,p}
\]
the constant being independent of $x$ and $u$. It follows that the Dirac distribution $\delta_x$ 
given by $\delta_x(\phi)=\phi(x)$ belongs to
$\left(W^{m,p}\left(\mathbb{R}^n\right)\right)^{\prime}=\left(W_0^{m, p}\left(\mathbb{R}^n\right)\right)^{\prime}=W^{-m, p^{\prime}}\left(\mathbb{R}^n\right)$.
Hence, if $m p>n$ a set $F$ cannot be $\left(m, p^{\prime}\right)$-polar unless it is empty.
Since $W^{m+1, p}(\Omega) \rightarrow W^{m,p}(\Omega)$ any bounded linear functional on the 
latter space is also bounded on the former.
Thus $W^{-m, p^{\prime}}(\Omega) \subset W^{-m-1, p^{\prime}}(\Omega)$ and, in particular,
any $\left(m+1, p^{\prime}\right)$-polar set is also $\left(m, p^{\prime}\right)$-polar.
The converse is, of course, generally not true.


\begin{para}[Zero Extensions]
  If function $u$ is defined on $\Omega$ let $\tilde{u}$ denote the zero extension of $u$
  to the complement $\Omega^c$ of $\Omega$ in $\mathbb{R}^n$ :
  \[
  \tilde{u}(x)= \begin{cases}u(x) & \text { if } x \in \Omega, \\ 0 & \text { if } x \in \Omega^c .\end{cases}
  \]
  The following lemma shows that the mapping $u \mapsto \tilde{u}$
  maps $W_0^{m, p}(\Omega)$ (isometrically) into $W^{m,p}\left(\mathbb{R}^n\right)$.
\end{para}

\begin{lemma}
  Let $u \in W_0^{m, p}(\Omega)$. If $|\alpha| \leq m$,
  then $D^\alpha \tilde{u}=\widetilde{D^\alpha u}$ in the distributional sense in $\mathbb{R}^n$. 
  Hence $\tilde{u} \in W^{m,p}\left(\mathbb{R}^n\right)$.
\end{lemma}

\begin{proof}
  Let $\left\{\phi_j\right\}$ be a sequence in $C_0^{\infty}(\Omega)$ converging to $u$ in $W_0^{m, p}(\Omega)$. If $\psi \in \mathscr{D}\left(\mathbb{R}^n\right)$, then for $|\alpha| \leq m$
  \[
  \begin{aligned}
  (-1)^{|\alpha|} \int_{\mathbb{R}^n} \tilde{u}(x) D^\alpha \psi(x) d x & =(-1)^{|\alpha|} \int_{\Omega} u(x) D^\alpha \psi(x) d x \\
  & =\lim _{j \rightarrow \infty}(-1)^{|\alpha|} \int_{\Omega} \phi_j(x) D^\alpha \psi(x) d x \\
  & =\lim _{j \rightarrow \infty} \int_{\Omega} D^\alpha \phi_j(x) \psi(x) d x \\
  & =\int_{\mathbb{R}^n} \widetilde{D^\alpha u(x) \psi(x) d x}
  \end{aligned}
  \]
  Thus $D^\alpha \tilde{u}=\widetilde{D^\alpha u}$ in the distributional sense
  in $\mathbb{R}^n$ and these locally integrable functions are equal a.e. in $\mathbb{R}^n$.
  It follows that $\|\tilde{u}\|_{m, p, \mathbb{R}^n}=\|u\|_{m, p, \Omega}$.
\end{proof}

We can now give a necessary and sufficient condition that the mapping $u \mapsto \tilde{u}$ 
carries $W_0^{m, p}(\Omega)$ onto $W^{m,p}\left(\mathbb{R}^n\right)$.

\begin{theorem}
  $C_0^{\infty}(\Omega)$ is dense in $W^{m,p}\left(\mathbb{R}^n\right)$ if and only if the 
  complement $\Omega^c$ of $\Omega$ is $\left(m, p^{\prime}\right)$-polar.
\end{theorem}

\begin{proof}
  First suppose $C_0^{\infty}(\Omega)$ is dense in $W^{m,p}\left(\mathbb{R}^n\right)$. Let $T \in W^{-m, p^{\prime}}\left(\mathbb{R}^n\right)$ have support in $\Omega^c$. If $u \in W^{m,p}\left(\mathbb{R}^n\right)$, then there exists a sequence $\left\{\phi_j\right\} \subset C_0^{\infty}(\Omega)$ converging to $u$ in $W^{m,p}\left(\mathbb{R}^n\right)$. Hence $T(u)=\lim _{j \rightarrow \infty} T\left(\phi_j\right)=0$ and so $T=0$. Thus $\Omega^c$ is $\left(m, p^{\prime}\right)$-polar.
  Conversely, suppose that $C_0^{\infty}(\Omega)$ is not dense in $W^{m,p}\left(\mathbb{R}^n\right)$. Then there exists $u \in W^{m,p}\left(\mathbb{R}^n\right)$ and a constant $k>0$ such that for all $\phi \in C_0^{\infty}(\Omega)$ we have $\|u-\phi\|_{m, p, \mathbb{R}^n} \geq k$. The Hahn-Banach theorem 1.13 can be used to show that there exists $T \in W^{-m, p^{\prime}}\left(\mathbb{R}^n\right)$ such that $T(\phi)=0$ for all $u \in C_0^{\infty}(\Omega)$ but $T(u) \neq 0$. Since $\supp(T) \subset \Omega^c$ but $T \neq 0, \Omega^c$ cannot be $\left(m, p^{\prime}\right)$-polar.
\end{proof}

As a final preparation for our investigation of the possible identity of $W_0^{m, p}(\Omega)$ and $W^{m,p}(\Omega)$ we establish a distributional analog of the fact, obvious for differentiable functions, that the vanishing of first derivatives over a rectangle implies constancy on that rectangle. We extend this first to distributions (in Corollary~3.30) and then to locally integrable functions.


\begin{lemma}
  Let $B=\left(a_1, b_1\right) \times \cdots \times\left(a_n, b_n\right)$ be an open rectangular box in $\mathbb{R}^n$ and let $\phi \in \mathscr{D}(B)$. If $\int_B \phi(x) d x=0$, then $\phi(x)=\sum_{j=1}^n \phi_j(x)$, where $\phi_j \in \mathscr{D}(B)$ and
  \begin{equation}\label{eq:3.8}
    \int_{a_j}^{b_j} \phi_j\left(x_1, \ldots, x_j, \ldots, x_n\right) d x_j=0
  \end{equation}
  for every fixed $\left(x_1, \ldots, x_{j-1}, x_{j+1}, \ldots, x_n\right) \in \mathbb{R}^{n-1}$.
\end{lemma}

\begin{proof}
  For $1 \leq j \leq n$ select functions $u_j \in C_0^{\infty}\left(a_j, b_j\right)$ such that $\int_{a_j}^{b_j} u_j(t) d t=1$. For $2 \leq j \leq n$, let
  \[
  \begin{aligned}
  B_j & =\left(a_j, b_j\right) \times\left(a_{j+1}, b_{j+1}\right) \times \cdots \times\left(a_n, b_n\right), \\
  \psi_j\left(x_j, \ldots, x_n\right) & =\int_{a_1}^{b_1} d t_1 \int_{a_2}^{b_2} d t_2 \cdots \int_{a_{j-1}}^{b_{j-1}} \phi\left(t_1, \ldots, t_{j-1}, x_j, \ldots, x_n\right) d t_{j-1}, \\
  \omega_j(x) & =u_1\left(x_1\right) \cdots u_{j-1}\left(x_{j-1}\right) \psi_j\left(x_j, \ldots, x_n\right)
  \end{aligned}
  \]
  Then $\psi_j \in \mathscr{D}\left(B_j\right)$ and $\omega_j \in \mathscr{D}(B)$. Moreover
  \[
  \int_{B_j} \psi_j\left(x_j, \ldots, x_n\right) d x_j \cdots d x_n=\int_B \phi(x) d x=0
  \]
  Let $\phi_1=\phi-\omega_2, \phi_j=\omega_j-\omega_{j+1}$ if $2 \leq j \leq n-1$, and $\phi_n=\omega_n$. Clearly $\phi_j \in \mathscr{D}(B)$ for $1 \leq j \leq n$, and $\phi=\sum_{j=1}^n \phi_j$. Finally,
  \[
  \begin{aligned}
  & \int_{a_1}^{b_1} \phi_1\left(x_1, \ldots, x_n\right) d x_1 \\
  & =\int_{a_1}^{b_1} \phi\left(x_1, \ldots, x_n\right) d x_1-\psi_2\left(x_2, \ldots, x_n\right) \int_{a_1}^{b_1} u_1\left(x_1\right) d x_1=0 \\
  & \int_{a_j}^{b_j} \phi_j\left(x_1, \ldots, x_n\right) d x_j \\
  & =u_1\left(x_1\right) \cdots u_{j-1}\left(x_{j-1}\right) \\
  & \times\left(\int_{a_j}^{b_j} \psi_j\left(x_1, \ldots, x_n\right) d x_j-\psi_{j+1}\left(x_{j+1}, \ldots, x_n\right) \int_{a_j}^{b_j} u_j\left(x_j\right) d x_j\right) \\
  & =0, \quad 2 \leq j \leq n-1 \\
  & \int_{a_n}^{b_n} \phi_n\left(x_1, \ldots, x_n\right) d x_n=u_1\left(x_1\right) \cdots u_{n-1}\left(x_{n-1}\right) \int_{a_n}^{b_n} \psi_n\left(x_n\right) d x_n \\
  & =u_1\left(x_1\right) \cdots u_{n-1}\left(x_{n-1}\right) \int_B \phi(x) d x=0 \\
  &
  \end{aligned}
  \]
\end{proof}


\begin{corollary}
  If $T \in \mathscr{D}^{\prime}(B)$ and $D_j T=0$ for $1 \leq j \leq n$, then there exists a constant $k$ such that for all $\phi \in \mathscr{D}(B)$
  \[
  T(\phi)=k \int_B \phi(x) d x
  \]
\end{corollary}

\begin{proof}
  First note that if $\int_B \phi(x) d x=0$, then $T(\phi)=0$, for, by the above lemma we may write $\phi=\sum_{j=1}^n \phi_j$, where $\phi_j \in \mathscr{D}(B)$ satisfies $(8)$, and hence $\phi_j=D_j \theta_j$, where $\theta_j$ defined by
  \[
  \theta_j(x)=\int_{a_j}^{x_j} \phi_j\left(x_1, \ldots, x_{j-1}, t, x_{j+1}, \ldots, x_n\right) d t
  \]
  belongs to $\mathscr{D}(B)$. Thus $T(\phi)=\sum_{j=1}^n T\left(D_j \theta_j\right)=-\sum_{j=1}^n\left(D_j T\right)\left(\theta_j\right)=0$.
  Now suppose $T \neq 0$. Then there exists $\phi_0 \in \mathscr{D}(B)$ such that $T\left(\phi_0\right)=k_1 \neq 0$. Thus $\int_B \phi_0(x) d x=k_2 \neq 0$ and $T\left(\phi_0\right)=k \int_B \phi_0(x) d x$, where $k=k_1 / k_2$. If $\phi \in \mathscr{D}(B)$ is arbitrary, let $K(\phi)=\int_B \phi(x) d x$. Then
  \[
  \int_B\left(\phi(x)-\frac{K(\phi)}{k_2} \phi_0(x)\right) d x=0
  \]
  and so $T\left(\phi-\left[K(\phi) / k_2\right] \phi_0\right)=0$. It follows that
  \[
  T(\phi)=\frac{T\left(\phi_0\right)}{k_2} K(\phi)=k K(\phi)=k \int_B \phi(x) d x .
  \]
\end{proof}

Note that this corollary can be extended to any connected set $\Omega \in \mathbb{R}^n$ via a partition of unity for $\Omega$ subordinate to some open cover of $\Omega$ by open rectangular boxes that are contained in $\Omega$. We do not, however, require this extension.

The following lemma shows that different locally integrable functions on an open set $\Omega$ determine different distributions on $\Omega$.

\begin{lemma}
  Let $u \in L_{\mathrm{loc}}^1(\Omega)$ satisfy $\int_{\Omega} u(x) \phi(x) d x=0$ for every $\phi$ in $\mathscr{D}(\Omega)$. Then $u(x)=0$ a.e. in $\Omega$.
\end{lemma}

\begin{proof}
  If $\psi \in C_0(\Omega)$, then for sufficiently small positive $\varepsilon$, the mollifier $J_\varepsilon * \psi$ belongs to $\mathscr{D}(\Omega)$. By Lemma $2.29, J_\varepsilon * \psi \rightarrow \psi$ uniformly on $\Omega$ as $\varepsilon \rightarrow 0+$. Hence $\int_{\Omega} u(x) \psi(x) d x=0$ for every $\psi \in C_0(\Omega)$.
  
  Let $K \subset\subset \Omega$ and let $\varepsilon>0$. Let $\chi_K$ be the characteristic function of $K$. Then $\int_K|u(x)| d x<\infty$. There exists $\delta>0$ such that for any measurable set $A \subset K$ with $\mu(A)<\delta$ we have $\int_A|u(x)| d x<\varepsilon / 2$ (see, for example, [Ru2, p. 124]). By Lusin's theorem 1.42 (f) there exists $\psi \in C_0\left(\mathbb{R}^n\right)$ with $|\psi(x)| \leq 1$ for all $x$, such that
  \[
  \mu\left(\left\{x \in \mathbb{R}^n: \psi(x) \neq \chi_K(x) \sgn \overline{u(x)}\right\}\right)<\delta .
  \]
  Here
  \[
  \sgn v(x)= \begin{cases}v(x) /|v(x)| & \text { if } v(x) \neq 0 \\ 0 & \text { if } v(x)=0\end{cases}
  \]
  Hence
  \[
  \begin{aligned}
  \int_K|u(x)| d x & =\int_{\Omega} u(x) \chi_K(x) \sgn \overline{u(x)} d x \\
  & =\int_{\Omega} u(x) \psi(x) d x+\int_{\Omega} u(x)\left(\chi_K(x) \sgn \overline{u(x)}-\psi(x)\right) d x \\
  & \leq 0+2 \int_{\left\{x \in \Omega: \psi(x) \neq \chi_K(x) \sgn \overline{u(x)\}}\right.}|u(x)| d x<\varepsilon .
  \end{aligned}
  \]
  Since $\varepsilon$ is arbitrary, $u(x)=0$ a.e.~in $K$ for each such $K$,
  and hence a.e.~in $\Omega$.
\end{proof}


\begin{corollary}
  If $B$ is a rectangular box as in Lemma 3.29 and $u$ in $L_{\text {loc }}^1(B)$ possesses weak derivatives $D_j u=0$ for $1 \leq j \leq n$, then for some constant $k, u(x)=k$ a.e. in $B$.
\end{corollary}

\begin{proof}
  By Corollary 3.30 , since $D_j T_u=0$ for $1 \leq j \leq n$, we have
  \[
  \int_B u(x) \phi(x) d x=T_u(\phi)=k \int_B \phi(x) d x
  \]
  Hence $u(x)-k=0$ a.e. in $B$.
\end{proof}


\begin{theorem}
  Let $m \geq 1$.
  \begin{enumerate}[label = (\alph*)]
    \item If $W^{m,p}(\Omega)=W_0^{m, p}(\Omega)$, then $\Omega^c$ is $\left(m, p^{\prime}\right)$-polar.
    \item If $\Omega^c$ is both $(1, p)$-polar and $\left(m, p^{\prime}\right)$-polar, then $W^{m,p}(\Omega)=W_0^{m, p}(\Omega)$.
  \end{enumerate}
\end{theorem}

\begin{proof}
  (a) Assume $W^{m,p}(\Omega)=W_0^{m, p}(\Omega)$. We deduce first that $\Omega^c$ must have measure zero. If not, there would exist some finite open rectangle $B \subset \mathbb{R}^n$ which intersects both $\Omega$ and $\Omega^c$ in sets of positive measure. Let $u$ be the restriction to $\Omega$ of a function in $C_0^{\infty}\left(\mathbb{R}^n\right)$ which is identically one on $B \cap \Omega$. Then $u \in W^{m,p}(\Omega)$ and so $u \in W_0^{m, p}(\Omega)$. By Lemma 3.27, the zero extension $\tilde{u}$ of $u$ to $\mathbb{R}^n$ belongs to $W^{m,p}\left(\mathbb{R}^n\right)$ and $D_j \tilde{u}=\widetilde{D_j u}$ in the distributional sense in $\mathbb{R}^n$ for $1 \leq j \leq n$. Now $\widetilde{D_j u}$ is identically zero on $B$ and so $D_j \tilde{u}=0$ as a distribution on $B$. By Corollary 3.32, $\tilde{u}$ must have a constant value a.e. in $B$. Since $\tilde{u}=1$ on $B \cap \Omega$ and $\tilde{u}=0$ on $B \cap \Omega^c$, we have a contradiction. Thus $\Omega^c$ has measure zero.
  
  Now if $v \in W^{m,p}\left(\mathbb{R}^n\right)$ and $u$ is the restriction of $v$ to $\Omega$, then $u$ belongs to $W^{m,p}(\Omega)$ and hence, by assumption, also to $W_0^{m, p}(\Omega)$. By Lemma 3.27, $\tilde{u} \in W^{m,p}\left(\mathbb{R}^n\right)$ and can be approximated by elements of $C_0^{\infty}(\Omega)$. But $v(x)=\tilde{u}(x)$ on $\Omega$, that is, a.e. in $\mathbb{R}^n$. Hence $v$ and $\tilde{u}$ have the same distributional derivatives, and coincide in $W^{m,p}\left(\mathbb{R}^n\right)$. Therefore $C_0^{\infty}(\Omega)$ is dense in $W^{m,p}\left(\mathbb{R}^n\right)$ and $\Omega^c$ is $\left(m, p^{\prime}\right)$-polar by Theorem 3.28.
  
  (b) Now assume $\Omega^c$ is $(1, p)$-polar and $\left(m, p^{\prime}\right)$-polar. Let $u \in W^{m,p}(\Omega)$. We show that $u \in W_0^{m, p}(\Omega)$. Since $\tilde{u} \in L^p\left(\mathbb{R}^n\right)$, the distribution $T_{D_j \tilde{u}}$, corresponding to $D_j \tilde{u}$, belongs to $W^{-1, p}\left(\mathbb{R}^n\right)$. Since $\widetilde{D_j u} \in L^p\left(\mathbb{R}^n\right) \subset H^{-1, p}\left(\mathbb{R}^n\right)$ (see Paragraph 3.13), therefore $T_{\widetilde{D_j u}} \in W^{-1, p}\left(\mathbb{R}^n\right)$. Hence $T_{D_j \tilde{u}-\widetilde{D_j u}} \in W^{-1, p}\left(\mathbb{R}^n\right)$. But $D_j \tilde{u}-\widetilde{D_j u}=0$ on $\Omega$ so $\supp\left(T_{D_j \tilde{u}-\widetilde{D_j u}}\right) \subset \Omega^c$. Since $\Omega^c$ is $(1, p)$ polar, $D_j \tilde{u}=\widetilde{D_j u}$ in the distributional sense on $\mathbb{R}^n$, whence $D_j \tilde{u} \in L^p\left(\mathbb{R}^n\right)$ and $\tilde{u} \in W^{m,p}\left(\mathbb{R}^n\right)$. Since $\Omega^c$ is $\left(m, p^{\prime}\right)$-polar, $C_0^{\infty}(\Omega)$ is dense in $W^{m,p}\left(\mathbb{R}^n\right)$, and thus $u \in W_0^{m, p}(\Omega)$
\end{proof}

\begin{para}
  If $\left(m, p^{\prime}\right)$-polarity implies $(1, p)$-polarity, then Theorem 3.33 amounts to the assertion that $\left(m, p^{\prime}\right)$-polarity of $\Omega^c$ is necessary and sufficient for the equality of $W^{m,p}(\Omega)$ and $W_0^{m, p}(\Omega)$. This is certainly the case if $p=2$.
\end{para}

The following two lemmas develop properties of polarity. The first of these shows that it is a local property.


\begin{lemma}
  $F \subset \mathbb{R}^n$ is $\left(m, p^{\prime}\right)$-polar if and only if $F \cap K$ is $\left(m, p^{\prime}\right)$-polar for every compact set $K \subset \mathbb{R}^n$.
\end{lemma}

\begin{proof}
  CLearly the $(m,p')$-polarity of $F$ implies that of $F\cap K$
  for every compact $K$. We need only prove the converse.

  Let $T\in W^{-m,p'}(\mathbb{R}^n)$ be given by $T  = \sum_{0\leq |\alpha|\leq m} (-1)^{|\alpha|}
  D^\alpha T_{v_{\alpha}}$, where sequence $\{v_\alpha\}\subset L^{p'}(\mathbb{R}^n)$.
  Suppose $T$ has support in $F$. We must show that $T = 0$.
  Let $f\in C_0^\infty(\mathbb{R}^n)$ satisfy $f(x) = 1$ if $|x|\leq 1$
  and $f(x) = 0$ if $|x|\geq 2$. For $\varepsilon > 0$, let $f_\varepsilon (x) = f(\varepsilon x)$
  so that $D^\alpha f_\varepsilon (x) = \varepsilon^{|\alpha|} D^\alpha f(\varepsilon x)\to 0$
  uniformly in $x$ as $\varepsilon\to 0+$.
  Then $f_\varepsilon T \in W^{-m,p'}(\mathbb{R}^n)$ bt induction on $m$,
  and for any $\phi\in\mathscr{D}(\mathbb{R}^n)$ we have
  \begin{align*}
    |T(\phi) - f_\varepsilon T(\phi)|
    & = |T(\phi) - T(f_\varepsilon \phi)| \\
    & = \Biggl|\sum_{0\leq |\alpha|\leq m} \int_{\mathbb{R}^n} v_\alpha(x)
        D^\alpha \bigl[\phi(x)\bigl(1 - f_\varepsilon (x)\bigr)\bigr] \d x \Biggr| \\
    & = \Biggl|\sum_{0\leq |\alpha|\leq m} \sum_{\beta\leq\alpha}
        \binom{\alpha}{\beta} \int_{\mathbb{R}^n} v_\alpha(x)
        D^\beta \phi(x) D^{\alpha-\beta} \bigl(1 - f_\varepsilon(x)\bigr) \d x \Biggr| \\
    & \leq \sum_{\beta\leq\alpha} \int_{\mathbb{R}^n} |w_\beta (x) D^\beta \phi(x)| \d x
      \leq \|\phi\|_{m,p} \bigl\|w; L^{P'}(\Omega^{(m)}) \bigr\|,
  \end{align*}
  where
  \[
    \begin{aligned}
    w_\beta(x) & =\sum_{|\alpha| \leq m, \beta \leq \alpha}\left(\begin{array}{c}
    \alpha \\
    \beta
    \end{array}\right) v_\alpha(x) D^{\alpha-\beta}\left(1-f_\varepsilon(x)\right) \\
    & =v_\beta(x)\left(1-f_\varepsilon(x)\right)-\sum_{|\alpha| \leq m, \beta \leq \alpha, \beta \neq \alpha}\left(\begin{array}{c}
    \alpha \\
    \beta
    \end{array}\right) v_\alpha(x) D^{\alpha-\beta} f_\varepsilon(x) .
    \end{aligned}
  \]
  Since $f_\varepsilon(x)=1$ for $|x| \leq 1 / \varepsilon$, we have $\lim _{\varepsilon \rightarrow 0+}\left\|w_\beta\right\|_{p^{\prime}}=0$. Thus $f_\varepsilon T \rightarrow T$ in $W^{-m, p^{\prime}}\left(\mathbb{R}^n\right)$ as $\varepsilon \rightarrow 0+$. But $f_\varepsilon T=0$ by assumption since it has compact support in $K$. Thus $T=0$.
\end{proof}


\begin{lemma}
  If $p^{\prime}<q^{\prime}$ (that is, $p>q$) and $f \subset \mathbb{R}^n$
  is $\left(m, p^{\prime}\right)$-polar, then $F$ is also $\left(m, q^{\prime}\right)$-polar.
\end{lemma}
    
\begin{proof}
  Let $K \subset \mathbb{R}^n$ be compact. By the previous lemma it is sufficient to show that $F \cap K$ is $\left(m, q^{\prime}\right)$-polar. Let $G$ be an open, bounded set in $\mathbb{R}^n$ containing $K$. By Theorem $2.14, W_0^{m, p}(G) \rightarrow W_0^{m, q}(G)$, so that $W^{-m, q^{\prime}}(G) \subset W^{-m, p^{\prime}}(G)$. Any distribution $T \in W^{-m, q^{\prime}}\left(\mathbb{R}^n\right)$ having support in $K \cap F$ also belongs to $W^{-m, q^{\prime}}(G)$ and so to $W^{-m, p^{\prime}}(G)$. Since $K \cap F$ is $\left(m, p^{\prime}\right)$-polar, $T=0$. Thus $K \cap F$ is also $\left(m, q^{\prime}\right)$-polar.
\end{proof}


\begin{theorem}
  Let $m \geq 1$ and $p \geq 2$. Then $W^{m,p}(\Omega)=W_0^{m, p}(\Omega)$ if and only of $\Omega^c$ is $\left(m, p^{\prime}\right)$-polar.
\end{theorem}

\begin{proof}
  Since $p^{\prime} \leq 2, \Omega^c$ is $(m, p)$-polar and therefore also $(1, p)$-polar. The result now follows by Theorem 3.33.
\end{proof}

\begin{para}
  The Sobolev Imbedding Theorem 4.12 can be used to extend the previous theorem to cover certain 
  values of $p<2$. If $(m-1) p<n$, the imbedding theorem gives
  \[
  W^{m,p}\left(\mathbb{R}^n\right) \rightarrow W^{1, q}\left(\mathbb{R}^n\right), \quad q=\frac{n p}{n-(m-1) p},
  \]
  which in turn implies that $W^{-1 \cdot q^{\prime}}\left(\mathbb{R}^n\right) \subset W^{-m, p^{\prime}}\left(\mathbb{R}^n\right)$.
  If also $p \geq 2 n /(n+m-1)$, then $q^{\prime} \leq p$ and so by Lemma~3.36, $\Omega^c$
  is $(1, p)$-polar if it is $\left(m, p^{\prime}\right)$-polar. Note that $2 n /(n+m-1)<2$
  if $m>1$. If, on the other hand, $(m-1) p \geq n$, then $m p>n$, and, as pointed out in 
  Paragraph~3.25, $\Omega^c$ cannot be $\left(m, p^{\prime}\right)$-polar unless it is empty,
  in which case it is trivially $(1, p)$-polar.
  
  The only values of $p$ for which we do not know that the $\left(m, p^{\prime}\right)$-polarity 
  of $\Omega^c$ implies $(1, p)$-polarity and hence is equivalent to the identity of $W^{m,p}(\Omega)$ and $W_0^{m, p}(\Omega)$, are given by $1 \leq p \leq \min \{n /(m-1), 2 n /(n+m-1)\}$
\end{para}

\begin{para}
  Whenever $W_0^{m, p}(\Omega) \neq W^{m,p}(\Omega)$, the former space is a closed subspace of 
  the latter. In the Hilbert space case, $p=2$, we may consider the space $W_0^{\perp}$ consisting 
  of all $v \in W^{m, 2}(\Omega)$ such that $(v, \phi)_m=0$ for all $\phi \in C_0^{\infty}(\Omega)$.
  Every $u \in W^{m, 2}(\Omega)$ can be uniquely decomposed in the form $u=u_0+v$,
  where $u_0 \in W_0^{m, 2}(\Omega)$ and $v \in W_0^{\perp}$.
  Integration by parts shows that any $v \in W_0^{\perp}$ must satisfy
  \[
  \sum_{0 \leq|\alpha| \leq m}(-1)^{|\alpha|} D^{2 \alpha} v(x)=0
  \]
  in the weak sense, and hence a.e. in $\Omega$.
\end{para}


\section{Coordinate Transformations}

\begin{para}
  Let $\Phi$ be a one-to-one transformation of a domain $\Omega \subset \mathbb{R}^n$ onto
  a domain $G \in \mathbb{R}^n$, having inverse $\Psi=\Phi^{-1}$.
  We say that $\Phi$ is $m$-smooth if, when we write $y=\Phi(x)$ and $x=\Psi(y)$ in the form
  \[
  \begin{aligned}
  y_1 & =\phi_1\left(x_1, \ldots, x_n\right), & x_1 & =\psi_1\left(y_1, \ldots, y_n\right), \\
  y_2 & =\phi_2\left(x_1, \ldots, x_n\right), & x_2 & =\psi_2\left(y_1, \ldots, y_n\right), \\
  \vdots & \vdots & & \\
  y_n & =\phi_n\left(x_1, \ldots, x_n\right), & x_n & =\psi_n\left(y_1, \ldots, y_n\right),
  \end{aligned}
  \]
  then $\phi_1, \ldots, \phi_n$ belong to $C^m(\overline{\Omega})$ and $\psi_1, \ldots, \psi_n$
  belong to $C^m(\bar{G})$.
  If $u$ is a measurable function on $\Omega$, we define a measurable function $A u$ on $G$ by
  \begin{equation}\label{eq:3.9}
    A u(y)=u(\Psi(y)) .
  \end{equation}
  Suppose that $\Phi$ is 1-smooth so that there exist constants $0<c<C$ such that
  for all $x \in \Omega$
  \begin{equation}\label{eq:3.10}
    c \leq\left|\det \Phi^{\prime}(x)\right| \leq C,
  \end{equation}
  where $\Phi^{\prime}$ denotes the Jacobian matrix $\partial\left(y_1, \ldots, y_n\right) / \partial\left(y_1, \ldots, y_n\right)$.
  Since smooth functions are dense in $L^p$ spaces, the operator $A$ defined by (9)
  transforms $L^p(\Omega)$ boundedly onto $L^p(G)$ and has a bounded inverse;
  in fact, for $1 \leq p<\infty$,
  \[
  c^{1 / p}\|u\|_{p, \Omega} \leq\|A u\|_{p, G} \leq C^{1 / p}\|u\|_{p, \Omega} .
  \]
\end{para}


\begin{theorem}
  We establish a similar result for Sobolev spaces.
  Let $\Phi$ be $m$-smooth, where $m \geq 1$. The operator $A$ defined by (9)
  transforms $W^{m,p}(\Omega)$ boundedly onto $W^{m,p}(G)$ and has a bounded inverse.
\end{theorem}

\begin{proof}
  We show that the inequality $\|A u\|_{m, p, G} \leq$ const $\|u\|_{m, p, \Omega}$ holds for 
  every $u \in W^{m,p}(\Omega)$, the constant depending only on the transformation $\Phi$.
  The reverse inequality $\|A u\|_{m, p, G} \geq$ const $\|u\|_{m, p, \Omega}$ (with a different 
  constant) can be established similarly, using the inverse operator $A^{-1}$.
  By Theorem~3.17 for given $u \in W^{m,p}(\Omega)$, there exists
  a sequence $\left\{u_j\right\} \subset C^{\infty}(\Omega)$ converging to $u$
  in $W^{m,p}(\Omega)$-norm.
  For such smooth $u_j$ it is readily checked by induction on $|\alpha|$ that
  \begin{equation}\label{eq:3.11}
    D^\alpha\left(A u_j\right)(y)=\sum_{\beta \leq \alpha} M_{\alpha \beta}(y) A\left(D^\beta u_j\right)(y),
  \end{equation}
  where $M_{\alpha \beta}$ is a polynomial of degree not exceeding $|\beta|$ in derivatives of 
  orders not exceeding $|\alpha|$ of the various components of $\Psi$.
  If $\theta \in \mathscr{D}(G)$ integration by parts gives
  \begin{equation}\label{eq:3.12}
    (-1)^{|\alpha|} \int_G\left(A u_j\right)(y) D^\alpha \theta(y) d y=\sum_{\beta \leq \alpha} \int_G A\left(D^\beta u_j\right)(y) M_{\alpha \beta}(y) \theta(y) d y,
  \end{equation}
  or, replacing $y$ by $\Phi(x)$ and expressing the integrals over $\Omega$,
  \begin{align}
  & (-1)^{|\alpha|} \int_{\Omega} u_j(x)\left(D^\alpha \theta\right)(\Phi(x))\left|\det \Phi^{\prime}(x)\right| d x \notag \\
  & \quad=\sum_{\beta \leq \alpha} \int_{\Omega} D^\beta u_j(x) M_{\alpha \beta}(\Phi(x)) \theta(\Phi(x))\left|\det \Phi^{\prime}(x)\right| d x . \label{eq:3.13}
  \end{align}
  Since $D^\beta u_j \rightarrow u$ in $L^p(\Omega)$ for $|\beta| \leq m$,
  we can take the limit through (13) as $n \rightarrow \infty$ and hence obtain (12) with $u$ 
  replacing $u_j$. Thus (11) holds in the weak sense for any $u \in W^{m,p}(\Omega)$. Therefore
  \[
  \begin{aligned}
  \int_G\left|D^\alpha(A u)(y)\right|^p d y & \leq\left(\sum_{\beta \leq \alpha} 1\right)^p \max _{|\beta| \leq|\alpha|}\left(\left.\sup _{y \in G}\left|M_{\alpha \beta}\right| \int_G\left|\left(D^\beta u\right)\right|(\Psi(y))\right|^p d y\right) \\
  & \leq \text { const } \max _{|\beta| \leq|\alpha|} \int_{\Omega}\left|D^\beta u(x)\right|^p d x
  \end{aligned}
  \]
  from which it follows that $\|A u\|_{m, p, G} \leq$ const $\|u\|_{m, p, \Omega}$.
\end{proof}

Of special importance in later chapters is the case of the above theorem corresponding to 
nonsingular linear transformations $\Phi$ or, more generally, affine transformations
(compositions of nonsingular linear transformations and translations).
For such transformations det $\Phi^{\prime}(x)$ is a nonzero constant.